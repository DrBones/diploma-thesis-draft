Nano-Spintronics involves the manipulation and control of electrons and their spin in solid-state systems.\par
With the discovery of the \emph{giant magnetoresistance} by \textsc{Albert Fert} and \textsc{Peter Gr\"unberg} in 1988, the manipulation of electron transport properties by exploiting the spin degree of freedom became apparent. This breakthrough, awarded with the \textsc{Nobel} Prize in Physics in 2007, became the starting point for the study and manipulation of the spin in electronic devices \cite{evegeny2010spin}. A prominent device proposal is the \textsc{Datta-Das} \emph{spin field-effect transistor} \cite{datta:665}. The conductance of such mesoscopic systems was found to be quantized \cite{PhysRevLett.45.494} at low temperatures and high magnetic fields revealing its quantum mechanical origin. For finite temperatures the spin field-effect transistor has only recently been realized as a proof of concept \cite{Wunderlich24122010}.\par
Central to the realization of the spin field-effect transistor is the \emph{two-dimensional electron gas} (2DEG). The 2DEG describes a system of non-interacting electrons confined to a two-dimensional region by a highly localized potential well. In devices within the 2DEG spatial constrictions called \emph{quantum point contacts} are used to control carrier density and spin through the effects of quantum confinement. Quantum point contacts are subject to active fundamental research as well as current applications in detectors \cite{PhysRevB.67.161308} or quantum bits \cite{PhysRevA.57.120}.\par
In this thesis, a quantum mechanical simulator based on the \emph{non-equilibrium \gfnc{} formalism} is developed. The basic numerical computations are sped up using the \emph{recursive \gfnc{} algorithm} (RGA) and extended by a graph-based matrix-reordering scheme enabling the simulation of 2DEGs of arbitrary geometry.
The electron density $n$, spin density $s_z$ and conductance $G$ are computed for quantum point contacts of transformed geometry and non-colinear systems with rotated leads. The simulations show how changes in the geometry can be used to alter the local electron and spin densities in the device.
The shape of the quantum point contacts has a strong influence on the quantization of the conductance profile. In the investigated \textsc{Aharonov-Bohm} interferometer the spin polarization in the ring is determined by the positions of the leads. The access to microscopic quantities by the simulator can be used to investigate possibilities of spin control in 2DEGs.
