The \gfnc{} formulation of quantummechanics is not the only possible method to compute the transport properties. A selection of transport models will be presented here. A detailed discussion of the formulations presented in \cref{tab:comparison} can be found in \cite{Biegel97quantumelectronic}. The table lists the capabilities and numerical efficiency in a direct computation. It includes the classical \textsc{Boltzmann} Transport Equation (BTE), the \textsc{Schr\"odinger} Equation (SE), the Transfer Matrix (TM), Density Matrix (DM), \cgfnc{} and \textsc{Wigner} Function \cite{Pourfath2007Thesis} formulations.\par
\begin{table}[!ht]
\centering
\begin{tabulary}{\textwidth}{l c c c c c c}\toprule
& \multicolumn{6}{c}{Formulations}  \\ \cmidrule{2-7}
Characteristic &BTE& SE&TM&DM&GF&WF\\ \midrule
State Function Bases &yes&yes&yes&yes&yes&yes  \\
Far-From-Equilibrium &yes&yes&yes&yes&yes&yes  \\
Irreversibility      &yes&no&no&yes&yes&yes\\
Transient Simulation &yes&yes&no&yes&yes&yes \\
Absorbing Boundaries &yes&no&yes&yes&yes&yes \\
Computational Efficiency &3&4&4&4&2&3 \\
Intuitive State Function &5&3&4&3&2&4 \\\bottomrule
\end{tabulary}
\caption{Comparison of quantum system analysis approaches. In the ranking 5 = good and 1 = poor. The \textsc{Boltzmann} Transport Equation (BTE), \textsc{Schr\"odinger} Equation (SE), Transfer Matrix (TM), Density Matrix (DM), \textsc{Green}'s Function (GF) and \textsc{Wigner} Function (WF) are compared. Slightly modified from \textsc{Biegel}, see \protect\cite{Biegel97quantumelectronic}}.
\label{tab:comparison}
\end{table}
The most obvious method, the direct solution of the \sdg{} in a many-body theory, soon becomes unmanagebly complex for a rising number of carriers inside the conductor. Also absorbing boundary conditions and inelastic scattering have not been treated accurately \cite{Biegel97quantumelectronic} although development has not halted \cite{JApplPhys.69.7153}\cite{gullapalli:2971}.\par
A popular quantum-device simulation method capable of describing steady state transport in a fast and straight forward way is the so-called Transfer Matrix Method (TMM) \cite{MacKinnon2003}. The TMM is in theory easily implemented as only matrix multiplications are involved but exhibits numerical complications.\par
TMM formulations are in general unstable because possible evanescent waves will lead to unphysically fast rising matrix coefficients \cite{PhysRevB.38.9945}.
While the instability issues can be resolved using a redesigned TMM technique by T.Usuki \cite{PhysRevB.50.7615}\cite{PhysRevB.52.8244} as a direct descendent of the \sdg{} the TMM inherits the same theoretical limitations.\par
Quantum statistical formulations present a different approach. They offer an analogon to the solution of classical electronic systems via the \textsc{Boltzmann} transport equation.
Due to the nature of many-body systems a statistical state function even in the limit of non-interacting particles should present a quite accurate representation of the 2DEG. The choice of a \gfnc{} as a state function has the following advantages over a \textsc{Wigner} function or density matrix approach.
The more general formalism makes it more flexible and powerful because it also profits from advances in the solution to \gfnc{} which can be applied in a diverse field.  The application of all quantum statistical formulations is currently limited to systems of reduced size or resolution as the computations involved are highly sensitive to the discretization.\par
An advantage of the \gfnc{} method over other state functions in relation to the implementation on a computer is the development of fast and efficient algorithms suitable to compute all the desired observables \cite{JApplPhys.91.2343} in two or even three dimensions or in a parallel computing environment \cite{Drouvelis2006parallel}.
