A two-dimensional electron gas (2DEG) is characterized by the confinement of electron motion in one direction and free motion in the other two directions. 
Electron systems with high lateral mobility can be realized by, e.g. transistors, surfaces of suited materials \cite{PhysRevLett.12.271} or heterojunctions \cite{JVSTB.4.853}. Structures containing 2DEGs are very versatile as they are easily fabricated by molecular beam epitaxy and modified by ion beam litho\-graphy \cite{Ingram1995}\cite{Nowack2009Thesis} or can even be rolled up \cite{Vorob'ev2004171}. 
\begin{figure}[!h]
\centering
\subfloat[]{\label{fig:heterostructure}\includegraphics[scale=0.15]{images/heterostructure}} \quad\quad
\subfloat[]{\label{fig:2degseparated}\includegraphics[scale=0.2]{images/2DEG_separated}} \quad\quad
\subfloat[]{\label{fig:2degjoined}\includegraphics[scale=0.2]{images/2DEG_combined}}
\caption{Heterostrucure with 2DEG at material boundary.(a) 2DEG (blue surface) in a heterojunction between AlGaAs (orange layer) and GaAs (green layer).(b) Bandstructure of separated AlGaAs and GaAs layers with corresponding energy of the conductance band $E_C$, valance band $E_V$, and \textsc{Fermi} energy $E_F$. The arrows denote the difference of energy of the conductance and valance bands.(c) Bending of conductance and valence bands of joined layers. The arrow denotes the position of the 2DEG below the \textsc{Fermi} energy. Slightly modified from \textsc{Datta} \cite{Datta1997}.}
\label{fig:hetero2deg}
\end{figure}
The 2DEG forms in the interior of the heterostructure between two layers, see \cref{fig:heterostructure}. In the case of heterojunctions at the interfaces of a heterostructure the tight confinement can be realized for example by joining two semiconducting materials with distinct band gaps and \textsc{Fermi} energies. When separated the band structures exhibit distinct \textsc{Fermi} energies, \cref{fig:2degseparated}. When the materials are joined the conduction and valance bands bend at the interface to match the respective \textsc{Fermi} energies of both materials, as illustrated in \cref{fig:2degjoined}.
The bending of the conductance band below the \textsc{Fermi} energy of the combined heterostructure creates a very pronounced dip in the bandstructure resulting in a potential well where the electrons can move freely. The confinement perpendicular to the 2DEG however leads to quantized energy levels of motion in $z$ direction as can be seen in \cref{fig:potentialwell} (a) and (b) for three states with different cutoff energies, i.e. different band bottoms. The offset conduction bands can often be treated independently. Usually the lowest energy level comprises the 2DEG \cite{Datta1997}.
\begin{figure}[t]
\centering
\subfloat[]{\label{fig:modesinzconf1}\begin{tikzpicture}\node at (0,0) [above]{\includegraphics[scale=0.55]{images/modesinzconf1}};
\node at (0.7,-0.4){$z$};
\node at (-1.5,2){$n=1$};
\node at (-1.5,4.4){$n=2$};
\node at (-1.5,5.6){$n=3$};
\node at (-1.5,7.2){$E$};
\end{tikzpicture}}
\subfloat[]{\label{fig:modesinzconf2}\begin{tikzpicture}\node at (0,0) [above]{\includegraphics[scale=0.55]{images/modesinzconf2}};
\node at (0,-0.4){$\v{k}$};
\node at (0.7,1.5){$E_1(\v{k})$};
\node at (0.7,3.7){$E_2(\v{k})$};
\node at (0.7,6.9){$E_3(\v{k})$};
\draw[latex-latex] (-2,0.2)--node[right]{$E_g$}(-2,1.7);
\end{tikzpicture}}
% \includegraphics[width=0.9\textwidth]{images/davies.png}
\caption{Modes in $z$-confined potential well. (a) Eigenstate densities in a potential well for energy quantum number $n$ and cut-off energies $E_1$,$E_2$,$E_3$ in dependence on the position. (b) Conductance bands in the potential well in k-space. The ground state energy is denoted by $E_g$.}
\label{fig:potentialwell}
\end{figure}
