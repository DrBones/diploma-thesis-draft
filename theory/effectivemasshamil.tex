The wavefunction and eigenvalues of the electrons in the conduction band describe the properties of the 2DEG. In the 2DEG the wavefuctions can be separated into a lateral ($x,y$) and perpendicular ($z$) part. Each perpendicular wavefunction belongs to a different cut-off energy $E_{\text{cut-off}}$, see \cref{fig:potentialwell}(b).
In steady state, the dynamics of the electrons in the separated lateral band can be described by the time independent \sdg{}.
\begin{align}
 \hat{H}_0\ket{\psi(x,y)} = E \ket{\psi(x,y)}\ .
	\label{eqn:effectivemasssdg}
\end{align}
The two dimensional single-band effective mass \hamil{} of orbital motion $\hat{H}_0$
\begin{align}
\hat{H}_0 = E_{\text{cut-off}} + \frac{1}{2m^*}(\hat{p}_{x}+\hat{p}_{y})^2+U(x,y)\ ,
\end{align}
includes the cut-off energy $E_{\text{cut-off}}$, the effective mass $m^*$, the momentum operator $\hat{p}_i = i\hbar \pd{}{i} + e\mat{A}$ with an appropriate vector field $\v{A}(x,y)$, the electron charge $e$, and the static potential $U(x,y)$.
The lateral geometry is included via the static potential $U(x,y)$.
This approximation yields smoothed out solutions due to the inclusion of the periodic lattice potential via an effective mass $m^*$ \cite{BastardBrum1986}.
The cut-off energy $E_{\text{cut-off}}$ includes the energy shift of the lowest occupied subband of the 2DEG and will be set to $E_{\text{cut-off}} = 0$ in all further calculations.
From the solutions of the single-band effective mass \sdg{} one can extract physical quantities suitable for comparison with experimental results called observables.
A central value of interest is the probability distribution
\begin{align}
	n(x,y) = \left< \psi^{\dagger} (x,y) \psi(x,y)\right>\ .
	\label{eqn:analyticalelectrondensity}
\end{align}
of the wavefunctions $\psi$. Here the expectation value has to be taken over the product of the wavefunctions.\par
