The quantummechanical values describing the properties of the 2DEG are the wavefunction and eigenvalues of the electrons in the conduction band. In a confined system like the 2DEG the wavefuctions can be separated into a lateral ($x,y$) and perpendicular ($z$) part. Each perpendicular wavefunction belongs to a different cut-off energy $E_{\text{cut-off}}$ in \cref{fig:potentialwell}.
In steady state, the dynamics of the electrons in the separated lateral band can be described by the time independend \sdg{}. Written in the second quantization as
\begin{align}
 \hat{H}_0\ket{\psi(x,y)} = E \ket{\psi(x,y)}.
	\label{eqn:effectivemasssdg}
\end{align}
With $\hat{p}_i = i\hbar \pd{}{i} + e\mat{A}$ being the momentum operator. As well as an appropriate vector field $\v{A}(x,y)$ and electron charge $e$ if magnetic fields are to be included. The two dimensional single-band effective mass \hamil{} of orbital motion $\hat{H}_0$ can be written as
\begin{align}
\hat{H}_0 = E_{\text{cut-off}} + \frac{1}{2m^*}(\hat{p}_{x}+\hat{p}_{y})^2+U(x,y).
\end{align}
The lateral geometry or any other static potential is included via the term $U(x,y)$.
This approximation yields smoothed out solutions due to the inclusion of the periodic lattice potential via an effective mass $m^*$ \cite{BastardBrum1986}.
The cut-off energy $E_{\text{cut-off}}$ includes the energy shift of the lowest occupied subband of the 2DEG and will be set to $E_{\text{cut-off}} = 0$ in all further calculations.
From the solutions of the single-band effective mass \sdg{} one can extract physical quantities suitable for comparison with experimental results called observables.
A central value of interest is the probability distribution of the electrons:
\begin{align}
	n(x,y) = \left< \psi^+ (x,y) \psi(x,y)\right>
	\label{eqn:analyticalelectrondensity}
\end{align}
Here the expectation value has to be taken over the product of the wavefunction.
