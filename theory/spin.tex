The dynamics of the electrons in the conductance band are subject to numerous quantummechanical phenomena and interactions which may be included via additive pertubations making up the single particle single-band effective mass \hamil{}:
\begin{align}
\hat{H}_{\text{spem}} = \hat{H}_0 + \hat{H}_{\text{int}}+\dotsb \label{eqn:generalhamil}
\end{align}
\todo[noline]{find better name than "spem"}
One important aspect to obtain a more realistic picture is the inclusion of the effects of the internal spin degree of freedom of the electrons.
\subsection{Spin Degree of Freedom}
The inclusion of the electron spin and associated effects ``blows up" the phase-space to be considered by two, reflecting the two possible spin orientations with the $2 \times 2$ spin identity matrix $\mat{I}_S$. 
\begin{align}
\hat{H} \rightarrow \hat{H} \otimes \mat{I}_S
\end{align}
When a particle in vacuum moves through an electric field its orbital and spin degrees of freedom become coupled. This essentially relativistic effect is known as spin-orbit interaction. If the \textsc{Dirac} equation governing this sort of interaction is non-relativistically expanded in powers of $1/c$ one finds as a first order correction\cite{Nowack2009Thesis}:
\begin{align}
\hat{H}_{SO} = \frac{1}{2m_0c^2} \v{\sigma} \otimes \left( \nabla V \times \frac{\v{p}}{m_0}\right)
\label{eqn:spinorbithamil}
\end{align}
Here $m_0$ is the electron rest mass, $\v{\sigma} = (\sigma_x, \sigma_y,\sigma_z)$ the spin vector of pauli-matrices and $V$ describing the electric field. 
\subsection{Spin Orbit Interaction}
For the movement of electrons in a crystal lattice the same phenomenon gives rise to spin-orbit interactions as the electron moves through the field of the atomic cores.
The potential that is responsible for the $z$-confinement of the 2DEG also interacts with the lateral movement of the electrons. Because of its nonzero gradient if put into \cref{eqn:spinorbithamil} it will give rise to the so called \rash{} \hamil{}:
\begin{align}
  \hat{H}_{RSO} =\frac{\alpha_{RSO}}{\hbar}(\hat{p}_{y} \otimes \hat{\sigma}_{x} - \hat{p}_{x} \otimes \hat{\sigma}_{y})
	\label{eqn:rashbahamiltonian}
\end{align}
The parameter $\alpha_{RSO}$ governs the strength of the interaction and has its origin in the lack of structural inversion symmetry of the boundary layer.
As can be seen in \cref{fig:hetero2deg} the confining potential is not strictly symmetric leading in addition to finite-wall effects of the potential well to a material constant defined by\cite{Metalidis2007Thesis}:
\begin{align}
\alpha_{RSO} = \frac{\hbar}{2m_0c^2} \left<\pd{V}{z} \right> 
\end{align}
The $\left< \right>$ denoting a suitable averaging procedure arising from the spatial dependence of $V$. The average is quite intricate as the effect depends on the detailed bandstructure including also the influence of holes\cite{JApplPhys.83.4324}.
As the \rash{} effect is a suitable candidate to realize certain spintronic applications it is worth noting that the strength of $\alpha_{RSO}$ and therefore the properties of conduction can be tuned by the addition of a gate electrode on top of the 2DEG\cite{PhysRevLett.78.1335}.
Although other corrections to the single-band effective-mass \hamil{} like the \textsc{Dresselhaus} spin-orbit coupling \cite{PhysRev.100.580} may be considered in small band-gap materials like GaAs and InAs the spin-orbit coupling is largely dominated by the \rash{} effect\cite{PhysRevB.61.15588}. As this work focuses on the 2DEG embedded in the aforementioned materials no further contributions to the \hamil{} will be added.\par
With the effects of the inherent spin degree of electrons considered it is now possible to compute additional observables..
The spatial probability distribution of up or down pointing electron-spin in respect to an arbitrary but fixed quantization axis $i$ is for example given by the spin-density $S^i$ \cite{JPhysA:MathGen.18.671}:
\begin{align}
S^i  = \frac{\hbar}{2} \left< \psi^+ \hat{\sigma}_i \psi \right>
\label{eqn:spindensity}
\end{align}
The $\hat{\sigma}_i$ is the spin-operator in the corresponding direction, i.e. in matrix representation a \textsc{Pauli} matrix. The arguments of the wavefunctions will be suppressed for brevity from now on.
