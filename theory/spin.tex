The dynamics of the electrons in the conductance band are subject to numerous quantum mechanical phenomena and interactions which may be included via additive influences like a spin interaction \hamil{} $H_{\text{int}}$ making up the single particle single-band effective mass \hamil{}
\begin{align}
\hat{H} = \hat{H}_0 + \hat{H}_{\text{int}}+\dotsb \label{eqn:generalhamil}\ .
\end{align}
To obtain a more realistic picture than the spin degenerate system the effects of the internal spin degree of freedom of the electrons are included.
\subsection{Spin Degree of Freedom}
The inclusion of the electron spin degree of freedom is achieved by the \textsc{Kronecker} multiplication of the single particle single-band effective mass \hamil{}  with the $2 \times 2$ spin identity matrix $\mat{I}_S$. 
\begin{align}
\hat{H} \rightarrow \hat{H} \otimes \mat{I}_S\ .
\end{align}
When a particle in vacuum moves through an electric field its orbital and spin degrees of freedom are coupled. This essentially relativistic effect is known as spin-orbit interaction. If the \textsc{Dirac} equation is non-relativistically expanded in powers of the inverse of the speed of light $c$, one finds as a first order correction \cite{Nowack2009Thesis}
\begin{align}
\hat{H}_{SO} = \frac{1}{2m_0c^2} \v{\sigma} \otimes \left( \nabla V \times \frac{\v{p}}{m_0}\right)\ .
\label{eqn:spinorbithamil}
\end{align}
Here, $m_0$ is the electron rest mass, $\v{\sigma} = (\sigma_x, \sigma_y,\sigma_z)$ the spin vector of \textsc{Pauli}-matrices and $V$ describes the electric field. 
\subsection{Spin Orbit Interaction}
For the motion of electrons in a crystal lattice the relativistic interaction of electrons and their spin give rise to spin-orbit interactions as the electron moves through the field of the atomic cores.
The potential that is responsible for the $z$-confinement of the 2DEG interacts with the lateral motion of the electrons. The nonzero gradient of the effective electrical field $V$ in \cref{eqn:spinorbithamil} gives rise to the \rash{} \hamil{}:
\begin{align}
  \hat{H}_{RSO} =\frac{\alpha_{RSO}}{\hbar}(\hat{p}_{y} \otimes \hat{\sigma}_{x} - \hat{p}_{x} \otimes \hat{\sigma}_{y})\ .
	\label{eqn:rashbahamiltonian}
\end{align}
The parameter $\alpha_{RSO}$ governs the strength of the interaction and has its origin in the lack of structural inversion symmetry of the boundary layer \cite{PhysRevB.70.233311}.
As can be seen in \cref{fig:hetero2deg} the confining potential is not strictly symmetric leading in addition to finite-wall effects of the potential well to a material constant \cite{Metalidis2007Thesis}
\begin{align}
\alpha_{RSO} = \frac{\hbar}{2m_0c^2} \left<\pd{V}{z} \right>\ .
\label{eqn:alpharashba}
\end{align}
The term $\left<\pd{V}{z} \right>$ denotes a spatial average required by the spatial dependence of $V$. The average is quite intricate as the effect depends on the detailed bandstructure including also the influence of holes \cite{JApplPhys.83.4324}.
As the \rash{} effect is a suitable candidate to realize certain spintronic applications it is worth noting that the strength of the \textsc{Rashba} parameter $\alpha_{RSO}$ and therefore the properties of conduction can be tuned by the addition of a gate electrode on top of the 2DEG altering the $z$ confinement potential \cite{PhysRevLett.78.1335}.
Although other corrections to the single-band effective-mass \hamil{} like the \textsc{Dresselhaus} spin-orbit coupling \cite{PhysRev.100.580} may be considered in small band-gap materials like GaAs and InAs the spin-orbit coupling is largely dominated by the \rash{} effect \cite{PhysRevB.61.15588}. As this work focuses on 2DEGs in such materials no further contributions to the \hamil{} will be added.\par
With the effects of the inherent spin degree of electrons considered it is now possible to compute additional observables. The spatial probability distribution of up or down pointing electron-spin with respect to an arbitrary but fixed quantization axis $i$ is for example given by the spin-density \cite{JPhysA:MathGen.18.671}
\begin{align}
S^i  = \frac{\hbar}{2} \left< \psi^{\dagger} \hat{\sigma}_i \psi \right>\ ,
\label{eqn:spindensity}
\end{align}
which includes the spin-operator $\hat{\sigma}_i$ in the corresponding direction, i.e. in matrix representation a \textsc{Pauli} matrix and the \textsc{Pauli} spinors $\psi,\psi^{\dagger}$. The arguments of the wavefunctions will be suppressed for brevity from now on.
