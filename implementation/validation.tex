Due to the lack of proper standard problems regarding the validity of a quantum mechanical simulator for transport properties the outcome of the numerical calculations via the \gfnc{} method will be compared to a selection of simple problems and experimental results.\par
For the case of a quantum wire the probability disrtibution can be calculated analytically. The quantum wire is simply a one-dimensional potential well in $y$ direction which ideally extends infinitely in the $x$ direction.
Because of this the wavefunction can be separated into transverse and longitudinal parts i.e:
\begin{align}
\Psi(x,y) = \sqrt{2/L}\text{ sin}(n\pi/L \cdot y) \cdot e^{ik_xx}
\end{align}
Because of the transverse confinement the eigenenergies become discrete similar to the case of the $z$ confinement illustrated in fig. (\ref{fig:potentialwell}).
\todo[noline]{average electro density for 100nm wire is about $10^17$ so that should fit}
The analytically calculated probability distribution is renormalized to experimentally measured electron densities $n \sim 10^{17}/m^2$ typical for a 2DEG embedded in heterojunctions containing InAs\,\cite{gelfand2006}\,\cite{JJAP.26.L59}
In figures (\dots ) the analytical calculation for the first three modes is superimposed with the simulated results for a nano-wire of 100nm. 

effective mass \cite{PhysRev.105.460}
\begin{figure}[t]
  \begin{center}
    % \showthe\columnwidth % Use this to determine the width of the figure.
\subfloat[Analytical wavefunction and probability distribution]{\label{fig:analytical1}\includegraphics[width=210pt]{images/analytical1}} \qquad
    \subfloat[Simulated electron density cut]{\includegraphics[width=210pt]{images/overlay1}}\\
    \subfloat[Relative error]{\includegraphics[width=210pt]{images/error1}}\qquad
    \subfloat[Simulated electron density]{\includegraphics[width=210pt]{images/dens1}}
    \caption{Comparison of analytical and simulated electron densities for the first mode.}
  \end{center}
\end{figure}
\begin{figure}[t]
  \begin{center}
    % \showthe\columnwidth % Use this to determine the width of the figure.
\subfloat[Analytical wavefunction and probability distribution]{\label{fig:analytical2}\includegraphics[width=210pt]{images/analytical2}} \qquad
    \subfloat[Simulated electron density cut]{\includegraphics[width=210pt]{images/overlay2}}\\
    \subfloat[Relative error]{\includegraphics[width=210pt]{images/error2}}\qquad
    \subfloat[Simulated electron density]{\includegraphics[width=210pt]{images/dens2}}
    \caption{Comparison of analytical and simulated electron densities for the second mode.}
  \end{center}
\end{figure}
\begin{figure}[t]
  \begin{center}
\subfloat[Analytical wavefunction and probability distribution]{\label{fig:analytical3}\includegraphics[width=210pt]{images/analytical3}} \qquad
    \subfloat[Simulated electron density cut]{\includegraphics[width=210pt]{images/overlay3}}\\
    \subfloat[Relative error]{\includegraphics[width=210pt]{images/error3}}\qquad
    \subfloat[Simulated electron density]{\includegraphics[width=210pt]{images/dens3}}
    \caption{Comparison of analytical and simulated electron densities for the third mode.}
  \end{center}
\end{figure}
