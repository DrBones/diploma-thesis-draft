Due to the lack of proper standard problems regarding the validity of a quantum mechanical simulator for transport properties the outcome of the numerical calculations via the \gfnc{} method will be compared to a selection of simple problems and experimental results.\par
\subsubsection{Electron Density}
For the case of a quantum wire the probability disrtibution can be calculated analytically. The quantum wire is simply a one-dimensional potential well in $y$ direction which ideally extends infinitely in the $x$ direction.
Because of this the wavefunction can be separated into transverse and longitudinal parts i.e
\begin{align}
\Psi(x,y) = \sqrt{2/L}\text{ sin}(n\pi/L \cdot y) \cdot e^{ik_xx}\,.
\end{align}
Because of the transverse confinement the eigenenergies become discrete similar to the case of the $z$ confinement illustrated in \cref{fig:potentialwell}.
\todo[noline]{average electro density for 100nm wire is about $10^17$ so that should fit}
The analytically calculated probability distribution is renormalized to experimentally measured electron densities $n \sim 10^{17}/m^2$ typical for a 2DEG embedded in heterojunctions containing InAs\,\cite{gelfand2006}\,\cite{JJAP.26.L59}
In figures (\dots ) the analytical calculation for the first three modes is superimposed with the simulated results for a nano-wire of 100nm. 
The parameters for the effective mass of an electron in a InAs 2DEG are taken to be $0.026\times m_0$ with $m_0$ being the electron rest mass\,\cite{PhysRev.105.460}.\par
\begin{figure}[h!]
  \begin{center}
    % \showthe\columnwidth % Use this to determine the width of the figure.
\subfloat[Analytical wavefunction and probability distribution]{\label{fig:analytical1}\includegraphics[width=210pt]{images/analytical1}} \qquad
    \subfloat[Simulated and re-normalized analytical transverse electron density cut]{\label{fig:overlay1}\includegraphics[width=210pt]{images/overlay1}}\\
    \subfloat[Relative error $\frac{n1_{ana}-n1_{sim}}{n1_{sim}}$]{\label{fig:relerror1}\includegraphics[width=210pt]{images/error1}}\qquad
    \subfloat[Simulated two dimensional electron density]{\label{fig:dens1}\includegraphics[width=210pt]{images/dens1}}
    \caption{Comparison of analytical and simulated electron densities for the first mode. Analytical probability amplitude \ref{fig:analytical1}, cut of electron density and analytical calculation overlay perpendicular to the length of the wire \ref{fig:overlay1}, relative error \ref{fig:relerror1} and two-dimensional electron density \ref{fig:dens1}}\label{fig:mode1}
  \end{center}
\end{figure}
The electrons populate only the first mode at the calculated energy using the dispersion relation of electrons on a lattice in transverse direction
\begin{align}
E_{n} = 2t_0(1-\text{cos}(n\pi a/L)).
\label{eqn:dispersionlattice}
\end{align}
Here $L=100$nm is the transverse extent of the nano-wire and $n=1$ the number of the mode. Figure~\ref{fig:analytical1} shows the normalized wavefunction and probability density for the first mode.\par
In \cref{fig:overlay1} of the tableau of \cref{fig:mode1} the overlay of re-normalized analytical wavefunction and simulated electron density shows the excellent agreement of theory and simulation. This becomes especially clear noting the relative error in \cref{fig:relerror1} of magnitude $10^{-11}$ normalized to \emph{mean electron density}.\par
\Cref{fig:dens1} displays a top view of the two-dimensional electron density. Red denoting high and blue low density. The electron density is uniform in $x$-direction as expected.\par
\begin{figure}[h]
  \begin{center}
    % \showthe\columnwidth % Use this to determine the width of the figure.
\subfloat[Analytical wavefunction and probability distribution]{\label{fig:analytical2}\includegraphics[width=210pt]{images/analytical2}} \qquad
    \subfloat[Simulated and re-normalized analytical transverse electron density cut]{\label{fig:overlay2}\includegraphics[width=210pt]{images/overlay2}}\\
    \subfloat[Relative error $\frac{n1_{ana}-n1_{sim}}{n1_{sim}}$]{\label{fig:relerror2}\includegraphics[width=210pt]{images/error2}}\qquad
    \subfloat[Simulated two dimensional electron density]{\label{fig:dens2}\includegraphics[width=210pt]{images/dens2}}
    \caption{Comparison of analytical and simulated electron densities for the first mode. Analytical probability amplitude \ref{fig:analytical2}, cut of electron density and analytical calculation overlay perpendicular to the length of the wire \ref{fig:overlay2}, relative error \ref{fig:relerror2} and two-dimensional electron density \ref{fig:dens2}}\label{fig:mode2}
  \end{center}
\end{figure}
The second mode is also appropriately populated for $n=2$. The analytical calculation does not consider the existence of lower modes at a certain energy, in contrast the model employed in the simulations does. The electron densities for the first and second modes have therefore to be superposed what is shown in \cref{fig:overlay2}. The total energy independent electron density is obtained by integrating over energy space as is shown in \cref{eqn:gfncdensity}.For transport properties it suffices to focus on energies $E=E_F\pm k_bT$ because the conductance is a \textsc{Fermi} surface property c.f Sec.~\ref{sec:landauerbuettiker}. $E_F$ denotes the \textsc{Fermi} energy and $k_bT$ the thermal energy contribution $k_b$ being the \textsc{Boltzmann} constant and $T$ the temperature.\par
The relative error increases considerably but stays within acceptable bounds considering the rather simple analytical model.
Also the uniform distribution along the wire remains and the two maxima in electron density can clearly be seen in \cref{fig:dens2}.\par
\begin{figure}[h]
  \begin{center}
\subfloat[Analytical wavefunction and probability distribution]{\label{fig:analytical3}\includegraphics[width=210pt]{images/analytical3}} \qquad
    \subfloat[Simulated and re-normalized analytical transverse electron density cut]{\label{fig:overlay3}\includegraphics[width=210pt]{images/overlay3}}\\
    \subfloat[Relative error $\frac{n1_{ana}-n1_{sim}}{n1_{sim}}$]{\label{fig:relerror3}\includegraphics[width=210pt]{images/error3}}\qquad
    \subfloat[Simulated two dimensional electron density]{\label{fig:dens3}\includegraphics[width=210pt]{images/dens3}}
    \caption{Comparison of analytical and simulated electron densities for the first mode. Analytical probability amplitude \ref{fig:analytical3}, cut of electron density and analytical calculation overlay perpendicular to the length of the wire \ref{fig:overlay3}, relative error \ref{fig:relerror3} and two-dimensional electron density \ref{fig:dens3}}\label{fig:mode3}
  \end{center}
\end{figure}
The results of simulation and simple analytical model begin to derivate stronger for the case of the third mode. Only the previous lower mode is considered in the renormalization of the analytical result in the overlay of transversal electron densities but the simulated results are for still within acceptable bounds for as \cref{fig:relerror3} shows. Derivations in the predictions of electron density for the compared models continue to exist as is expected because the analytical model does not incorporate the many-particle nature of the 2DEG like the \gfnc{} theory does.
The electron density calculation show throughout satisfactory and acceptable results.\par
\FloatBarrier
\subsubsection{Spin densities}
Validating the results of the spin density calculations is a rather intricate process because of the number of parameters and physical influences involved.\par
The spin density of flowing electrons in a conductor with spin-orbit coupling exhibits a characteristic spin split in the local magnetic field due to the \textsc{Rashba} interaction.
The spin-split has been measured through optical means by \textsc{Kato} et. al.\,\cite{Kato2004Science} and is shown in \cref{fig:spinplitkato}. It shows the split of electron spin flowing along the wire. There is an accumulation of upward spin in the upper half and of downward spin in the lower half.\par
\begin{figure}[h]
  \begin{center}
    \subfloat[Experimentally measured spin density $s_z$\,\cite{Kato2004Science} in arbitrary units]{\label{fig:spinplitkato}\includegraphics[width=\textwidth]{images/spinsplitkato}}\qquad
    \subfloat[Simulated spin density $s_z$ in arbitrary units]{\label{fig:spinsplitme}\includegraphics[width=\textwidth]{images/spinsplitme}}\qquad
    \caption{Spin-split. Comparison of experimental and simulated spin densities in a nanowire.}
  \end{center}
\end{figure}
\todo[noline]{Spin oscillation??}The spin split is also shown in the results of the simulations in the \gfnc{} model. Due to the approximative and simplifying nature of the model considered only an idealized system is described therefore lacking common physical features like temperature induced noise although they can be implemented in an efficient fashion.\par
\FloatBarrier
\subsubsection{Transmission}
Validating the results from the transmission calculations can be done in a very straightforward manner. For quantum point contacts of varying constriction or if the energy parameter is altered the conductance will vary only in discrete steps.\par
This is a direct quantum mechanical consequence as from a classical point of view a linear relation between conductance and device width is expected.
Is has been found out that the conductance is quantized in steps of the conductance quantum $e^2/h$ c.f. \cref{eqn:conductance}.
It is due to the finite transverse size of the modes within the conductor that only if space permits another mode is established. The transmission probability function \ref{eqn:transmissionfunction} will reach a plateau when each new mode is established and ideally performs a unit step per mode when the next higher or lower state begins to exist in the conductor. The shape and height of the steps is subject to a multitude of influences but because of its discrete step height under certain conditions a valuable tool for comparison.\par
\begin{figure}[h]
  \begin{center}
    \subfloat[Experimentally measured conductance for varying gate voltage, effectivly "opening" the quantum point contact\,\cite{PhysRevLett.60.848}] {\includegraphics[]{images/qpcwees}}
    \subfloat[Simulated conductance from opening a QPC constriction geometrically]{\includegraphics[]{images/qpcme}}
    \caption{Comparison of experimental and simulated conductances of an QPC constriction of variable size. Plotted is the conductance in terms of the conductance quantum $e^2/h$ against two different but proportional measures of width of the QPC.}
\end{center}
\end{figure}
The comparison of experimentally measured and simulated conductance quantization show a very good agreement in step height as new propagating modes are established.\par
The simulated results show in summary acceptable results to merit its application in the theoretical analysis of quantum devices embedded in a 2DEG.
The roughness and shape of the steps is subject to investigation of the following chapter.\par
\FloatBarrier
