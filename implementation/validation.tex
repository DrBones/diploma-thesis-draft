Due to the lack of proper standard problems regarding the validity of a quantum mechanical simulator for transport properties the outcome of the numerical calculations via the \gfnc{} method will be compared to a selection of simple problems and experimental results.\par
For the case of a quantum wire the electron density can be calculated analytically. The quantum wire is simply a one-dimensional potential well in $y$ direction which ideally extends infinitely in the $x$ direction.
Because of this the wavefunction can be separated into transverse and longitudinal parts i.e:
\begin{align}
\Psi(x,y) = \sqrt{2/L}\text{ sin}(n\pi/L \cdot y) \cdot e^{ik_xx}
\end{align}
Because of the transverse confinement the eigenenergies become discrete similar to the case of the $z$ confinement illustrated in fig. (\ref{fig:potentialwell}.
\todo[noline]{average electro density for 100nm wire is about $10^17$ so that should fit}
