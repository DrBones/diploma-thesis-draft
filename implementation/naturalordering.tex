The dimension of the matrices obtained by inclusion of the self energy term in \cref{eqn:finitegreensfunctionwithselfenergy} is determined by the number of lattice points which represent the discretized system.
With a lattice spacing in the nanometer range, the number of lattice-points $N$ easily exceeds magnitudes of $10^5$ for a mesoscopic device in the range of micrometers. 
The number of lattice points can be reduced by discarding the environment of the device, effectively mapping the full rectangular lattice to a \emph{sparse grid}. Especially for devices of non-rectangular geometry this reduces the size of the matrix by a factor of about two.\par
The full inversion of matrices of this size has high demands on memory and computational power due to its $\mathcal{O}(\text{N})$ complexity. Even with specialised algorithms direct inversion is still very time-consuming \cite{Datta2000.2.53}, \cite{Li2009Thesis}. Faster and more efficient techniques for the computation of the \gfnc{} are required. There exist several techniques for example based on \textsc{Takahashi}'s observation \cite{Takahashi1973} or the \textsc{Dyson} equation that deliver much higher performance.\par
For structures like quantum wires the \hamil{} turns out to be in a block tridiagonal form, a \emph{natural ordering}. This natural ordering leads to so-called \emph{layered devices}, see \cref{fig:layered}, in which the device is cut into slices perpendicular to the longest dimension as shown in \cref{sec:discretematrixrep}. Each slice is represented by a block on the diagonal of the \hamil{} and their interactions on the off-diagonal.
\begin{figure}[h!]
\centering
\includegraphics[trim=0cm 38cm 0cm 0cm,clip=true,width=0.5\textwidth]{images/layeredstructure}
\caption{(a) Schematic of a discretized nanowire as layered structure. The dots denote lattice-points. The slices are indicated by dashed lines. (b) Hamiltonian of the discretized nanowire. The Blocks are highlighted by dashed lines.)}
\label{fig:layered}
\end{figure}
