The now finite dimension of the matrices obtained by inclusion of the self energy term is determined by the number of lattice points which make up the discretized system.
If a reasonable resolution i.e lattice spacing in the nanometer range is desired the number of lattice-points easily exceeds magnitudes of $10^5$ for a mesoscopic device in the range of micrometers which would lead. 
The full inversion of matrices of this size has high demands on memory and computational speed due to its $\mathcal{O}(\text{Number of lattice-points})$ complexity and is therefore, even with specialised direct algorithms such as LU decomposition based inversion, a very time-consuming affair\cite{Datta2000.2.53} and \cite{Li2009Thesis}.
Therefore faster and more efficient techniques for the computation of the \gfnc{} are desireable.
\todo{serialization and sparse representation}For structures of serial nature i.e relatively narrow structures with colinear leads like a quantum wire the \hamil{} turns out to be in a block tri-diagonal form a \emph{natural ordering}. This natural ordering leads to so called \emph{layered devices} seen in fig. (\ref{fig:layered}) in which the device is cut into slices perpendicular to the longest dimension as shown in chapter \ref{sec:discretematrixrep}. There exist several techniques for example based on \textsc{Takahashi}'s observation \cite{Takahashi1973} or the \textsc{Dyson} equation that deliver much higher performance.
\begin{figure}[h!]
\centering
\includegraphics[trim=0cm 38cm 0cm 0cm,clip=true,width=0.5\textwidth]{images/layeredstructure}
\caption{Layered structure (Sample picture)}
\label{fig:layered}
\end{figure}
