To calculate the transport properties of the device in the discrete tight-binding approximation the expressions for transmission in \cref{eqn:transcoeff}, electron density in \cref{eqn:analyticalelectrondensity} and spin denisty in \cref{eqn:spindensity} have to be recast in terms of steady-state tight-binding quantities, namely spin dependent \gfnc s.\par
The energy dependent local density of states (LDOS) is written in terms of the retarded \gfnc{} \cite{AnLunNik2008}
\begin{align}
\text{LDOS}_{\v{m}}(E)=-\frac{1}{\pi} \text{Im}(\text{Tr}_S[\mat{G}_{\v{mm}}^r(E)]) \ .
\label{eqn:ldos}
\end{align}
The matrix index (\v{mm}) shows that the density of states can be calculated only from the diagonal of the retarded \gfnc{}.
To obtain real space information the spin degree of freedom has to be reducted. This is done by taking the \emph{partial trace} \cite{Jacobs} of the operator effectively adding up the density of states for spin up and spin down electrons.\par
It is important to note that for open systems there are no predetermined eigenenergies and therefore the energy $E$ is a free parameter. The \gfnc{} measures the \emph{response of the system} to electron states in one of the leads of a given energy $E$. The electron density can be obtained from the lesser \gfnc{} given as an example in \cref{eqn:edensfromgreensfnc}. For experimentally measurable quantities an integration over energy has to be performed, which can be divided into equilibrium and non-equilibrium parts if a distinction between total and conducting electrons is desired
\begin{align}
	\expv{n_{\v{m}}} = \frac{1}{2\pi} \int_{-\infty}^{\infty}\text{d}E\text{Tr}_S[\mat{G}_{\v{mm}}^<(E)] \ .
	\label{eqn:gfncdensity}
\end{align}
The lesser \gfnc{} also allows the definition and calculation  of a local spin density by multiplication with the \textsc{Pauli} matrix $\mat{\sigma}_i$, here for the out-of-plane polarization $\expv{s^z_{\v{m}}}$\,\cite{Wimmer2009Thesis}:
\begin{align}
\expv{s^z_{\v{m}}} = \frac{\hbar}{4 \pi i} \int^{\infty}_{-\infty}\text{d}E\text{ Tr}_S\left[\mat{\sigma}_z\mat{G}^<_{\v{mm}}\right]\ .
\label{eqn:gfncspindensity}
\end{align}
In the tight-binding picture one can also define a bond spin-current operator \cite{EPL.80.47001}. The bond spin current is the amount of spin polarization flowing between two adjacent lattice points
\begin{align}
\expv{\v{j}_{\v{m}\v{m}'}^{s_z(tot)}} &=\expv{\v{j}_{\v{m}\v{m}'}^{s_z(eq)}}+ \expv{\v{j}_{\v{m}\v{m}'}^{s_z(neq)}}\\
	&=\frac{t_{SO}}{2} \int_{E_{\text{cut-off}}}^{E_F-eV/2} \frac{\text{d}E}{2 \pi} \text{Tr}_S \left[\mat{\sigma}_z\left(\mat{G}^<_{\v{m}'\v{m}}(E)- \mat{G}^<_{\v{m}\v{m}'}(E)\right)\right]\\
	&+\frac{t_{SO}}{2} \int_{E_F-eV/2}^{E_F+eV/2} \frac{\text{d}E}{2 \pi} \text{Tr}_S \left[\mat{\sigma}_z\left(\mat{G}^<_{\v{m}'\v{m}}(E)- \mat{G}^<_{\v{m}\v{m}'}(E)\right)\right] \ .
	\label{eqn:gfnccurrent}
\end{align}
The bond current can be separated into equilibrium current in the interval $[E_{\text{cut-off}}, E-eV/2]$ and non-equilibrium current in the interval $[E-eV/2, E+eV/2]$. In the linear response regime the energy $E\pm eV/2$ du to the lead potentials $eV/2$ can be replaced by the thermal energy of the electrons $\approx k_B T$. Here $k_B$ denotes the \textsc{Boltzmann} constant and $T$ the temperature in \textsc{Kelvin}.
With the help of the so called \textsc{Fisher-Lee} relation \cite{PhysRevB.23.6851} connecting the scattering formalism of \textsc{Landauer-B\"uttiker} with the \gfnc s the total transmission probability can be written as
\begin{align}
T_{pq} = \text{Tr}(\Gamma_p G_{pq} \Gamma_q G^+_{pq}) = \sum_{n,m} \abs{t^{pq}_{ll'}}^2\ .
\label{eqn:transmissionfunction}
\end{align}
This formula is known as the \textsc{Caroli} expression. In the linear response regime it is equivalent to \textsc{Landauer}'s formula \cite{PhysRevB.72.035450}.
