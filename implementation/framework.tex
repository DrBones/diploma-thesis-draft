The algorithms and logic neccessary for the simulation are mainly implemented in an object oriented programming language called \emph{python}.
As an interpreted scripting language python suffers in comparison to low level languages like FORTRAN or C and its derivatives a performance penalty. However the ease of use, powerful numerical modules and large scientific userbase make it a suitable choice if one writes a simulator from ground up.\par
Especially because of pythons object-orientation and the existance of a multitude of interfaces to lower level programming languages and libraries computing intensive parts of the program are easily modulized.\par
It exists a large suite of numerical modules wrapping fast numerical algebra routines like BLAS, ATLAS or MKL called \emph{numpy} \cite{numpy} which is heavily  used for the simulations in this work for high performance computations.\par
Higher level mathmatical functions and operations like the \textsc{Schur} decomposition are part \emph{scipy} \cite{scipy} which is a superpackage of numpy.\par
The RGA is implemented in pure python on top of the numeric algebra package with the addition of custom routines.\par
For the handling of graph based calculations and reoderings \emph{networkx} \cite{networkx} is employed. The networkx package supplies a graph class upon which the custom BFS and graph partitioning algorithm used by \textsc{Wimmer} are also written in pure python to ensure interoperability.
