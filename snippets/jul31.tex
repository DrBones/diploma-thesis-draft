% ***********************************************************
% ******************* PHYSICS HEADER ************************
% ***********************************************************
%\documentclass[11pt]{article} 
% \documentclass[12pt,twoside,a4paper]{article}
\documentclass[12pt,twoside,a4paper]{scrartcl}
\setkomafont{sectioning}{\normalcolor\bfseries}
%%%%%%%%%%%%%% PACKAGE INCLUDES %%%%%%%%%%%%%%%
\usepackage[utf8]{inputenc} %allows input of utf8 characters
\usepackage[american,ngerman]{babel}
\usepackage[T1]{fontenc} %more complete but lower quality than ComMod fontpacke
% \usepackage{ae,aecompl}
\usepackage[tbtags]{mathtools}
%\usepackage{amsmath} % AMS Math Package
\usepackage{amsthm} % Theorem Formatting
\usepackage{amssymb}	% Math symbols such as \mathbb
\usepackage{graphicx} % Allows for eps images
\usepackage{multirow}
\usepackage{multicol} % Allows for multiple columns
\usepackage{color} % well, supplies colors
\usepackage{todonotes}
\usepackage{booktabs}
\usepackage{setspace} %Zeilenabstand
%\onehalfspacing
% \usepackage{mathptmx}
% \usepackage{bbold}
% \usepackage{bold-extra} % Ugly bitmap fonts, but holds bold glyphs!
% \usepackage{kpfonts} % Pretty complete fontpackage but looks 'different'
\usepackage[
    pdftitle={Spin-dependend transport simulations in two-dimensionsal electron systems},    % title
    pdfauthor={Jonas Siegl},     % author
    pdfsubject={Quantum Transport},   % subject of the document
    pdfcreator={Jonas Siegl},   % creator of the document
    pdfproducer={Jonas Siegl}, % producer of the document
    pdfkeywords={Green's Function } {Numerical } {Transport}, % list of keywords
    % linktocpage=true,
    linkcolor=red,          % color of internal links
    citecolor=green,        % color of links to bibliography
    filecolor=magenta,      % color of file links
    urlcolor=cyan,           % color of external links
pagebackref=true]{hyperref}
\usepackage[all]{hypcap} % shifts the link point to the top of tables or figures so it is visible when clicked
%\usepackage{array}
%\usepackage[final]{svninfo}
\usepackage{tabulary}
\usepackage{tabularx}
%\usepackage{ifthen}
%\usepackage[dvips]{graphicx}
%\usepackage[small,bf]{caption}
\usepackage{subfig} 		%enabled the use of subfloats within a float
%\usepackage{textcomp}
%\usepackage{pstricks}
%\usepackage{courier}
%\usepackage{geometry}
%\usepackage{url}
\usepackage{tikz}
\tikzstyle{every picture}+=[remember picture]
\usetikzlibrary{decorations.pathreplacing}
%\usepackage{paralist}
%\usepackage{xspace}
%\usepackage[dvips,letterpaper,margin=0.75in,bottom=0.5in]{geometry}
 % Sets margins and page size
%\pagestyle{empty} % Removes page numbers
%\makeatletter % Need for anything that contains an @ command 
%\renewcommand{\maketitle} % Redefine maketitle to conserve space
%{ \begingroup \vskip 10pt \begin{center} \large {\bf \@title}
%	\vskip 10pt \large \@author \hskip 20pt \@date \end{center}
%  \vskip 10pt \endgroup \setcounter{footnote}{0} }
%\makeatother % End of region containing @ commands
\renewcommand{\labelenumi}{(\alph{enumi})} 		% Use letters for enumerate

% Shortcuts to Named Objects
\newcommand{\gfnc}{\textsc{Green}'s function} 		% call with {} like \gfnc{} to ensure following whitespace
\newcommand{\cgfnc}{\textsc{Green}'s Function} 		% call with {} like \gfnc{} to ensure following whitespace
\newcommand{\hamil}{\textsc{Hamilton}ian}
\newcommand{\rash}{\textsc{Rashba}}
\newcommand{\sdg}{\textsc{Schr\"odinger} equation}
\newcommand{\lanbform}{\textsc{Landauer-B\"uttiker} formalism}
\newcommand{\clanbform}{\textsc{Landauer-B\"uttiker} Formalism}

% Vectors and matrices
\let\vaccent=\v 					% rename builtin command \v{} to \vaccent{}
\renewcommand{\v}[1]{\ensuremath{\boldsymbol{#1}}} 	% for vectors
\newcommand{\mat}[1]{\ensuremath{\boldsymbol{#1}}} 	% for matrices
\newcommand{\uv}[1]{\ensuremath{\mathbf{\hat{#1}}}} 	% for unit vector
\newcommand{\abs}[1]{\left| #1 \right|} 		% for absolute value
\newcommand{\avg}[1]{\left< #1 \right>} 		% for average
\providecommand{\abs}[1]{\lvert#1\rvert}
\providecommand{\norm}[1]{\lVert#1\rVert}

% Derivatives: 
\let\underdot=\d 					% rename builtin command \d{} to \underdot{}
\renewcommand{\d}[2]{\frac{d #1}{d #2}} 		% for derivatives
\newcommand{\dd}[2]{\frac{d^2 #1}{d #2^2}} 		% for double derivatives
\newcommand{\pd}[2]{\frac{\partial #1}{\partial #2}} 	% for partial derivatives
\newcommand{\pdd}[2]{\frac{\partial^2 #1}{\partial #2^2}} 	% for double partial derivatives
\newcommand{\pdc}[3]{\left( \frac{\partial #1}{\partial #2} \right)_{#3}} % for thermodynamic partial derivatives

%Dirac style
\newcommand{\ket}[1]{\left| #1 \right>} 	% for Dirac kets
\newcommand{\bra}[1]{\left< #1 \right|} 	% for Dirac bras
\newcommand{\braket}[2]{\left< #1 \vphantom{#2} \right| \left. #2 \vphantom{#1} \right>} % for Dirac brackets
\newcommand{\matrixel}[3]{\left< #1 \vphantom{#2#3} \right| #2 \left| #3 \vphantom{#1#2} \right>} % for Dirac matrix elements

%Gradient and curls
\newcommand{\grad}[1]{\v{\nabla} #1} 		% for gradient
\let\divsymb=\div 				% rename builtin command \div to \divsymb
\renewcommand{\div}[1]{\v{\nabla} \cdot #1} 	% for divergence
\newcommand{\curl}[1]{\v{\nabla} \times #1}	% for curl

\let\baraccent=\= 				% rename builtin command \= to \baraccent
\renewcommand{\=}[1]{\stackrel{#1}{=}} 		% for putting numbers above =
\newtheorem{prop}{Proposition}
\newtheorem{thm}{Theorem}[section]
\newtheorem{lem}[thm]{Lemma}
\theoremstyle{definition}
\newtheorem{dfn}{Definition}
\theoremstyle{remark}
\newtheorem*{rmk}{Remark}

%%%%%%%%%%%%%% STYLE %%%%%%%%%%%%%%%
%\geometry{hmargin={1in,1in},vmargin={2.0in,1.5in}}
\usepackage{fancyhdr}
\pagestyle{fancy}
% \renewcommand{\chaptermark}[1]{\markboth{\MakeUppercase{\chaptername} \ \thechapter. \ \textbf{#1}}{}}
% \renewcommand{\sectionmark}[1]{\markright{\textbf{\thesection \ #1}}}
\fancyhead[RE,LO]{}
% \fancyhead[RO]{\rightmark}
% \fancyhead[LE]{\leftmark}
%%\fancyfoot{}
%%\fancyfoot[RO,LE]{\thepage}
\renewcommand{\footrulewidth}{0.4pt}
% \renewcommand{\headrulewidth}{0.4pt}

\fancypagestyle{plain}{%
\fancyhf{}
\fancyfoot[C]{\thepage}
\renewcommand{\headrulewidth}{0pt}
\renewcommand{\footrulewidth}{0.4pt}}


% configure subfig
\captionsetup[subfloat]{labelformat=simple,listofformat=subsimple}
\def\thesubfigure{(\alph{subfigure})} 
%
%% url
%\makeatletter
%\def\url@leostyle{%
%  \@ifundefined{selectfont}{\def\UrlFont{\sf}}{\def\UrlFont{\small\ttfamily}}}
%\makeatother
%\urlstyle{leo}
%
%
\numberwithin{equation}{section}
%
%%%%%%%%%%%%%%%% CODE LISTING SETTINGS %%%%%%%%%%%%%%%
%\usepackage{listings}
%
%\lstset{
%frame=tb,
%framesep=5pt,
%tabsize=2,
%basicstyle=\scriptsize\ttfamily,
%showstringspaces=false,
%stringstyle=\color{red},
%commentstyle=\color{gray},
%breaklines=true,
%numbers=left,
%numbersep=5pt
%%numberstyle=\ttfamily,
%%keywordstyle=\color{green},
%%identifierstyle=\color{blue},
%%xleftmargin=5pt,
%%xrightmargin=5pt,
%%aboveskip=\bigskipamount,
%%belowskip=\bigskipamount,
%%rulecolor=\color{Gray},
%%numberstyle=\color{blue},
%}

%\lstdefinelanguage{JavaScript} {
%morekeywords={
%break,const,continue,delete,do,while,export,for,in,function,
%if,else,import,in,instanceOf,label,let,new,return,switch,this,
%throw,try,catch,typeof,var,void,with,yield
%},
%sensitive=false,
%morecomment=[l]{//},
%morecomment=[s]{/*}{*/},
%morestring=[b]",
%morestring=[d]'
%}
%
%% MACROS
%\include{macros}

% ***********************************************************
% ********************** END HEADER *************************
% ***********************************************************

% INFO
\author{Jonas Siegl}
%set spacing in multiline environments like align and gather
\setlength{\jot}{10pt}
\begin{document}
\selectlanguage{american}
\section{Self energy, surface green's function}
\begin{equation}
	\left[\mabol{H}_{ll'}\right]_{\alpha\beta} = \bra{\Phi_\alpha(l)}\mabol{H}\ket{\Phi_\beta(l')}
	\label{eq:Hmatrixelements}
\end{equation}

\begin{equation}
	\left[ \mabol{G}_{ll'}(E^+) \right]_{\alpha\beta} = \bra{\Phi_\alpha(l)}\frac{1}{\mabol{1}E^+-\mabol{H}}\ket{\Phi_\beta(l')}
	\label{eq:Gmatrixelements}
\end{equation}

lead  to:

(Wikipedia:start)

Linear homogeneous recurrence relations with constant coefficients.
An order ''d'' linear homogeneous recurrence relation with constant coefficients  is an equation of the form:

\begin{equation}
	a_n = c_1a_{n-1} + c_2a_{n-2}+\cdots+c_da_{n-d} 
	\label{eq:linearrecformula}
\end{equation}

where the ''d'' coefficients $c_i$ (for all ''i'') are constants.

More precisely, this is an infinite list of simultaneous linear equations, one for each $n> d-1$. A sequence which satisfies a relation of this form is called a '''linear recursive sequence''' or LRS. There are ''d'' degrees of freedom for LRS, the initial values $a_0,\dots,a_{d-1}$ can be taken to be any values but then the linear recurrence determines the sequence uniquely.

The same coefficients yield the characteristic polynomial (also ``auxiliary polynomial'')
\begin{equation}
	p(t)= t^d - c_1t^{d-1} - c_2t^{d-2}-\cdots-c_{d}
	\label{eq:charactersitic_polynomial}
\end{equation}
(Wikipedia:end)
For a layered system in which the inslice Hamiltonian $H_{00}$ is Hermitian and the hopping-matrices obey $H_{01}=H_{10}^{\dagger}$ the Schr\"odinger equation
reads:
\begin{equation}
	\mabol{H}_{00}\psi_{l}+\mabol{H}_{01}\psi_{l+1}+\mabol{H}_{10}{\psi_{l-1}} = E \psi_{l}
	\label{eq:hamilonian}
\end{equation}
The integer subscript $l$ indicating the location of a layer measured in interslice distances, lattice constants.
The number of available degrees of freedom will be $\mu$ and the corresponding components will be declared e.g.
$\psi_{l}^{\mu}$ in our case $\mu$ will run through all points on the lattice
times the spin degree of freedom (2).So $\mu = 1,2,\cdots,2M$ if $M$ is the number of nodes per slice.
Introducing the Bloch state with $\phi_{k_{\|}}$ being a normalized 2M column vector:
\begin{equation}
	\psi_{l}=const \cdot e^{ik_{\|}l} \phi_{k_{\|}}
	\label{eq:blochstate}
\end{equation}
Substitution into equation (\ref{eq:hamilonian}) yields:
\begin{equation}
	(E-\mabol{H}_{00}-\mabol{H}_{01}e^{ik_{\|}l}-\mabol{H}_{10}e^{-ik_{\|}l})\phi_{k_{\|}} = 0
	\label{eq:substhamilonian}
\end{equation}
Unlike usual band theory in which the $M$ values of $E$ are determined for a real $k$ the goal of this calculation is all possible complex values of $k_{\|}$ given a real $E$.
The energy bands are degenerated in respect to the $k$-vector, so for any given energy there exists a pair of real values for one $k_{\|}=k$, corresponding to positive group velocity (right-moving):
\begin{equation}
	v_{k}=\frac{1}{\hbar}\pd{E(k)}{k} > 0
	\label{posgroupvelocity}
\end{equation}
or $k_{\|}=\bar{k}$ with negative group velocity (left moving)\cite{PhysRevB.59.11936}:
\begin{equation}
	v_{k}=\frac{1}{\hbar}\pd{E(\bar{k})}{\bar{k}} < 0
	\label{neggroupvelocity}
\end{equation}

All the real wave vectors constitute the open scattering channels available for transport through the device.The simplest case for example has $\bar{k}=-k$ if $H_{01}=H_{10}$.
All wave vectors with finite imaginary part make up the closed or evanecent channels into which can be scattered but can in some situations be nelected.
As can be seen by taking the Hermitian conjugate of equation (\ref{eq:substhamilonian}), each complex conjugate $k_{\|}^*$ of $k_{\|}$ is also a possible solution, 
hence to every solution with a positive imaginary part, decaying with increasing $l$ i.e. right-decaying, there exists a solution with negative imaginary part which is left-decaying.
By adding the small imaginary number to the energy in the calculations the modulus of the eigenvalues will be less than 1 for $k$s with positive imaginary part and greater than one if negative effectively enabling us to separate all the eigenvectors into equal parts of left- and right-movers\cite{JChemPhys.120.7733}.


\subsection{Localized Basis}

In this work it is assumed the the electron wavefunction can be expanded in terms of localized orbitals around each atomic site, or Linear Combination of Atomic Orbitals (LCAO), but the employed formalism allows a greater generality as for example the use of Muffin-Tin orbitals (LMTO). Because of a more direct physical accessabilty and a simpler formulation LCAO is used in the following.

In the literature concerning the treatment of layered structured the term \emph{principle layer} is used for the unit cell consisting of one or more atomic layers for which a periodicity is given.
The interaction beyond neighbouring principle layers (PL) is neglected. In the case under consideration there exists only one atomic layer in each PL so the terms coincide but PL will be used in the remainder for better recognition.
A planar orbital is defined as a state 
\begin{align}
	&(\mabol{1}E^+-\mabol{H}_{00})\mabol{G}_{00}-\mabol{H}_{01}\mabol{G}_{10}=\mabol{1}\nonumber \\
	&(\mabol{1}E^+-\mabol{H}_{11})\mabol{G}_{10}-\mabol{H}_{12}\mabol{G}_{20}-\mabol{H}_{10}\mabol{G}_{00}=\mabol{0}\\
	&(\mabol{1}E^+-\mabol{H}_{22})\mabol{G}_{20}-\mabol{H}_{23}\mabol{G}_{30}-\mabol{H}_{21}\mabol{G}_{10}=\mabol{0}\nonumber \\
	&\cdots\nonumber
	\label{eq:LRS}
\end{align}


Formulas used in surface green's Function calculation:

\begin{equation}
	G_{L}^{S}= \left[(\mabol{1}E^{+}-\mabol{H}_{00}) -\mabol{H}_{01}\mabol{S}_{2}\mabol{S}_{1}^{-1}\right]
	\label{surfgreenleft}
\end{equation}
\begin{equation}
	G_{R}^{S}= \left[(\mabol{1}E^{+}-\mabol{H}_{00}) -\mabol{H}_{0\bar{1}}\mabol{S}_{3}\mabol{S}_{4}^{-1}\right]
	\label{surfgreenright}
\end{equation}
\clearpage
\bibliographystyle{amsalpha}
\bibliography{thesis}
\end{document}
