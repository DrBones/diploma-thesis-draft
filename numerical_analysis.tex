\subsection{Discretization of Hamiltonian on 2D lattice}
In this section it will be described how the discretization of the following Hamiltonian is performed.
\begin{equation}
  \v{H} = -\frac{\hbar^{2}}{2m^*}\left(\pdd{}{x}+\pdd{}{y}\right) + U(x,y)
  \label{FreeHamiltonian}
\end{equation}

\textbf{WRONG}

Discretization means that the continuous real space in X and Y directions will be replaced by an infinite net of lattice points. For simplicity the same spacing $a$ is assumed in both directions. The lattice points are located in the pre-discrete space at the coordinates $x=i*a$ and $y=j*a$ with $i,j$ being integers and $a$ the lattice spacing. The Hamiltonian now operates on functions defined on discrete points in space obtained by:

\begin{equation}
  T_{i,j} \leftrightarrow T(x=ia,y=ja) \mbox{ and } U_{i,j}\leftrightarrow U(x=ia,y=ja)
  \label{FunctionDescrete}
\end{equation}
The Hamiltonian is expressed accordingly:
\begin{equation}
  \left[\v{H}T\right]_{x=ia,y=ja}=-\frac{\hbar^{2}}{2m^*}\left[\pdd{T}{x}+\pdd{T}{y}\right]_{x=ia,y=ja}+U_{i,j}T_{i,j}
  \label{DiscreteHamiltonian}
\end{equation}
Using finite differences and assuming small $a$ the first derivative in each direction is approximated by:
\begin{equation}
  \left[\pd{T}{x}\right]_{x=(i+1/2)a} \sim  \frac{1}{a}\left(T_{i+1}-T_{i}\right)
  \label{ApproximateFirstDerivative}
\end{equation}
In this case the subscript denotes evaulation at that point. By applying the concept a second time for the second derivative:
\begin{align}
  \left[\pdd{T}{x}\right]_{x=(i+1/2)a} &\sim \frac{1}{a}\left( \left[\pd{T}{x}\right]_{x=(i+1/2)a}-\left[\pd{T}{x}\right]_{x=(i-1/2)a} \right) \\
  &\sim \frac{1}{a^2}\left(T_{i+1}-2T_{i}+T_{i-1}\right)
  \label{ApproximateSecondDerivative}
\end{align}
If executed for both dimensions individually Thus the approximated Hamiltonian is written as:
\begin{equation}
  \left[\v{H}T\right]_{x=ia,y=ja}= \left(U_{i}\right)
  \label{<++>}
\end{equation}<++>
