% ***********************************************************
% ******************* PHYSICS HEADER ************************
% ***********************************************************
%\documentclass[11pt]{article} 
% \documentclass[12pt,twoside,a4paper]{article}
\documentclass[12pt,twoside,a4paper]{scrartcl}
\setkomafont{sectioning}{\normalcolor\bfseries}
%%%%%%%%%%%%%% PACKAGE INCLUDES %%%%%%%%%%%%%%%
\usepackage[utf8]{inputenc} %allows input of utf8 characters
\usepackage[american,ngerman]{babel}
\usepackage[T1]{fontenc} %more complete but lower quality than ComMod fontpacke
% \usepackage{ae,aecompl}
\usepackage[tbtags]{mathtools}
%\usepackage{amsmath} % AMS Math Package
\usepackage{amsthm} % Theorem Formatting
\usepackage{amssymb}	% Math symbols such as \mathbb
\usepackage{graphicx} % Allows for eps images
\usepackage{multirow}
\usepackage{multicol} % Allows for multiple columns
\usepackage{color} % well, supplies colors
\usepackage{todonotes}
\usepackage{booktabs}
\usepackage{setspace} %Zeilenabstand
%\onehalfspacing
% \usepackage{mathptmx}
% \usepackage{bbold}
% \usepackage{bold-extra} % Ugly bitmap fonts, but holds bold glyphs!
% \usepackage{kpfonts} % Pretty complete fontpackage but looks 'different'
\usepackage[
    pdftitle={Spin-dependend transport simulations in two-dimensionsal electron systems},    % title
    pdfauthor={Jonas Siegl},     % author
    pdfsubject={Quantum Transport},   % subject of the document
    pdfcreator={Jonas Siegl},   % creator of the document
    pdfproducer={Jonas Siegl}, % producer of the document
    pdfkeywords={Green's Function } {Numerical } {Transport}, % list of keywords
    % linktocpage=true,
    linkcolor=red,          % color of internal links
    citecolor=green,        % color of links to bibliography
    filecolor=magenta,      % color of file links
    urlcolor=cyan,           % color of external links
pagebackref=true]{hyperref}
\usepackage[all]{hypcap} % shifts the link point to the top of tables or figures so it is visible when clicked
%\usepackage{array}
%\usepackage[final]{svninfo}
\usepackage{tabulary}
\usepackage{tabularx}
%\usepackage{ifthen}
%\usepackage[dvips]{graphicx}
%\usepackage[small,bf]{caption}
\usepackage{subfig} 		%enabled the use of subfloats within a float
%\usepackage{textcomp}
%\usepackage{pstricks}
%\usepackage{courier}
%\usepackage{geometry}
%\usepackage{url}
\usepackage{tikz}
\tikzstyle{every picture}+=[remember picture]
\usetikzlibrary{decorations.pathreplacing}
%\usepackage{paralist}
%\usepackage{xspace}
%\usepackage[dvips,letterpaper,margin=0.75in,bottom=0.5in]{geometry}
 % Sets margins and page size
%\pagestyle{empty} % Removes page numbers
%\makeatletter % Need for anything that contains an @ command 
%\renewcommand{\maketitle} % Redefine maketitle to conserve space
%{ \begingroup \vskip 10pt \begin{center} \large {\bf \@title}
%	\vskip 10pt \large \@author \hskip 20pt \@date \end{center}
%  \vskip 10pt \endgroup \setcounter{footnote}{0} }
%\makeatother % End of region containing @ commands
\renewcommand{\labelenumi}{(\alph{enumi})} 		% Use letters for enumerate

% Shortcuts to Named Objects
\newcommand{\gfnc}{\textsc{Green}'s function} 		% call with {} like \gfnc{} to ensure following whitespace
\newcommand{\cgfnc}{\textsc{Green}'s Function} 		% call with {} like \gfnc{} to ensure following whitespace
\newcommand{\hamil}{\textsc{Hamilton}ian}
\newcommand{\rash}{\textsc{Rashba}}
\newcommand{\sdg}{\textsc{Schr\"odinger} equation}
\newcommand{\lanbform}{\textsc{Landauer-B\"uttiker} formalism}
\newcommand{\clanbform}{\textsc{Landauer-B\"uttiker} Formalism}

% Vectors and matrices
\let\vaccent=\v 					% rename builtin command \v{} to \vaccent{}
\renewcommand{\v}[1]{\ensuremath{\boldsymbol{#1}}} 	% for vectors
\newcommand{\mat}[1]{\ensuremath{\boldsymbol{#1}}} 	% for matrices
\newcommand{\uv}[1]{\ensuremath{\mathbf{\hat{#1}}}} 	% for unit vector
\newcommand{\abs}[1]{\left| #1 \right|} 		% for absolute value
\newcommand{\avg}[1]{\left< #1 \right>} 		% for average
\providecommand{\abs}[1]{\lvert#1\rvert}
\providecommand{\norm}[1]{\lVert#1\rVert}

% Derivatives: 
\let\underdot=\d 					% rename builtin command \d{} to \underdot{}
\renewcommand{\d}[2]{\frac{d #1}{d #2}} 		% for derivatives
\newcommand{\dd}[2]{\frac{d^2 #1}{d #2^2}} 		% for double derivatives
\newcommand{\pd}[2]{\frac{\partial #1}{\partial #2}} 	% for partial derivatives
\newcommand{\pdd}[2]{\frac{\partial^2 #1}{\partial #2^2}} 	% for double partial derivatives
\newcommand{\pdc}[3]{\left( \frac{\partial #1}{\partial #2} \right)_{#3}} % for thermodynamic partial derivatives

%Dirac style
\newcommand{\ket}[1]{\left| #1 \right>} 	% for Dirac kets
\newcommand{\bra}[1]{\left< #1 \right|} 	% for Dirac bras
\newcommand{\braket}[2]{\left< #1 \vphantom{#2} \right| \left. #2 \vphantom{#1} \right>} % for Dirac brackets
\newcommand{\matrixel}[3]{\left< #1 \vphantom{#2#3} \right| #2 \left| #3 \vphantom{#1#2} \right>} % for Dirac matrix elements

%Gradient and curls
\newcommand{\grad}[1]{\v{\nabla} #1} 		% for gradient
\let\divsymb=\div 				% rename builtin command \div to \divsymb
\renewcommand{\div}[1]{\v{\nabla} \cdot #1} 	% for divergence
\newcommand{\curl}[1]{\v{\nabla} \times #1}	% for curl

\let\baraccent=\= 				% rename builtin command \= to \baraccent
\renewcommand{\=}[1]{\stackrel{#1}{=}} 		% for putting numbers above =
\newtheorem{prop}{Proposition}
\newtheorem{thm}{Theorem}[section]
\newtheorem{lem}[thm]{Lemma}
\theoremstyle{definition}
\newtheorem{dfn}{Definition}
\theoremstyle{remark}
\newtheorem*{rmk}{Remark}

%%%%%%%%%%%%%% STYLE %%%%%%%%%%%%%%%
%\geometry{hmargin={1in,1in},vmargin={2.0in,1.5in}}
\usepackage{fancyhdr}
\pagestyle{fancy}
% \renewcommand{\chaptermark}[1]{\markboth{\MakeUppercase{\chaptername} \ \thechapter. \ \textbf{#1}}{}}
% \renewcommand{\sectionmark}[1]{\markright{\textbf{\thesection \ #1}}}
\fancyhead[RE,LO]{}
% \fancyhead[RO]{\rightmark}
% \fancyhead[LE]{\leftmark}
%%\fancyfoot{}
%%\fancyfoot[RO,LE]{\thepage}
\renewcommand{\footrulewidth}{0.4pt}
% \renewcommand{\headrulewidth}{0.4pt}

\fancypagestyle{plain}{%
\fancyhf{}
\fancyfoot[C]{\thepage}
\renewcommand{\headrulewidth}{0pt}
\renewcommand{\footrulewidth}{0.4pt}}


% configure subfig
\captionsetup[subfloat]{labelformat=simple,listofformat=subsimple}
\def\thesubfigure{(\alph{subfigure})} 
%
%% url
%\makeatletter
%\def\url@leostyle{%
%  \@ifundefined{selectfont}{\def\UrlFont{\sf}}{\def\UrlFont{\small\ttfamily}}}
%\makeatother
%\urlstyle{leo}
%
%
\numberwithin{equation}{section}
%
%%%%%%%%%%%%%%%% CODE LISTING SETTINGS %%%%%%%%%%%%%%%
%\usepackage{listings}
%
%\lstset{
%frame=tb,
%framesep=5pt,
%tabsize=2,
%basicstyle=\scriptsize\ttfamily,
%showstringspaces=false,
%stringstyle=\color{red},
%commentstyle=\color{gray},
%breaklines=true,
%numbers=left,
%numbersep=5pt
%%numberstyle=\ttfamily,
%%keywordstyle=\color{green},
%%identifierstyle=\color{blue},
%%xleftmargin=5pt,
%%xrightmargin=5pt,
%%aboveskip=\bigskipamount,
%%belowskip=\bigskipamount,
%%rulecolor=\color{Gray},
%%numberstyle=\color{blue},
%}

%\lstdefinelanguage{JavaScript} {
%morekeywords={
%break,const,continue,delete,do,while,export,for,in,function,
%if,else,import,in,instanceOf,label,let,new,return,switch,this,
%throw,try,catch,typeof,var,void,with,yield
%},
%sensitive=false,
%morecomment=[l]{//},
%morecomment=[s]{/*}{*/},
%morestring=[b]",
%morestring=[d]'
%}
%
%% MACROS
%\include{macros}

% ***********************************************************
% ********************** END HEADER *************************
% ***********************************************************

% INFO
\author{Jonas Siegl}
\title{Diploma Thesis - Draft}

\begin{document}
%\svnInfo $Id: main.tex 205 2010-03-16 10:23:32Z claas $

\selectlanguage{american}

\maketitle
% \begin{titlepage}
\begin{center}
\vspace{2cm}
  \begin{spacing}{1.1}
    \huge\textbf{Spin-Dependent Transport Simulations in Two-Dimensional Electron Systems}
  \end{spacing}
  \par
  \vspace{.5in}
  {\large vorgelegt von Jonas Siegl}\par
  \vspace{.5in}
  {\Large Diplomarbeit}
  \par

  \vfill
  \vspace{.5in}
  \includegraphics[width=0.7\textwidth]{images/title}
  \par
  %\vspace{-1.1in}
  %\hspace{.6in}
  %{\Large \textcolor{white}{$T=100\,\text{K}$}}
  % \vspace{-0.75in}
  % \hspace{2.7in}
  \vspace{.5in}
  Institut f\"ur Angewandte Physik\\
  Universit\"at Hamburg
  \par
  \vspace{0.5in}
  Januar 2012 
  % {\large \textcolor{gray}{$T=100\,\text{K}$}}
\end{center}
\end{titlepage}

% \thispagestyle{plain} %removes the header 

% emtpy page
\clearpage
\thispagestyle{empty}
\mbox{}
\clearpage
% TABLE OF CONTENTS
\thispagestyle{plain} %removes the header from TOC
\tableofcontents
\clearpage
\section{Introduction}
\section{Theory}
  \subsection{Two Dimensional Electron Gas}
  \subsection{Second Quantization}
  \subsection{Spin}
    \subsubsection{Spin Degree of Freedom}
    \subsubsection{Spin Orbit Interaction}
  \subsection{Electron Transport}
    \subsubsection{Quantized Conduction}
  \subsection{Compare to Transfer Matrix Method}
  The TMM is in general unstable because possible evanescent waves will lead to unphysically fast rising matrix coefficients.
  D.Y.K. Ko and J.C. Inkson reviews a case where the TMM revealed instability flaws (1988).
\section{Method}
  \subsection{Discrete Matrix Representation}
  A practical way to perform the necessary inversion of the \hamil{} to obtain the \gfnc{} is to leverage numerics to obtain an approximation to the solution for
\begin{equation}
	\hat{G} = \left[E-\hat{H} \right]^{-1}\text{.}
  \label{eqn:greensfnccompmethods}
\end{equation}
Spin interactions are neglected to outline the discretization method. They are however included in an analogous way. To use numerical methods to find the inverse in \cref{eqn:greensfnccompmethods} the system has to be discretized. The energy $E$ is discretized by multiplication with the appropriate identity matrix $\mathds{1}$. This ensures the same energy parameter $E_i=E$ for every discrete state $i$. The choice of space and basis to discretize the \hamil{} is in principle arbitrary but significant advantages can be achieved if a discretization method suitable to the system is used. Due to the cubic zincblende crystal structure of the 2DEG host materials, like GaAs or InAs, it is convenient to discretize on a square lattice.
The procedure will be outlined for the \hamil{} in the effective mass $m^*$ approximation
\begin{equation}
  \hat{H}=-\frac{\hbar^{2}}{2m^*}\left[\pdd{}{x}+\pdd{}{y}\right]+U(x,y)\ .
    \label{eqn:Hamil}
\end{equation}

To achieve a discrete matrix-representation of the \hamil{}, self-energies and\sloppy{~\gfnc{} one can expand the equations in any localized basis functions like Muffin-Tin orbitals, \textsc{Wannier} functions, or \textsc{Dirac} distributions giving rise to a so-called tight-binding \hamil{} limiting the interaction, e.g. to nearest neighbors. A coordinate in discrete space will be refered to as a lattice point regardless of the nature and space of the localized function.}
Using a \textsc{Dirac} delta distribution basis amounts to replacing the continuous function with it's values at the respective points in space \cite{JApplPhys.92.3730}.
\begin{equation}
  \int \delta (\v{r}-\v{r}') f(\v{r}) \mathrm{d}\v{r} = f(\v{r}')\ .
  \label{eqn:deltabasis}
\end{equation}
In order to study the influence of geometry on the transport properties of a 2DEG a real-space approach with \textsc{Dirac} delta distributions is chosen. Real-space discretization leads to the replacement of continuous real space $x$ and $y$ directions by an infinite net of lattice points. Here the same spacing $a$ is assumed in both directions. The lattice points are located in the continuous space at the coordinates $x=i*a$ and $y=j*a$ with $i,j$ being integers and $a$ the lattice spacing. The \hamil{} now operates on functions defined on discrete points in space obtained by
\begin{equation}
  T_{i,j} \leftrightarrow T(x=ia,y=ja) \mbox{ and } U_{i,j}\leftrightarrow U(x=ia,y=ja)\ .
  \label{FunctionDescrete}
\end{equation}
With $T_{ij}$ describing the state of the system and $U_{ij}$ being the discrete version of the lateral potential.
The \hamil{} reads
\begin{equation}
  \left[\v{H}T\right]_{x=ia,y=ja}=-\frac{\hbar^{2}}{2m^*}\left[\pdd{T}{x}+\pdd{T}{y}\right]_{x=ia,y=ja}+U_{i,j}T_{i,j}\ .
  \label{DiscreteHamil}
\end{equation}
Using the finite-differences scheme and assuming small $a$ the first derivative in each direction is approximated 
\begin{equation}
  \left[\pd{T}{x}\right]_{x=(i+1/2)a} \sim  \frac{1}{a}\left(T_{i+1}-T_{i}\right)\ .
  \label{ApproximateFirstDerivative}
\end{equation}
In this case the subscript denotes evaluation at a point between two lattice points $x=(i+1/2)$. By applying the concept of finite differences a second time for the second derivative one arrives at
\begin{align}
  \left[\pdd{T}{x}\right]_{x=(i+1/2)a} &\sim \frac{1}{a}\left( \left[\pd{T}{x}\right]_{x=(i+1/2)a}-\left[\pd{T}{x}\right]_{x=(i-1/2)a} \right) \notag \\
  &\sim \frac{1}{a^2}\left(T_{i+1}-2T_{i}+T_{i-1}\right)\ .
  \label{ApproximateSecondDerivative}
\end{align}
Following Nicoli\'c \cite{Nikolic2010} the bras $\bra{\v{m}}$ and kets $\ket{\v{n}}$ denote localized states at the lattice-points. The finite-difference aproximation is  expressend in the second quantization formalism with a point $(i,j) = \v{m}$ as
\begin{align}
	\bra{\v{m}} \pd{}{x} \ket{\v{n}} &= \frac{\braket{\v{m}+1}{\v{n}}-\braket{\v{m}-1}{\v{n}}}{2a}\notag\\ 
				&= \frac{\delta_{\v{n},\v{m}+1}-\delta_{\v{n},\v{m}-1}}{2a}\ .
	\label{eqn:finitedifffirstderivative}
\end{align}
and
\begin{align}
	\bra{\v{m}} \dd{}{x} \ket{\v{n}} &= \frac{\braket{\v{m}+1}{\v{n}}-2\braket{\v{m}}{\v{n}}+\braket{\v{m}-1}{\v{n}}}{a^2} \\
				 &= \frac{\delta_{\v{n},\v{m}+1}-2\delta_{\v{n},\v{m}}+\delta_{\v{n},\v{m}-1}}{a^2}\ .
	\label{eqn:finitediffsecondderivative}
\end{align}
With the help of these relations any discrete one-particle operators can be expressed in second quantization
\begin{align}
	\hat{A} = \sum_{\v{m},\v{n}} \bra{\v{m}}\hat{A}\ket{\v{n}} \hat{c}^{\dagger}_{\v{m}} \hat{c}_{\v{n}}.
 \label{eqn:singleparticleopinsecondquantization}
\end{align}
with the creation and annihilation operators $\hat{c}^{\dagger}_{\v{m}}$ and $\hat{c}_{\v{n}}$ in a localized basis.
Inserting \cref{eqn:finitedifffirstderivative}, \ref{eqn:finitediffsecondderivative} and \ref{eqn:singleparticleopinsecondquantization} leads to the tight-binding version of the \hamil{}
\begin{align}
	\hat{H} = \sum_{\v{m},\sigma} \epsilon_{\v{m}} \hat{c}^{\dagger}_{\v{m}\sigma} \hat{c}_{\v{m}\sigma} +
	\sum_{\v{m},\v{m}',\sigma} t_{\v{m},\v{m}'} \hat{c}^{\dagger}_{\v{m}\sigma} \hat{c}_{\v{m}'\sigma}\ .
	\label{eqn:discretizedeffectivemassHamil}
\end{align}
The sum runs in addition to the states $\v{m}$ over the spin $\sigma=\uparrow,\downarrow$. The $\epsilon$ account for local potentials in the form of $U(x,y)$ like lateral confinement or atomic disorder.
The spin independent hopping parameter $t_{\v{m}\v{m'}}$ is given by
\begin{align}
t_{\v{m}\v{m'}} = \left\{ \begin{array}[c]{cl} -t_0 & \text{if } \v{m} = \v{m}' \pm a \cdot \v{e}_{x,y} \\ 0 & \text{otherwise} \end{array} \right.\ .
	\label{eqn:t0hopping}
\end{align}
The hopping parameter $t_0 = \hbar^2/2m^*a^2$ describes the probability of one electron propagating to a neighboring lattice point. Introducing the discretized \rash{} \hamil{}
\begin{align}
	\hat{H} = \sum_{\v{m},\sigma} \epsilon_{\v{m}} \hat{c}^{\dagger}_{\v{m}\sigma} \hat{c}_{\v{m}\sigma} +
	\sum_{\v{m},\v{m'},\sigma}  \hat{c}^{\dagger}_{\v{m}\sigma} \mat{t}_{\v{m}\v{m'}\sigma\sigma '} \hat{c}_{\v{m'}\sigma}\ .
	\label{eqn:discretizedrashbahamil}
\end{align}
Now the hopping parameter $\mat{t}_{\v{m}\v{m'}\sigma\sigma '}$ is a non-trivial $2 \times 2$ matrix with distinct coefficients depending on hopping direction:
\begin{align}
	\mat{t}_{\v{m}\v{m'}\sigma\sigma '} = \left\{ \begin{array}[c]{cl} -t_0\mathds{1}_S - i t_{SO} \mat{\sigma}_y & \text{if } \v{m} = \v{m}' \pm a \v{e}_{x} \\
		-t_0\mathds{1}_S + i t_{SO} \mat{\sigma}_x & \text{if } \v{m} = \v{m}' \pm a \v{e}_{y} \end{array} \right.\ .
	\label{eqn:tsohopping}
\end{align}
In the matrix representation each element of the spin-less \hamil{} becomes a $2 \times 2$ matrix of itself increasing the number of matrix-elements four-fold.
\begin{figure}[h!]
\centering
\begin{align*}
\mat{H}=
\renewcommand*{\arraystretch}{1.6}
\begin{pmatrix} 
\ddots& & & & & & & & & \\
& \mat{H}_p& \mat{V}_p& & & && &&\\
& \mat{V}^{\dagger}_p& \mat{H}_p& \mat{H}_{p,0}& &\ddots& & &\\
& &\mat{H}_{0,p}&\mat{H}_{0,0} &\mat{H}_{0,1} & &0 & &\\
& & &\mat{H}_{1,0}& &\ddots & &\ddots &\\
& \ddots& & &\ddots &\mat{H}_{N,N} &\mat{H}_{N,q}& & \\
& &0 & & &\mat{H}_{q,N} &\mat{H}_q &\mat{V}_q&\\
& & &\ddots & & &\mat{V}^{\dagger}_q &\mat{H}_q&\\
& & & & & & & &\ddots
\end{pmatrix}
% \includegraphics[width=0.9\textwidth]{images/blocktridiagonal}
\end{align*}
\caption{Block tridiagonal matrix of the tight-binding \hamil{}, $\mat{H}_{i,i}$ denotes the \hamil{} of layer $i$ and $\mat{H}_{i,i+1},\mat{H}_{i+1,i}$the interaction between layers. The matrices $\mat{H}_{p,0},\mat{H}_{0,p},\mat{H}_{q,N}$ and $\mat{H}_{N,q}$ represent the coupling of conductor and lead $p$ and $q$. The matrices $\mat{H}_p,\mat{V}_p,\mat{H}_q,\mat{V}_q$ denote the \hamil s and coupling matrices of leads $p$ and $q$ respectively.}
\label{fig:blocktridiagonal}
\end{figure}
The on-site energies $\epsilon_{\v{m}}$ in this \hamil{} are
\begin{equation}
\epsilon_{\v{m}} = 4t_0 + U_{\v{m}}\ .
\end{equation}
The potential offset $4t_0$ arises naturally in the process of approximation with finite differences. Because of the offset the dispersion relation for free electrons ($V_{\v{m}}=0$) in an infinite empty lattice reads
\begin{equation}
\epsilon = 4t_0(1-\text{cos}(\abs{\v{k}}a))\ .
\end{equation}
The tight-binding model is a good approximation if the lattice spacing is below the \textsc{Fermi}-wavelength
\begin{equation}
\abs{\v{k}}a \ll  2\pi \Leftrightarrow a \ll \lambda_F = \frac{2\pi}{\abs{\v{k}}} \ .
\end{equation}
In this regime the dispersion relation is approximately parabolic as in the continuum case \cite{Metalidis2007Thesis}.
For simple systems with collinear leads the matrix representaion in the given approximation of the \hamil{} takes a natural block tridiagonal form. Each block on the main diagonal belongs to a set of lattice points called a layer. All interactions within the layer correspond to matrix elements within the block. Interactions across layers are found in blocks on the first off-diagonal, see \cref{fig:blocktridiagonal} \cite{AnLunNik2008}.

  \subsection{How to to reach Tractability}
  Although a discrete representation of the \hamil{} is found the matrix inversion is not trivial. The device is an open system and the incorporation of in- and outflow via the leads that stretch to infinity results in an infinite matrix.
Simple truncation of the matrices under consideration would effectively describe a closed system. Via the concept of the self-energy however the influence of the leads may be accurately projected onto the device.
The technique will be outlined following \textsc{Datta} for a system with only one contact as depicted in \cref{fig:selfenergy} but can be applied to an arbitray number of leads. \begin{figure}[h!]
\centering
\begin{tikzpicture}
\node at (0,0)[above] {\includegraphics[scale=0.7]{images/selfenergy}};
\draw[style={latex-latex,line width=2pt}] (-4,1) node[right]{$y$} -- +(0,-1) node [below left]{$z$} -- +(1,-1) node[below]{$x$};
\draw (-4,0) circle (4pt) [fill=black];
% \draw[latex-latex,line width=1.5pt]  (0,0) to (2,0);
\draw[latex-latex, line width=1.5pt] (-1,3.4) node (x) {}
            +(1,0) node (y) {}
            (node cs:name=x) .. controls +(-0.2,0.8) and +(0.2,0.8) ..
            (node cs:name=y);
\node at (-0.5,4.5) {$\tau$} ;
\end{tikzpicture}
\caption{Separation of conductor and lead. The dots denote lattice points. The white area represents the semi-infinite lead with the \hamil{} $\mat{H}_L$, the gray area denotes the conductor with the \hamil{} $\mat{H}_C$. The arrow indicates the coupling by the matrix $\mat{\tau}$.}
\label{fig:selfenergy}
\end{figure}
\subsection{The Self Energy}
The \gfnc{} matrix can be partitioned as
\begin{align}
  \begin{bmatrix}
  \mat{G}_{L} & \mat{G}_{L/C}\\
  \mat{G}_{C/L} & \mat{G}_{C}
  \end{bmatrix}
  =
  \begin{bmatrix}
  (E+i\eta)\mat{1}_{L} - \mat{H}_L  & -\v{\tau}^{\dagger} \\
	-\v{\tau} & E\mat{1} - \mat{H}_C
  \end{bmatrix}^{-1}\ .
  \label{eqn:greendivided}
\end{align}
Here $\mat{G}_L$ is the infinite \gfnc{} matrix of the isolated lead, $\mat{G}_{L/C}$ and $\mat{G}_{C/L}$ represent inifinite \gfnc s matrices between lead and conductor. The finite \gfnc{} of the isolated conductor is $\mat{G}_L$. The coupling matrices $\tau$ are only nonzero on the interface of lead and conductor, confer \cref{fig:selfenergy}.
Basic algebraic transformations yield \cite{Datta1997}
\begin{align}
\mat{G}_C& = \left[E\mat{1}-\mat{H}_C -\v{\tau}^{\dagger} \mat{g}_L^r \v{\tau} \right]^{-1} \ .
\label{eqn:finitegreensfunction}
\end{align}
With the so-called retarded \emph{lead surface \gfnc{}}
\begin{align}
\mat{g}_L^r = [(E+i\eta)\mat{1}_L - \mat{H}_L]^{-1}
\label{eqn:leadsurfacegfnc}
\end{align}
one can write the \emph{self energy} term $\mat{\Sigma}_L$ as
\begin{align}
\mat{\Sigma}_L = \v{\tau}^{\dagger} \mat{g}_L^r \v{\tau}\ .
\label{eqn:sigmal}
\end{align}
The concept of the self energy allows for further inclusions of interactions like phonon or impurity scattering. In contrast to the theoretical accurate self energy of the lead the self energy due to scattering is usually only feasible to compute approximatly. The self energy includes a left lead $\mat{\Sigma}_{Ll}$ and a right lead $\mat{\Sigma}_{Lr}$ and for example a scattering term $\mat{\Sigma}_{\text{scat}}$. With this the \gfnc{} can be written as
\begin{align}
\mat{G}_C& = \left[E\mat{1}-\mat{H}_C -\mat{\Sigma}_{Ll}-\mat{\Sigma}_{Lr}-\mat{\Sigma}_{\text{scat}} \right]^{-1} \ .
\label{eqn:finitegreensfunctionwithselfenergy}
\end{align}
This expression is the central equation to evaluate in the numerical calculation of the \gfnc{} of the 2DEG. 
\subsection{Lead Surface \cgfncs{}}
Even with the separation of the lead surface \gfnc{} the direct evaluation of \cref{eqn:finitegreensfunctionwithselfenergy} is still unfeasible as it still includes $\mat{H}_L$ via \cref{eqn:sigmal} and \cref{eqn:leadsurfacegfnc} which is infinite.
If translation invariance along the lead exists, the so-called \textsc{Ando} method \cite{PhysRevB.44.8017} computes the surface \gfnc{} for a homogeneous lead consisting of repeating supercells, i.e. the smallest layer that exhibits periodicity.\par
For simple systems without fields and spin phenomena \cref{eqn:leadsurfacegfnc} can be calculated analytically \cite{Datta1997}. For leads including fields and further interactions it has to be calculated semi-analytically via the \emph{eigenstate-decomposition} of their \emph{companion matrix} \cite{PhysRevB.25.3975} or \emph{transfer matrix} \cite{PhysRevB.55.5266} \cite{PhysRevB.66.205319}
\begin{align}
  \mat{C} =
  \begin{bmatrix}
  \mat{0}  &\mat{1} \\
  -\mat{H}_{+}^{-1}\mat{H}_{-} &\mat{H}_{+}^{-1} (E\mat{1} - \mat{H}_{\perp})
  \end{bmatrix}\ .
  \label{eqn:companionmatrix}
\end{align}
Here $\mat{H}_{\pm}$ denotes matrices governing the interaction between slices with \hamil{} $\mat{H}_{\perp}$.
For non-homogeneous leads one can apply a continued fraction method \cite{Velev2004} that connects \gfncs{} from one layer to neighboring layers and repeats that process until the interactions between layers becomes negligible.\par
For spin-dependent simulations the leads need to include fields and spin-orbit effects and analytical methods do not suffice. With the appropriate gauge one can ensure invariance along the lead and therefore a modification of the eigenstate-decomposition method is employed.\par
The lead surface \gfnc{} is expanded in eigenstates of an \emph{invariant subspace} of unidirectional modes which yields higher numerical stability as expansions in eigenstates of the full \hamil{} with the same result \cite{Wimmer2009JComPhys}.\par
One obtains the lead surface \gfnc{} by first calculating the \textsc{Schur} decomposition of matrix $\mat{C}$ into the diagonal matrix $\mat{D}$ and the unitary matric $\mat{Q}$
\begin{align}
\mat{C} = \mat{Q}\mat{D}\mat{Q}^{\dagger}\ .
\label{eqn:schurdecomposition}
\end{align}
One reorders matrix \mat{D} such that the eigenvalues corresponding to modes traveling either towards or away from the scattering region align in the upper left quarter. Only these modes are then used to construct the surface \gfnc{}
\begin{align}
\mat{g}^r_L = \mat{Q}_{llq}\mat{Q}_{ulq}^{-1}\mat{H}_{-}^{-1}\ .
\end{align}
With $\mat{Q}_{llq}$ and $\mat{Q}_{ulq}$ represent the lower-left and upper-left quarter of $\mat{Q}$ respectively. 

\section{Implementation}
\subsection{Natural Ordering}
\subsection{Graph Representation}
\subsection{Implementaion of RGA}
pseudocode
\subsection{Observables from Retarded Green's Function}
\subsection{Validation}
\subsection{Computer Language}
\section{Results}
\section{Conclusion}
\section{Outlook}
\section{Appendix}

\section{Questions}
\begin{itemize}
  \item temperature $\rightarrow$ 0 necessary?
  \item How is the coupling to the magnetic field done?
  \item What are the Channel coefficients \cite{PhysRevB.31.6207}
\end{itemize}

\clearpage
\bibliographystyle{amsalpha}
\bibliography{thesis}

\end{document}
