% ***********************************************************
% ******************* PHYSICS HEADER ************************
% ***********************************************************
%\documentclass[11pt]{article} 
% \documentclass[12pt,twoside,a4paper]{article}
\documentclass[12pt,twoside,a4paper]{scrartcl}
\setkomafont{sectioning}{\normalcolor\bfseries}
%%%%%%%%%%%%%% PACKAGE INCLUDES %%%%%%%%%%%%%%%
\usepackage[utf8]{inputenc} %allows input of utf8 characters
\usepackage[american,ngerman]{babel}
\usepackage[T1]{fontenc} %more complete but lower quality than ComMod fontpacke
% \usepackage{ae,aecompl}
\usepackage[tbtags]{mathtools}
%\usepackage{amsmath} % AMS Math Package
\usepackage{amsthm} % Theorem Formatting
\usepackage{amssymb}	% Math symbols such as \mathbb
\usepackage{graphicx} % Allows for eps images
\usepackage{multirow}
\usepackage{multicol} % Allows for multiple columns
\usepackage{color} % well, supplies colors
\usepackage{todonotes}
\usepackage{booktabs}
\usepackage{setspace} %Zeilenabstand
%\onehalfspacing
% \usepackage{mathptmx}
% \usepackage{bbold}
% \usepackage{bold-extra} % Ugly bitmap fonts, but holds bold glyphs!
% \usepackage{kpfonts} % Pretty complete fontpackage but looks 'different'
\usepackage[
    pdftitle={Spin-dependend transport simulations in two-dimensionsal electron systems},    % title
    pdfauthor={Jonas Siegl},     % author
    pdfsubject={Quantum Transport},   % subject of the document
    pdfcreator={Jonas Siegl},   % creator of the document
    pdfproducer={Jonas Siegl}, % producer of the document
    pdfkeywords={Green's Function } {Numerical } {Transport}, % list of keywords
    % linktocpage=true,
    linkcolor=red,          % color of internal links
    citecolor=green,        % color of links to bibliography
    filecolor=magenta,      % color of file links
    urlcolor=cyan,           % color of external links
pagebackref=true]{hyperref}
\usepackage[all]{hypcap} % shifts the link point to the top of tables or figures so it is visible when clicked
%\usepackage{array}
%\usepackage[final]{svninfo}
\usepackage{tabulary}
\usepackage{tabularx}
%\usepackage{ifthen}
%\usepackage[dvips]{graphicx}
%\usepackage[small,bf]{caption}
\usepackage{subfig} 		%enabled the use of subfloats within a float
%\usepackage{textcomp}
%\usepackage{pstricks}
%\usepackage{courier}
%\usepackage{geometry}
%\usepackage{url}
\usepackage{tikz}
\tikzstyle{every picture}+=[remember picture]
\usetikzlibrary{decorations.pathreplacing}
%\usepackage{paralist}
%\usepackage{xspace}
%\usepackage[dvips,letterpaper,margin=0.75in,bottom=0.5in]{geometry}
 % Sets margins and page size
%\pagestyle{empty} % Removes page numbers
%\makeatletter % Need for anything that contains an @ command 
%\renewcommand{\maketitle} % Redefine maketitle to conserve space
%{ \begingroup \vskip 10pt \begin{center} \large {\bf \@title}
%	\vskip 10pt \large \@author \hskip 20pt \@date \end{center}
%  \vskip 10pt \endgroup \setcounter{footnote}{0} }
%\makeatother % End of region containing @ commands
\renewcommand{\labelenumi}{(\alph{enumi})} 		% Use letters for enumerate

% Shortcuts to Named Objects
\newcommand{\gfnc}{\textsc{Green}'s function} 		% call with {} like \gfnc{} to ensure following whitespace
\newcommand{\cgfnc}{\textsc{Green}'s Function} 		% call with {} like \gfnc{} to ensure following whitespace
\newcommand{\hamil}{\textsc{Hamilton}ian}
\newcommand{\rash}{\textsc{Rashba}}
\newcommand{\sdg}{\textsc{Schr\"odinger} equation}
\newcommand{\lanbform}{\textsc{Landauer-B\"uttiker} formalism}
\newcommand{\clanbform}{\textsc{Landauer-B\"uttiker} Formalism}

% Vectors and matrices
\let\vaccent=\v 					% rename builtin command \v{} to \vaccent{}
\renewcommand{\v}[1]{\ensuremath{\boldsymbol{#1}}} 	% for vectors
\newcommand{\mat}[1]{\ensuremath{\boldsymbol{#1}}} 	% for matrices
\newcommand{\uv}[1]{\ensuremath{\mathbf{\hat{#1}}}} 	% for unit vector
\newcommand{\abs}[1]{\left| #1 \right|} 		% for absolute value
\newcommand{\avg}[1]{\left< #1 \right>} 		% for average
\providecommand{\abs}[1]{\lvert#1\rvert}
\providecommand{\norm}[1]{\lVert#1\rVert}

% Derivatives: 
\let\underdot=\d 					% rename builtin command \d{} to \underdot{}
\renewcommand{\d}[2]{\frac{d #1}{d #2}} 		% for derivatives
\newcommand{\dd}[2]{\frac{d^2 #1}{d #2^2}} 		% for double derivatives
\newcommand{\pd}[2]{\frac{\partial #1}{\partial #2}} 	% for partial derivatives
\newcommand{\pdd}[2]{\frac{\partial^2 #1}{\partial #2^2}} 	% for double partial derivatives
\newcommand{\pdc}[3]{\left( \frac{\partial #1}{\partial #2} \right)_{#3}} % for thermodynamic partial derivatives

%Dirac style
\newcommand{\ket}[1]{\left| #1 \right>} 	% for Dirac kets
\newcommand{\bra}[1]{\left< #1 \right|} 	% for Dirac bras
\newcommand{\braket}[2]{\left< #1 \vphantom{#2} \right| \left. #2 \vphantom{#1} \right>} % for Dirac brackets
\newcommand{\matrixel}[3]{\left< #1 \vphantom{#2#3} \right| #2 \left| #3 \vphantom{#1#2} \right>} % for Dirac matrix elements

%Gradient and curls
\newcommand{\grad}[1]{\v{\nabla} #1} 		% for gradient
\let\divsymb=\div 				% rename builtin command \div to \divsymb
\renewcommand{\div}[1]{\v{\nabla} \cdot #1} 	% for divergence
\newcommand{\curl}[1]{\v{\nabla} \times #1}	% for curl

\let\baraccent=\= 				% rename builtin command \= to \baraccent
\renewcommand{\=}[1]{\stackrel{#1}{=}} 		% for putting numbers above =
\newtheorem{prop}{Proposition}
\newtheorem{thm}{Theorem}[section]
\newtheorem{lem}[thm]{Lemma}
\theoremstyle{definition}
\newtheorem{dfn}{Definition}
\theoremstyle{remark}
\newtheorem*{rmk}{Remark}

%%%%%%%%%%%%%% STYLE %%%%%%%%%%%%%%%
%\geometry{hmargin={1in,1in},vmargin={2.0in,1.5in}}
\usepackage{fancyhdr}
\pagestyle{fancy}
% \renewcommand{\chaptermark}[1]{\markboth{\MakeUppercase{\chaptername} \ \thechapter. \ \textbf{#1}}{}}
% \renewcommand{\sectionmark}[1]{\markright{\textbf{\thesection \ #1}}}
\fancyhead[RE,LO]{}
% \fancyhead[RO]{\rightmark}
% \fancyhead[LE]{\leftmark}
%%\fancyfoot{}
%%\fancyfoot[RO,LE]{\thepage}
\renewcommand{\footrulewidth}{0.4pt}
% \renewcommand{\headrulewidth}{0.4pt}

\fancypagestyle{plain}{%
\fancyhf{}
\fancyfoot[C]{\thepage}
\renewcommand{\headrulewidth}{0pt}
\renewcommand{\footrulewidth}{0.4pt}}


% configure subfig
\captionsetup[subfloat]{labelformat=simple,listofformat=subsimple}
\def\thesubfigure{(\alph{subfigure})} 
%
%% url
%\makeatletter
%\def\url@leostyle{%
%  \@ifundefined{selectfont}{\def\UrlFont{\sf}}{\def\UrlFont{\small\ttfamily}}}
%\makeatother
%\urlstyle{leo}
%
%
\numberwithin{equation}{section}
%
%%%%%%%%%%%%%%%% CODE LISTING SETTINGS %%%%%%%%%%%%%%%
%\usepackage{listings}
%
%\lstset{
%frame=tb,
%framesep=5pt,
%tabsize=2,
%basicstyle=\scriptsize\ttfamily,
%showstringspaces=false,
%stringstyle=\color{red},
%commentstyle=\color{gray},
%breaklines=true,
%numbers=left,
%numbersep=5pt
%%numberstyle=\ttfamily,
%%keywordstyle=\color{green},
%%identifierstyle=\color{blue},
%%xleftmargin=5pt,
%%xrightmargin=5pt,
%%aboveskip=\bigskipamount,
%%belowskip=\bigskipamount,
%%rulecolor=\color{Gray},
%%numberstyle=\color{blue},
%}

%\lstdefinelanguage{JavaScript} {
%morekeywords={
%break,const,continue,delete,do,while,export,for,in,function,
%if,else,import,in,instanceOf,label,let,new,return,switch,this,
%throw,try,catch,typeof,var,void,with,yield
%},
%sensitive=false,
%morecomment=[l]{//},
%morecomment=[s]{/*}{*/},
%morestring=[b]",
%morestring=[d]'
%}
%
%% MACROS
%\include{macros}

% ***********************************************************
% ********************** END HEADER *************************
% ***********************************************************

% INFO
\author{Jonas Siegl}
\title{Spin-Dependent Transport Simulations in Two-Dimensional Electron Systems}

\begin{document}
%\svnInfo $Id: main.tex 205 2010-03-16 10:23:32Z claas $

\selectlanguage{american}

% \maketitle
\begin{titlepage}
\begin{center}
\vspace{2cm}
  \begin{spacing}{1.1}
    \huge\textbf{Spin-Dependent Transport Simulations in Two-Dimensional Electron Systems}
  \end{spacing}
  \par
  \vspace{.5in}
  {\large vorgelegt von Jonas Siegl}\par
  \vspace{.5in}
  {\Large Diplomarbeit}
  \par

  \vfill
  \vspace{.5in}
  \includegraphics[width=0.7\textwidth]{images/title}
  \par
  %\vspace{-1.1in}
  %\hspace{.6in}
  %{\Large \textcolor{white}{$T=100\,\text{K}$}}
  % \vspace{-0.75in}
  % \hspace{2.7in}
  \vspace{.5in}
  Institut f\"ur Angewandte Physik\\
  Universit\"at Hamburg
  \par
  \vspace{0.5in}
  Januar 2012 
  % {\large \textcolor{gray}{$T=100\,\text{K}$}}
\end{center}
\end{titlepage}

% \thispagestyle{plain} %removes the header 

% emtpy page
% \clearpage
% \thispagestyle{empty}
% \mbox{}
% \clearpage
% TABLE OF CONTENTS
\pagestyle{bib} %removes the header from TOC
\tableofcontents
\clearpage
\pagestyle{abstract}
\addcontentsline{toc}{section}{Abstract}
\section*{Abstract}
Nano-Spintronics involves the manipulation and control of electrons and their spin in solid-state systems.\par
With the discovery of the \emph{giant magnetoresistance} by \textsc{Albert Fert} and \textsc{Peter Gr\"unberg} in 1988 the manipulation of electron transport properties by exploiting the spin degree of freedom became apparent. This breakthrough, awarded with the \textsc{Nobel} Prize in Physics in 2007, became the starting point for the study and manipulation of the spin in electronic devices \cite{evegeny2010spin}. A prominent device proposal is the \textsc{Datta-Das} \emph{spin field effect transistor} \cite{datta:665}. The conductance of such mesoscopic systems was found to be quantized \cite{PhysRevLett.45.494} at low temperatures and high magnetic fields revealing its quantum mechanical origin. For finite temperatures the spin field effect transistor has only recently been realized as a proof of concept \cite{Wunderlich24122010}.\par
Central to the realization of the spin field effect transistor is the \emph{two-dimensional electron gas} (2DEG). The 2DEG describes a system of non-interacting electrons confined to a two-dimensional region by a highly localized potential well. In devices within the 2DEG spatial constrictions called \emph{quantum point contacts} (QPC) are used to control carrier density and spin through the effects of quantum confinement. Quantum point contacts are subject to active fundamental research as well as current applications in detectors \cite{PhysRevB.67.161308} or quantum bits \cite{PhysRevA.57.120}.\par
In this thesis a quantum mechanical simulator based on the \emph{non-equilibrium \gfnc{} formalism} is developed. The basic numerical computations are sped up using the \emph{recursive \gfnc{} algorithm} (RGA) and extended by a graph-based matrix reordering scheme enabling the simulation of 2DEGs of arbitrary geometry.
The electron density $n$, spin density $s_z$ and conductance $G$ are computed for QPCs of transformed geometry and non-colinear systems with rotated leads. The simulations show how changes in the geometry can be used to alter the local electron and spin densities in the device.
The shape of the quantum point contacts has a strong influence on the quantization of the conductance profile. In the investigated \textsc{Aharonov-Bohm} interferometer the spin polarization in the ring is determined by the positions of the leads. The access to microscopic quantities by the simulator can be used to investigate possibilities of spin control in 2DEGs.

\clearpage
\section*{Kurzfassung}
\addcontentsline{toc}{section}{Kurzfassung}
In der Nano-Spintronik steht die Manipulation und Kontrolle von Elektronen und ihres Spins im Mittelpunkt.\par
Mit der Endeckung des \emph{GMR-Effekts} durch \textsc{Albert Fert} und \textsc{Peter Gr\"unberg} in 1988 wurde die Manipulation des Elektronentransports durch die Nutzung des Spinfreiheitsgrads offenkundig. Dieser wissenschaftliche Durchbruch, f\"ur den der \textsc{Nobel}preis in 2007 verliehen wurde, markiert den Anfang des Studiums und der Manipulation des Spins in elektronischen Bauelementen~\cite{evegeny2010spin}. Der \textsc{Datta-Das} \emph{Spin-Feldeffekttransistor} geh\"ort zu den bekanntesten zu realisiernenden Bauelementen der Spintronik~\cite{datta:665}. Es stellte sich heraus, dass die Leitf\"ahigkeit eines solchen mesoskopischen Bauelements in starken magnetischen Feldern und unter niedrigen Temperaturen quantisiert ist \cite{PhysRevLett.45.494}. Hierdurch wurde der quantenmechanische Ursprung der Leitf\"ahigkeit deutlich. F\"ur endliche Temperaturen wurde der Spin-Feldeffekttransistor erst k\"urzlich experimentell umgesetzt~\cite{Wunderlich24122010}.\par
Der zentrale Bestandteil des Spin-Feldeffekttransistors ist das \emph{zweidimensionale Elektronengas} (2DEG). Das 2DEG beschreibt ein System nicht wechselwirkender Elektronen, die durch ein stark lokalisiertes Potential in ihrer Bewegung auf einen zweidimensionalen Raumbereich beschr\"ankt sind. Die Ladungstr\"agerdichte und der Elektronenspin im 2DEG k\"onnen durch Quantenpunktkontakte \"uber die Effekte des Quanten-Confinements beeinflusst werden. Quantenpunktkontakte sind sowohl Gegenstand aktiver Grundlagenforschung als auch von Bedeutung in experimentellen Umsetzungen von Detektoren~\cite{PhysRevB.67.161308} und Quantenbits~\cite{PhysRevA.57.120}.\par
Bestandteil dieser Arbeit ist die Entwicklung eines quantenmechanischen Simulators auf Basis des Formalismus der \textsc{Green}schen Funktionen im Nichtgleichgewicht. Die numerischen Berechnungen werden durch die Anwendung des \emph{rekursiven \textsc{Green}sche Funktionen Algorithmus} beschleunigt. Desweiteren erm\"oglicht die Anwendung einer graphbasierten Methode zur Matrix Umordnung die Simulation von 2DEGs beliebiger Geometrie. Die Elektronendichte $n$, die Spindichte $s_z$ und die Leitf\"ahigkeit $G$ werden sowohl f\"ur Quantenpunktkontakte unterschiedlicher Geometrie als auch f\"ur nicht kollineare Systeme mit gegeneinander rotierten Zuleitungen berechnet. Aus den Ergebnissen der Simulation ist ersichtlich, dass durch \"Anderungen in der Geometrie Einfluss auf die lokalen Elektronen- und Spindichten genommen werden kann. Ausserdem hat die Form des Quantenpunktkontaktes erheblichen Einfluss auf die Quantisierung des Leitf\"ahigkeitprofils. In dem untersuchten \textsc{Aharonov-Bohm} Interferometer ist die Spinausrichtung der Elektronen im Ring gegeben durch die Position der Zuleitungen. Der Zugang zu mikroskopischen Gr\"o\ss en durch den Simulator erm\"oglicht die Erforschung von Methoden zur Spinmanipulation in zweidimensionalen Elektronengasen .

\clearpage
\pagestyle{main}
\chapter{Theory of Quantum Transport}
The transport properties of the electrons and their spin in two dimensional electron system is considered. This chapter presents the \textsc{Landauer-B\"uttiker} formalism describing the conductance and the \gfnc{} formalism which permits the calculation of electron and spin densities.
\section{Quantized Conduction}
  In transport measurements of mesoscopic systems at low temperatures a quantized conduction has been observed \cite{PhysRevLett.45.494}. Thus conductance of mesoscopic systems exhibits peculiarities with no classical analogon.
\subsection{\clanbform{}}\label{sec:landauerbuettiker}
In an ideal lattice the electrons would move like in vacuum (empty lattice) but with an effective mass. Impurities or phonons introduce scattering into the system altering the momentum of moving electrons through collisions.
If the size of the electronic device shrinks below the mean free path of an electron, i.e. the distance an electron has to travel until its initial momentum is destroyed, the transport becomes ballistic\,\cite{datta1989quantum}. 
\begin{align}
	\text{length of device} \lesssim \lambda_{\text{mfp}}\quad \Rightarrow \quad\text{ballistic regime}
	\label{eqn:meanfreepath}
\end{align}
Due to the lack of scattering one would suspect zero resistivity. Nonetheless a finite resistivity which is quantized as a function of the width of the conductor was found \cite{PhysRevLett.60.848}.
A popular approach to the effects of nano-scale devices initialized by \textsc{Landauer} \cite{PhilMag.21.863} and extended by \textsc{B\"uttiker} is the so called \lanbform{} \cite{PhysRevB.31.6207}.
\begin{figure}[h]
\centering
\begin{tikzpicture}
      % \draw[step=.5cm,gray,very thin] (-4,0) grid (4,4);
      \node at (0,0)[above] {\includegraphics[width=.5\textwidth]{images/landauer}};
      \node at (-4,4)[below] {$T_1,\mu_1,f_1$};
      \node at (3,4)[below] {$T_2,\mu_2,f_2$};
      \node at (4,0.5)[above] {$T_3,\mu_3,f_3$};
      \node at (-0.1,2)[align=center] {Scattering\\ region};
\end{tikzpicture}
\caption{Schematic of device in \lanbform{} with three reservoirs and leads. The blue area is the active scattering region. Each reservoir has distinct Temperature $T_i$, chemical potential $\mu_i$ and therefore \textsc{Fermi} distribution $f_i$.}
\label{fig:lanbform}
\end{figure}
The \textsc{Landauer-B\"uttiker} model is in essence quite general but all devices are composed of a scattering region and one or more leads. The device of interest is theoretically divided into a scattering area henceforth called conductor and multiple leads connecting the conductor to macroscopic reservoirs. All reservoirs and their adjacent leads are assumed to be in thermal equilibrium. Each lead has a distinct \textsc{Fermi} distribution $f_R$ of temperature $T_R$ and chemical potential $\mu_R$, see \cref{fig:lanbform}.
The \lanbform{} relates the current flow between leads to the chemical potential of the reservoirs. In hindsight of the inclusion of spin the expressions are given per spin state of the electron. Otherwise one would have to include a 2 in the calculations for spin degeneracy. The current in lead $p$ \cite{PhysRevLett.68.2512}
\begin{align}
I_p=\frac{-e}{h} \sum_q \int T_{pq}(E) [f_p(E) - f_q(E)] \text{ d}E.
\label{eqn:current}
\end{align}
results from the leads $q$ with their respective \textsc{Fermi} distributions $f_i(E)$ via the tranmission coefficients $T_{pq}(E)$. The current includes the electron charge $e$ and the \textsc{Planck} constant $\hbar$. The \lanbform{} relates the conductance of the device with the possibility of an electron passing through it. The transmission coefficients $T_{pq}(E)$ state the probability of an electron traveling from lead $q$ to lead $p$. 
In the linear response regime with low temperatures the expression can be linearized 
\begin{align}
I_p=\frac{-e}{h} \sum_q T_{pq}(E_F) [\mu_p - \mu_q].
\label{eqn:currentlin}
\end{align}
In the low temperature regime the \textsc{Fermi} distributions can be replaced by their respective \emph{chemical potentials} $\mu_i$. The possibility that the energy $E$ can be replaced by the \textsc{Fermi} energy $E_F$ shows that the conductance is effectively a \emph{\textsc{Fermi} surface} property depending only on states near the \textsc{Fermi} energy.
\subsection{Conductance from Transmission}\label{sec:conductancefromtransmission}
Assuming that the conductance can be expressed in terms of current and chemical potential results in a relation between the conductance from lead $q$ to lead $p$ and the transmission coefficients 
\begin{align}
G_{pq}=\frac{\abs{I\cdot e}}{(\mu_p - \mu_q)}=\frac{e^2}{h} T_{pq}(E_F).
\label{eqn:conductance}
\end{align}
The conductance can be calculated by quantum mechanical methods. The transmission coefficients $T_{pq}(E)$ are directly related to the \emph{transmission probabilities amplitudes} $t^{pq}_{ll'}$ leading to
\begin{align}
G_{pq}=\frac{e^2}{h} T_{pq} =\frac{e^2}{h} \sum_{ll'} \abs{t^{pq}_{ll'}}^2.
\label{eqn:transcoeff}
\end{align}
The transmission probability amplitudes $\abs{t^{pq}_{ll'}}$ describe the electron flux amplitude for an electron traveling from channel $l'$ in lead $q$ to channel $l$ in lead $p$. This definition only holds for leads that are longitudinally translational invariant. The transmission coefficients are matrix elements of a so-called \emph{scattering wavefunction} which governs the electron flow from and to the leads \cite{Datta1997}. There exists an alternative method to direct quantum mechanical evaluation of the scattering wave function to obtain the transmission coefficients as will be shown in \cref{sec:observables}.

  \section{Two Dimensional Electron Gas}
    A two-dimensional electron gas (2DEG) is characterized by the confinement of electron motion in one direction and free motion in the other two directions. 
Electron systems with high lateral mobility can be realized by, e.g. transistors, surfaces of suited materials \cite{PhysRevLett.12.271} or heterojunctions \cite{JVSTB.4.853}. Structures containing 2DEGs are very versatile as they are easily fabricated by molecular beam epitaxy and modified by ion beam litho\-graphy \cite{Ingram1995}\cite{Nowack2009Thesis} or can even be rolled up \cite{Vorob'ev2004171}. 
\begin{figure}[!h]
\centering
\subfloat[]{\label{fig:heterostructure}\includegraphics[scale=0.15]{images/heterostructure}} \quad\quad
\subfloat[]{\label{fig:2degseparated}\includegraphics[scale=0.2]{images/2DEG_separated}} \quad\quad
\subfloat[]{\label{fig:2degjoined}\includegraphics[scale=0.2]{images/2DEG_combined}}
\caption{Heterostrucure with 2DEG at material boundary.(a) 2DEG (blue surface) in a heterojunction between AlGaAs (orange layer) and GaAs (green layer).(b) Bandstructure of separated AlGaAs and GaAs layers with corresponding energy of the conductance band $E_C$, valance band $E_V$, and \textsc{Fermi} energy $E_F$. The arrows denote the difference of energy of the conductance and valance bands.(c) Bending of conductance and valence bands of joined layers. The arrow denotes the position of the 2DEG below the \textsc{Fermi} energy. Slightly modified from \textsc{Datta} \cite{Datta1997}.}
\label{fig:hetero2deg}
\end{figure}
The 2DEG forms in the interior of the heterostructure between two layers, see \cref{fig:heterostructure}. In the case of heterojunctions at the interfaces of a heterostructure the tight confinement can be realized for example by joining two semiconducting materials with distinct band gaps and \textsc{Fermi} energies. When separated the band structures exhibit distinct \textsc{Fermi} energies, \cref{fig:2degseparated}. When the materials are joined the conduction and valance bands bend at the interface to match the respective \textsc{Fermi} energies of both materials, as illustrated in \cref{fig:2degjoined}.
The bending of the conductance band below the \textsc{Fermi} energy of the combined heterostructure creates a very pronounced dip in the bandstructure resulting in a potential well where the electrons can move freely. The confinement perpendicular to the 2DEG however leads to quantized energy levels of motion in $z$ direction as can be seen in \cref{fig:potentialwell} (a) and (b) for three states with different cutoff energies, i.e. different band bottoms. The offset conduction bands can often be treated independently. Usually the lowest energy level comprises the 2DEG \cite{Datta1997}.
\begin{figure}[t]
\centering
\subfloat[]{\label{fig:modesinzconf1}\begin{tikzpicture}\node at (0,0) [above]{\includegraphics[scale=0.55]{images/modesinzconf1}};
\node at (0.7,-0.4){$z$};
\node at (-1.5,2){$n=1$};
\node at (-1.5,4.4){$n=2$};
\node at (-1.5,5.6){$n=3$};
\node at (-1.5,7.2){$E$};
\end{tikzpicture}}
\subfloat[]{\label{fig:modesinzconf2}\begin{tikzpicture}\node at (0,0) [above]{\includegraphics[scale=0.55]{images/modesinzconf2}};
\node at (0,-0.4){$\v{k}$};
\node at (0.7,1.5){$E_1(\v{k})$};
\node at (0.7,3.7){$E_2(\v{k})$};
\node at (0.7,6.9){$E_3(\v{k})$};
\draw[latex-latex] (-2,0.2)--node[right]{$E_g$}(-2,1.7);
\end{tikzpicture}}
% \includegraphics[width=0.9\textwidth]{images/davies.png}
\caption{Modes in $z$-confined potential well. (a) Eigenstate densities in a potential well for energy quantum number $n$ and cut-off energies $E_1$,$E_2$,$E_3$ in dependence on the position. (b) Conductance bands in the potential well in k-space. The ground state energy is denoted by $E_g$.}
\label{fig:potentialwell}
\end{figure}

      \subsection{Effective Mass \hamil{}}
	The wavefunction and eigenvalues of the electrons in the conduction band describe the properties of the 2DEG. In the 2DEG the wavefuctions can be separated into a lateral ($x,y$) and perpendicular ($z$) part. Each perpendicular wavefunction belongs to a different cut-off energy $E_{\text{cut-off}}$, see \cref{fig:potentialwell}(b).
In steady state, the dynamics of the electrons in the separated lateral band can be described by the time independent \sdg{}.
\begin{align}
 \hat{H}_0\ket{\psi(x,y)} = E \ket{\psi(x,y)}\ .
	\label{eqn:effectivemasssdg}
\end{align}
The two dimensional single-band effective mass \hamil{} of orbital motion $\hat{H}_0$
\begin{align}
\hat{H}_0 = E_{\text{cut-off}} + \frac{1}{2m^*}(\hat{p}_{x}+\hat{p}_{y})^2+U(x,y)\ ,
\end{align}
includes the cut-off energy $E_{\text{cut-off}}$, the effective mass $m^*$, the momentum operator $\hat{p}_i = i\hbar \pd{}{i} + e\mat{A}$ with an appropriate vector field $\v{A}(x,y)$, the electron charge $e$, and the static potential $U(x,y)$.
The lateral geometry is included via the static potential $U(x,y)$.
This approximation yields smoothed out solutions due to the inclusion of the periodic lattice potential via an effective mass $m^*$ \cite{BastardBrum1986}.
The cut-off energy $E_{\text{cut-off}}$ includes the energy shift of the lowest occupied subband of the 2DEG and will be set to $E_{\text{cut-off}} = 0$ in all further calculations.
From the solutions of the single-band effective mass \sdg{} one can extract physical quantities suitable for comparison with experimental results called observables.
A central value of interest is the probability distribution
\begin{align}
	n(x,y) = \left< \psi^{\dagger} (x,y) \psi(x,y)\right>\ .
	\label{eqn:analyticalelectrondensity}
\end{align}
of the wavefunctions $\psi$. Here the expectation value has to be taken over the product of the wavefunctions.\par

	The dynamics of the electrons in the conductance band are subject to numerous quantummechanical phenomena and interactions which may be included via additive influences like a spin interaction \hamil{} $H_{\text{int}}$ making up the single particle single-band effective mass \hamil{}
\begin{align}
\hat{H} = \hat{H}_0 + \hat{H}_{\text{int}}+\dotsb \label{eqn:generalhamil}\ .
\end{align}
To obtain a more realistic picture than the spin degenerate system the effects of the internal spin degree of freedom of the electrons are included.
\subsection{Spin Degree of Freedom}
The inclusion of the electron spin degree of freedom is achieved by the \textsc{Kronecker} multiplication of the single particle single-band effective mass \hamil{}  with the $2 \times 2$ spin identity matrix $\mat{I}_S$. 
\begin{align}
\hat{H} \rightarrow \hat{H} \otimes \mat{I}_S\ .
\end{align}
When a particle in vacuum moves through an electric field its orbital and spin degrees of freedom are coupled. This essentially relativistic effect is known as spin-orbit interaction. If the \textsc{Dirac} equation is non-relativistically expanded in powers of the inverse of the speed of light $c$, one finds as a first order correction \cite{Nowack2009Thesis}
\begin{align}
\hat{H}_{SO} = \frac{1}{2m_0c^2} \v{\sigma} \otimes \left( \nabla V \times \frac{\v{p}}{m_0}\right)\ .
\label{eqn:spinorbithamil}
\end{align}
Here $m_0$ is the electron rest mass, $\v{\sigma} = (\sigma_x, \sigma_y,\sigma_z)$ the spin vector of \textsc{Pauli}-matrices and $V$ describes the electric field. 
\subsection{Spin Orbit Interaction}
For the motion of electrons in a crystal lattice the relativistic interaction of electrons and their spin give rise to spin-orbit interactions as the electron moves through the field of the atomic cores.
The potential that is responsible for the $z$-confinement of the 2DEG interacts with the lateral motion of the electrons. The nonzero gradient of the effective electrical field $V$ in \cref{eqn:spinorbithamil} gives rise to the \rash{} \hamil{}:
\begin{align}
  \hat{H}_{RSO} =\frac{\alpha_{RSO}}{\hbar}(\hat{p}_{y} \otimes \hat{\sigma}_{x} - \hat{p}_{x} \otimes \hat{\sigma}_{y})\ .
	\label{eqn:rashbahamiltonian}
\end{align}
The parameter $\alpha_{RSO}$ governs the strength of the interaction and has its origin in the lack of structural inversion symmetry of the boundary layer \cite{PhysRevB.70.233311}.
As can be seen in \cref{fig:hetero2deg} the confining potential is not strictly symmetric leading in addition to finite-wall effects of the potential well to a material constant \cite{Metalidis2007Thesis}
\begin{align}
\alpha_{RSO} = \frac{\hbar}{2m_0c^2} \left<\pd{V}{z} \right>\ .
\end{align}
The term $\left<\pd{V}{z} \right>$ denotes a spatial average required by the spatial dependence of $V$. The average is quite intricate as the effect depends on the detailed bandstructure including also the influence of holes \cite{JApplPhys.83.4324}.
As the \rash{} effect is a suitable candidate to realize certain spintronic applications it is worth noting that the strength of the \textsc{Rashba} parameter $\alpha_{RSO}$ and therefore the properties of conduction can be tuned by the addition of a gate electrode on top of the 2DEG altering the $z$ confinement potential \cite{PhysRevLett.78.1335}.
Although other corrections to the single-band effective-mass \hamil{} like the \textsc{Dresselhaus} spin-orbit coupling \cite{PhysRev.100.580} may be considered in small band-gap materials like GaAs and InAs the spin-orbit coupling is largely dominated by the \rash{} effect \cite{PhysRevB.61.15588}. As this work focuses on 2DEGs in such materials no further contributions to the \hamil{} will be added.\par
With the effects of the inherent spin degree of electrons considered it is now possible to compute additional observables. The spatial probability distribution of up or down pointing electron-spin with respect to an arbitrary but fixed quantization axis $i$ is for example given by the spin-density \cite{JPhysA:MathGen.18.671}
\begin{align}
S^i  = \frac{\hbar}{2} \left< \psi^{\dagger} \hat{\sigma}_i \psi \right>\ .
\label{eqn:spindensity}
\end{align}
which includes the spin-operator $\hat{\sigma}_i$ in the corresponding direction, i.e. in matrix representation a \textsc{Pauli} matrix and the \textsc{Pauli} spinors $\psi,\psi^{\dagger}$. The arguments of the wavefunctions will be suppressed for brevity from now on.

  \section{\cgfnc{} Formalism}
  \todo[noline]{metalidis. haug, now there will be vectors $(x,y)\rightarrow \v{r}$}
Due to the size of mesoscopic systems the calculation of its properties should be treated as a many-body problem. This leads to a many-body \hamil{} like:
\begin{align}
\mathcal{H}(\v{r};t) = \mathcal{H}_0(\v{r}) + \mathcal{H}_{\text{int}}(\v{r})+\mathcal{H}_{\text{ext}}(t)
\end{align}
Here the $\mathcal{H}_0$ indicates a non-interacting system and the other terms introduce the interactions and possible external time dependend pertubation. 
All information about the systems may be obtained by solving its time-dependend \sdg{}:
\begin{align}
i\hbar \pd{}{t} \ket{\psi(\v{r};t)} = \mathcal{H}(\v{r};t) \ket{\psi(\v{r};t)}
\label{eqn:timedependendsdg}
\end{align}
Currently such large ensembles cannot be solved exactly so some kind of physically motivated approximation is needed. In the regime of high electron density and strongly coupled external influences one can model a 2DEG reasonable well in terms of non-interactig particles in an effective potential\cite{fetter2003quantum}. \par
How does one go to find the quantum mechanical terms to compute the desired observables? It turns out it is advantageous to skip the computation of eigenstates used in the previous chapters and to use a technique to compute the transport properties in a direct way. One such technique which combines a powerful and general formalism and is especially suited to be numerically implemented is the non-equilibrium \gfnc{} formalism (NEGF) developed by \textsc{Kadanoff} and \textsc{Baym} \cite{kadanoff1962quantum} and refined by \textsc{Keldysh} \cite{keldysh1965}. This is especially the case as several efficient algorithms exist that compute for example the electron density or the transmission coefficients one of which will be outlined in chapter \ref{sec:recursivegreenfncalgorithm}.\par
The non-equilibrium \gfnc{} formalism is very powerful and as it is based on a quantum-field theoretical approach capable of describing non-equilibrium systems with interactions. As this might seems excessive in face of the relative simplicity of the problem at hand and a rigorous derivation would obscure its central results only key concepts neccessary for the computation of the desired observables shall be discussed following \textsc{Wimmer}s(math) \cite{Wimmer2009Thesis} and \textsc{Datta}(physics) \cite{Datta1997}.\par
\subsection{Equilibrium \cgfnc s}
In the field of electrodynamics the \gfnc{} where first introduced as a means to solve a inhomogeneous differential equation of the form
\begin{align}
\hat{LD}G(\v{r},\v{r}';t,t') = \delta(\v{r}-\v{r}')\delta(t-t')
\label{eqn:lineardiff}
\end{align}
where $\hat{LD}$ is any linear differential operator. It \todo{in the appendix?} is easily shown that the proper \gfnc{} for non-interacting many-body systems is identical to the \gfnc{} obtained by solving the single-particle effective mass \sdg{} \cite{ferry1999transport}:
\begin{align}
\left[ E -\hat{H}(\v{r})\right] \hat{G}(\v{r},\v{r}';E) = \delta (\v{r}-\v{r}')
\end{align}
For equilibrium \cite{fetter2003quantum} and non-equilibrium steady state \cite{haug2008quantum} the \gfnc{} only depends on time differencens and can therefore be \textsc{Fourier} transformed to the energy domain ($t \rightarrow E$). The \gfnc{} that solves the above equation is actually not unique. Two solutions can be found. One can be considered a wavefunction at point $\v{r}$ as a result from a unit excitation at point $\v{r}'$. The other describes the situation with the points exchanged. In simple systems this would in the first case correspond to an incoming wave in point $\v{r}$ in the other to an outgoing wave.
Those two solutions are called retarded and advanced \gfnc{} respectively and can be distinguished by the addition of an infinitismal imaginary part $i\eta$.
\begin{align}
\left[ E \pm i \eta -\hat{H}(\v{r})\right] \hat{G}^{r(a)}(\v{r},\v{r}';E) = \delta (\v{r}-\v{r}')
\end{align}
This equation can be formally inverted and the \gfnc{} can be defined in the second quantization in terms of an operator identity as:
\begin{align}
\hat{G}^{r(a)}(\v{r},\v{r}';E) = \bra{\v{r}} [E\pm i \eta - \hat{H}(\v{r})]^{-1} \ket{\v{r}'}
\end{align}
Positive for the retarded and negative for the advanced \gfnc{}. One sees immediately that the retarded and advanced \gfnc{} are the complex conjugate from each other why the superscripts will be dropped and in non-ambiguous cases the retarded \gfnc{} will be referred to simply as \gfnc{}
\begin{align}
\hat{G}^r = (\hat{G}^a)^+ \equiv \hat{G}
\end{align}
\subsection{Non-Equilibrium \cgfnc s}
In the case of an non-interacting system in equilibrium and also to some degree for non-equilibrium systems with interactions the retarded \gfnc{} describes the spectrum that is the eigenstates and eigenenergies but one needs to know how this spectrum is filled.\par 
There exist numerous but interdependent variations of \gfnc{} each one especially suited to access certain information about the system.
The second \gfnc{} usually employed in transport calculations is the so called \emph{lesser \gfnc{}} \cite{haug2008quantum}.\par
Within a non-equilibirum pertubation theory describing the dependence of the system on the full \hamil{} $\mathcal{H}$ only in terms of the non-interacting \hamil{} $\mathcal{H}_0$\cite{Jauho2006} the lesser \gfnc{} can be expressed in terms of the retarded \gfnc{}. With the retarded self-energy term $\hat{\Sigma}_x$ which will be discussed in further detail in chapter (\ref{sec:tractability}) the following expression describes the strength of the coupling of the central scattering region to the leads \cite{datta2005quantum}:
\begin{align}
\hat{\Gamma}_p = i\left[\hat{\Sigma}_p(E)-\hat{\Sigma}_p^+(E)\right]
\end{align}
with the lesser self-energy
\begin{align}
\hat{\Sigma}^<=i\left[\hat{\Gamma}_l(E)f_l(E)+\hat{\Gamma}_r(E)f_r(E) \right]
\end{align}
one can write the lesser \gfnc{} in a compact expression known as the \textsc{Keldysh} equation:
\begin{align}
\hat{G}^<(E) = \hat{G}^r(E) \hat{\Sigma}^<(E) \hat{G}^a(E)
\label{eqn:keldyshequation}
\end{align}
For example the electron density  $n(\v{r})$ can be obtained easily from the lesser \gfnc{}:
\begin{align}
	n(\v{r}) = \left< \psi^+ (\v{r}) \psi(\v{r})\right> = -i\hbar \hat{G}^<(\v{r},\v{r};E)
	\label{eqn:edensfromgreensfnc}
\end{align}


  \section{Comparison of Transport Models}
  The \gfnc{} formulation of quantum mechanics is not the only possible method to compute the transport properties desired.\par
There exist numerous approaches to analytical and numerical calculation of quantum mechanical observalbles only a selection will be presented here. A detailed discussion of the Formulations presented in tab.(\ref{tab:comparison}) can be found in \cite{Biegel97quantumelectronic}. The table lists the capabilities and numerical efficiency in a direct computation. It includes for reference the classical \textsc{Boltzmann} Transport Equation (BTE), the \sdg{} Equation (SE), the Transfer Matrix (TM), Density Matrix (DM), \cgfnc{} and \textsc{Wigner} Function formulations.\par
\begin{table}[!ht]
\centering
\begin{tabulary}{\textwidth}{l c c c c c c}\toprule
& \multicolumn{6}{c}{Formulations}  \\ \cmidrule{2-7}
Characteristic &BTE& SE&TM&DM&GF&WF\\ \midrule
State Function Bases &yes&yes&yes&yes&yes&yes  \\
Far-From-Equilibrium &yes&yes&yes&yes&yes&yes  \\
Irreversibility      &yes&no&no&yes&yes&yes\\
Transient Simulation &yes&yes&no&yes&yes&yes \\
Absorbing Boundaries &yes&no&yes&yes&yes&yes \\
Computational Efficiency &3&4&4&4&2&3 \\
Intuitive State Function &5&3&4&3&2&4 \\ \bottomrule
\end{tabulary}
\caption{Comparison of quantum system analysis approaches, see \cite{Biegel97quantumelectronic}. In the ranking 5 = good and 1 = poor.}
\label{tab:comparison}
\end{table}
The most obvious method the direct solution of the \sdg{} is unsuitable for a couple of reasons.In a many body theory it soon becomes unmanagebly complex for a rising number of carriers inside the conductor. Also absorbing boundary conditions and inelastic scattering have not been treated accurately \cite{Biegel97quantumelectronic} although developement has not halted \cite{JApplPhys.69.7153}\cite{gullapalli:2971}\par
A popular quantum device simulation method capable of describing steady state transport in a fast and straight forward to implement way is the so called Transfer Matrix Method (TMM)\cite{MacKinnon2003}.\par
TMM formulations are in general unstable because possible evanescent waves will lead to unphysically fast rising matrix coefficients \cite{PhysRevB.38.9945}.
While the instability issues can be resolved using a redesigned TMM technique by T.Usuki \cite{PhysRevB.50.7615}\cite{PhysRevB.52.8244} as a direct descendend of the \sdg{} the TMM presents the same theoretical limitations.\par
A different approach in spirit related to the solution of classical electronic systems via a \textsc{Boltzmann} transport equation present quantum statistical formulations. 
Due to the nature of many body systems a statistical state function even in the limit of non-interacting particle should present a quite accurate representation of the 2DEG.\par
The choice of a \gfnc{} as a state function has several advantages over a \textsc{Wigner} function or density matrix approach.
The more general formalism makes it more flexible and powerful because it also profits from advances in the solution to \gfnc{} which can be applied in a diverse field.\par
The the application of all quantum statistical formulations is currently limited to systems of reduced size or resolution as the computations involved are highly sensitive to the discretization.\par
An advantage of the \gfnc{} method over other state functions in relation to the implementation on a computer is the development of fast and efficient algorithms suitable to compute all the desired observables\cite{JApplPhys.91.2343} in two or even three dimensions or in a parallel computing environment \cite{Drouvelis2006parallel}.
\todo[noline]{ small chatpers ? cite pourfath 3.10 \cite{Pourfath2007Thesis}}


\chapter{Computational Methods}
This chapter will outline the numerical and mathematical methods employed in the calculation of transport properties.
  \section{Discrete Matrix Representation}\label{sec:discretematrixrep}
  A practical way to perform the necessary inversion of the \hamil{} to obtain the \gfnc{} is to leverage numerics to obtain an approximation to the solution for
\begin{equation}
	\hat{G} = \left[E-\hat{H} \right]^{-1}\text{.}
  \label{eqn:greensfnccompmethods}
\end{equation}
Spin interactions are neglected to outline the discretization method. They are however included in an analogous way. To use numerical methods to find the inverse in \cref{eqn:greensfnccompmethods} the system has to be discretized. The energy $E$ is discretized by multiplication with the appropriate identity matrix $\mathds{1}$. This ensures the same energy parameter $E_i=E$ for every discrete state $i$. The choice of space and basis to discretize the \hamil{} is in principle arbitrary but significant advantages can be achieved if a discretization method suitable to the system is used. Due to the cubic zincblende crystal structure of the 2DEG host materials, like GaAs or InAs, it is convenient to discretize on a square lattice.
The procedure will be outlined for the \hamil{} in the effective mass $m^*$ approximation
\begin{equation}
  \hat{H}=-\frac{\hbar^{2}}{2m^*}\left[\pdd{}{x}+\pdd{}{y}\right]+U(x,y)\ .
    \label{eqn:Hamil}
\end{equation}

To achieve a discrete matrix-representation of the \hamil{}, self-energies and\sloppy{~\gfnc{} one can expand the equations in any localized basis functions like Muffin-Tin orbitals, \textsc{Wannier} functions, or \textsc{Dirac} distributions giving rise to a so-called tight-binding \hamil{} limiting the interaction, e.g. to nearest neighbors. A coordinate in discrete space will be refered to as a lattice point regardless of the nature and space of the localized function.}
Using a \textsc{Dirac} delta distribution basis amounts to replacing the continuous function with it's values at the respective points in space \cite{JApplPhys.92.3730}.
\begin{equation}
  \int \delta (\v{r}-\v{r}') f(\v{r}) \mathrm{d}\v{r} = f(\v{r}')\ .
  \label{eqn:deltabasis}
\end{equation}
In order to study the influence of geometry on the transport properties of a 2DEG a real-space approach with \textsc{Dirac} delta distributions is chosen. Real-space discretization leads to the replacement of continuous real space $x$ and $y$ directions by an infinite net of lattice points. Here the same spacing $a$ is assumed in both directions. The lattice points are located in the continuous space at the coordinates $x=i*a$ and $y=j*a$ with $i,j$ being integers and $a$ the lattice spacing. The \hamil{} now operates on functions defined on discrete points in space obtained by
\begin{equation}
  T_{i,j} \leftrightarrow T(x=ia,y=ja) \mbox{ and } U_{i,j}\leftrightarrow U(x=ia,y=ja)\ .
  \label{FunctionDescrete}
\end{equation}
With $T_{ij}$ describing the state of the system and $U_{ij}$ being the discrete version of the lateral potential.
The \hamil{} reads
\begin{equation}
  \left[\v{H}T\right]_{x=ia,y=ja}=-\frac{\hbar^{2}}{2m^*}\left[\pdd{T}{x}+\pdd{T}{y}\right]_{x=ia,y=ja}+U_{i,j}T_{i,j}\ .
  \label{DiscreteHamil}
\end{equation}
Using the finite-differences scheme and assuming small $a$ the first derivative in each direction is approximated 
\begin{equation}
  \left[\pd{T}{x}\right]_{x=(i+1/2)a} \sim  \frac{1}{a}\left(T_{i+1}-T_{i}\right)\ .
  \label{ApproximateFirstDerivative}
\end{equation}
In this case the subscript denotes evaluation at a point between two lattice points $x=(i+1/2)$. By applying the concept of finite differences a second time for the second derivative one arrives at
\begin{align}
  \left[\pdd{T}{x}\right]_{x=(i+1/2)a} &\sim \frac{1}{a}\left( \left[\pd{T}{x}\right]_{x=(i+1/2)a}-\left[\pd{T}{x}\right]_{x=(i-1/2)a} \right) \notag \\
  &\sim \frac{1}{a^2}\left(T_{i+1}-2T_{i}+T_{i-1}\right)\ .
  \label{ApproximateSecondDerivative}
\end{align}
Following Nicoli\'c \cite{Nikolic2010} the bras $\bra{\v{m}}$ and kets $\ket{\v{n}}$ denote localized states at the lattice-points. The finite-difference aproximation is  expressend in the second quantization formalism with a point $(i,j) = \v{m}$ as
\begin{align}
	\bra{\v{m}} \pd{}{x} \ket{\v{n}} &= \frac{\braket{\v{m}+1}{\v{n}}-\braket{\v{m}-1}{\v{n}}}{2a}\notag\\ 
				&= \frac{\delta_{\v{n},\v{m}+1}-\delta_{\v{n},\v{m}-1}}{2a}\ .
	\label{eqn:finitedifffirstderivative}
\end{align}
and
\begin{align}
	\bra{\v{m}} \dd{}{x} \ket{\v{n}} &= \frac{\braket{\v{m}+1}{\v{n}}-2\braket{\v{m}}{\v{n}}+\braket{\v{m}-1}{\v{n}}}{a^2} \\
				 &= \frac{\delta_{\v{n},\v{m}+1}-2\delta_{\v{n},\v{m}}+\delta_{\v{n},\v{m}-1}}{a^2}\ .
	\label{eqn:finitediffsecondderivative}
\end{align}
With the help of these relations any discrete one-particle operators can be expressed in second quantization
\begin{align}
	\hat{A} = \sum_{\v{m},\v{n}} \bra{\v{m}}\hat{A}\ket{\v{n}} \hat{c}^{\dagger}_{\v{m}} \hat{c}_{\v{n}}.
 \label{eqn:singleparticleopinsecondquantization}
\end{align}
with the creation and annihilation operators $\hat{c}^{\dagger}_{\v{m}}$ and $\hat{c}_{\v{n}}$ in a localized basis.
Inserting \cref{eqn:finitedifffirstderivative}, \ref{eqn:finitediffsecondderivative} and \ref{eqn:singleparticleopinsecondquantization} leads to the tight-binding version of the \hamil{}
\begin{align}
	\hat{H} = \sum_{\v{m},\sigma} \epsilon_{\v{m}} \hat{c}^{\dagger}_{\v{m}\sigma} \hat{c}_{\v{m}\sigma} +
	\sum_{\v{m},\v{m}',\sigma} t_{\v{m},\v{m}'} \hat{c}^{\dagger}_{\v{m}\sigma} \hat{c}_{\v{m}'\sigma}\ .
	\label{eqn:discretizedeffectivemassHamil}
\end{align}
The sum runs in addition to the states $\v{m}$ over the spin $\sigma=\uparrow,\downarrow$. The $\epsilon$ account for local potentials in the form of $U(x,y)$ like lateral confinement or atomic disorder.
The spin independent hopping parameter $t_{\v{m}\v{m'}}$ is given by
\begin{align}
t_{\v{m}\v{m'}} = \left\{ \begin{array}[c]{cl} -t_0 & \text{if } \v{m} = \v{m}' \pm a \cdot \v{e}_{x,y} \\ 0 & \text{otherwise} \end{array} \right.\ .
	\label{eqn:t0hopping}
\end{align}
The hopping parameter $t_0 = \hbar^2/2m^*a^2$ describes the probability of one electron propagating to a neighboring lattice point. Introducing the discretized \rash{} \hamil{}
\begin{align}
	\hat{H} = \sum_{\v{m},\sigma} \epsilon_{\v{m}} \hat{c}^{\dagger}_{\v{m}\sigma} \hat{c}_{\v{m}\sigma} +
	\sum_{\v{m},\v{m'},\sigma}  \hat{c}^{\dagger}_{\v{m}\sigma} \mat{t}_{\v{m}\v{m'}\sigma\sigma '} \hat{c}_{\v{m'}\sigma}\ .
	\label{eqn:discretizedrashbahamil}
\end{align}
Now the hopping parameter $\mat{t}_{\v{m}\v{m'}\sigma\sigma '}$ is a non-trivial $2 \times 2$ matrix with distinct coefficients depending on hopping direction:
\begin{align}
	\mat{t}_{\v{m}\v{m'}\sigma\sigma '} = \left\{ \begin{array}[c]{cl} -t_0\mathds{1}_S - i t_{SO} \mat{\sigma}_y & \text{if } \v{m} = \v{m}' \pm a \v{e}_{x} \\
		-t_0\mathds{1}_S + i t_{SO} \mat{\sigma}_x & \text{if } \v{m} = \v{m}' \pm a \v{e}_{y} \end{array} \right.\ .
	\label{eqn:tsohopping}
\end{align}
In the matrix representation each element of the spin-less \hamil{} becomes a $2 \times 2$ matrix of itself increasing the number of matrix-elements four-fold.
\begin{figure}[h!]
\centering
\begin{align*}
\mat{H}=
\renewcommand*{\arraystretch}{1.6}
\begin{pmatrix} 
\ddots& & & & & & & & & \\
& \mat{H}_p& \mat{V}_p& & & && &&\\
& \mat{V}^{\dagger}_p& \mat{H}_p& \mat{H}_{p,0}& &\ddots& & &\\
& &\mat{H}_{0,p}&\mat{H}_{0,0} &\mat{H}_{0,1} & &0 & &\\
& & &\mat{H}_{1,0}& &\ddots & &\ddots &\\
& \ddots& & &\ddots &\mat{H}_{N,N} &\mat{H}_{N,q}& & \\
& &0 & & &\mat{H}_{q,N} &\mat{H}_q &\mat{V}_q&\\
& & &\ddots & & &\mat{V}^{\dagger}_q &\mat{H}_q&\\
& & & & & & & &\ddots
\end{pmatrix}
% \includegraphics[width=0.9\textwidth]{images/blocktridiagonal}
\end{align*}
\caption{Block tridiagonal matrix of the tight-binding \hamil{}, $\mat{H}_{i,i}$ denotes the \hamil{} of layer $i$ and $\mat{H}_{i,i+1},\mat{H}_{i+1,i}$the interaction between layers. The matrices $\mat{H}_{p,0},\mat{H}_{0,p},\mat{H}_{q,N}$ and $\mat{H}_{N,q}$ represent the coupling of conductor and lead $p$ and $q$. The matrices $\mat{H}_p,\mat{V}_p,\mat{H}_q,\mat{V}_q$ denote the \hamil s and coupling matrices of leads $p$ and $q$ respectively.}
\label{fig:blocktridiagonal}
\end{figure}
The on-site energies $\epsilon_{\v{m}}$ in this \hamil{} are
\begin{equation}
\epsilon_{\v{m}} = 4t_0 + U_{\v{m}}\ .
\end{equation}
The potential offset $4t_0$ arises naturally in the process of approximation with finite differences. Because of the offset the dispersion relation for free electrons ($V_{\v{m}}=0$) in an infinite empty lattice reads
\begin{equation}
\epsilon = 4t_0(1-\text{cos}(\abs{\v{k}}a))\ .
\end{equation}
The tight-binding model is a good approximation if the lattice spacing is below the \textsc{Fermi}-wavelength
\begin{equation}
\abs{\v{k}}a \ll  2\pi \Leftrightarrow a \ll \lambda_F = \frac{2\pi}{\abs{\v{k}}} \ .
\end{equation}
In this regime the dispersion relation is approximately parabolic as in the continuum case \cite{Metalidis2007Thesis}.
For simple systems with collinear leads the matrix representaion in the given approximation of the \hamil{} takes a natural block tridiagonal form. Each block on the main diagonal belongs to a set of lattice points called a layer. All interactions within the layer correspond to matrix elements within the block. Interactions across layers are found in blocks on the first off-diagonal, see \cref{fig:blocktridiagonal} \cite{AnLunNik2008}.

  \section{How to reach Tractability}\label{sec:tractability}
  Although a discrete representation of the \hamil{} is found the matrix inversion is not trivial. The device is an open system and the incorporation of in- and outflow via the leads that stretch to infinity results in an infinite matrix.
Simple truncation of the matrices under consideration would effectively describe a closed system. Via the concept of the self-energy however the influence of the leads may be accurately projected onto the device.
The technique will be outlined following \textsc{Datta} for a system with only one contact as depicted in \cref{fig:selfenergy} but can be applied to an arbitray number of leads. \begin{figure}[h!]
\centering
\begin{tikzpicture}
\node at (0,0)[above] {\includegraphics[scale=0.7]{images/selfenergy}};
\draw[style={latex-latex,line width=2pt}] (-4,1) node[right]{$y$} -- +(0,-1) node [below left]{$z$} -- +(1,-1) node[below]{$x$};
\draw (-4,0) circle (4pt) [fill=black];
% \draw[latex-latex,line width=1.5pt]  (0,0) to (2,0);
\draw[latex-latex, line width=1.5pt] (-1,3.4) node (x) {}
            +(1,0) node (y) {}
            (node cs:name=x) .. controls +(-0.2,0.8) and +(0.2,0.8) ..
            (node cs:name=y);
\node at (-0.5,4.5) {$\tau$} ;
\end{tikzpicture}
\caption{Separation of conductor and lead. The dots denote lattice points. The white area represents the semi-infinite lead with the \hamil{} $\mat{H}_L$, the gray area denotes the conductor with the \hamil{} $\mat{H}_C$. The arrow indicates the coupling by the matrix $\mat{\tau}$.}
\label{fig:selfenergy}
\end{figure}
\subsection{The Self Energy}
The \gfnc{} matrix can be partitioned as
\begin{align}
  \begin{bmatrix}
  \mat{G}_{L} & \mat{G}_{L/C}\\
  \mat{G}_{C/L} & \mat{G}_{C}
  \end{bmatrix}
  =
  \begin{bmatrix}
  (E+i\eta)\mat{1}_{L} - \mat{H}_L  & -\v{\tau}^{\dagger} \\
	-\v{\tau} & E\mat{1} - \mat{H}_C
  \end{bmatrix}^{-1}\ .
  \label{eqn:greendivided}
\end{align}
Here $\mat{G}_L$ is the infinite \gfnc{} matrix of the isolated lead, $\mat{G}_{L/C}$ and $\mat{G}_{C/L}$ represent inifinite \gfnc s matrices between lead and conductor. The finite \gfnc{} of the isolated conductor is $\mat{G}_L$. The coupling matrices $\tau$ are only nonzero on the interface of lead and conductor, confer \cref{fig:selfenergy}.
Basic algebraic transformations yield \cite{Datta1997}
\begin{align}
\mat{G}_C& = \left[E\mat{1}-\mat{H}_C -\v{\tau}^{\dagger} \mat{g}_L^r \v{\tau} \right]^{-1} \ .
\label{eqn:finitegreensfunction}
\end{align}
With the so-called retarded \emph{lead surface \gfnc{}}
\begin{align}
\mat{g}_L^r = [(E+i\eta)\mat{1}_L - \mat{H}_L]^{-1}
\label{eqn:leadsurfacegfnc}
\end{align}
one can write the \emph{self energy} term $\mat{\Sigma}_L$ as
\begin{align}
\mat{\Sigma}_L = \v{\tau}^{\dagger} \mat{g}_L^r \v{\tau}\ .
\label{eqn:sigmal}
\end{align}
The concept of the self energy allows for further inclusions of interactions like phonon or impurity scattering. In contrast to the theoretical accurate self energy of the lead the self energy due to scattering is usually only feasible to compute approximatly. The self energy includes a left lead $\mat{\Sigma}_{Ll}$ and a right lead $\mat{\Sigma}_{Lr}$ and for example a scattering term $\mat{\Sigma}_{\text{scat}}$. With this the \gfnc{} can be written as
\begin{align}
\mat{G}_C& = \left[E\mat{1}-\mat{H}_C -\mat{\Sigma}_{Ll}-\mat{\Sigma}_{Lr}-\mat{\Sigma}_{\text{scat}} \right]^{-1} \ .
\label{eqn:finitegreensfunctionwithselfenergy}
\end{align}
This expression is the central equation to evaluate in the numerical calculation of the \gfnc{} of the 2DEG. 
\subsection{Lead Surface \cgfncs{}}
Even with the separation of the lead surface \gfnc{} the direct evaluation of \cref{eqn:finitegreensfunctionwithselfenergy} is still unfeasible as it still includes $\mat{H}_L$ via \cref{eqn:sigmal} and \cref{eqn:leadsurfacegfnc} which is infinite.
If translation invariance along the lead exists, the so-called \textsc{Ando} method \cite{PhysRevB.44.8017} computes the surface \gfnc{} for a homogeneous lead consisting of repeating supercells, i.e. the smallest layer that exhibits periodicity.\par
For simple systems without fields and spin phenomena \cref{eqn:leadsurfacegfnc} can be calculated analytically \cite{Datta1997}. For leads including fields and further interactions it has to be calculated semi-analytically via the \emph{eigenstate-decomposition} of their \emph{companion matrix} \cite{PhysRevB.25.3975} or \emph{transfer matrix} \cite{PhysRevB.55.5266} \cite{PhysRevB.66.205319}
\begin{align}
  \mat{C} =
  \begin{bmatrix}
  \mat{0}  &\mat{1} \\
  -\mat{H}_{+}^{-1}\mat{H}_{-} &\mat{H}_{+}^{-1} (E\mat{1} - \mat{H}_{\perp})
  \end{bmatrix}\ .
  \label{eqn:companionmatrix}
\end{align}
Here $\mat{H}_{\pm}$ denotes matrices governing the interaction between slices with \hamil{} $\mat{H}_{\perp}$.
For non-homogeneous leads one can apply a continued fraction method \cite{Velev2004} that connects \gfncs{} from one layer to neighboring layers and repeats that process until the interactions between layers becomes negligible.\par
For spin-dependent simulations the leads need to include fields and spin-orbit effects and analytical methods do not suffice. With the appropriate gauge one can ensure invariance along the lead and therefore a modification of the eigenstate-decomposition method is employed.\par
The lead surface \gfnc{} is expanded in eigenstates of an \emph{invariant subspace} of unidirectional modes which yields higher numerical stability as expansions in eigenstates of the full \hamil{} with the same result \cite{Wimmer2009JComPhys}.\par
One obtains the lead surface \gfnc{} by first calculating the \textsc{Schur} decomposition of matrix $\mat{C}$ into the diagonal matrix $\mat{D}$ and the unitary matric $\mat{Q}$
\begin{align}
\mat{C} = \mat{Q}\mat{D}\mat{Q}^{\dagger}\ .
\label{eqn:schurdecomposition}
\end{align}
One reorders matrix \mat{D} such that the eigenvalues corresponding to modes traveling either towards or away from the scattering region align in the upper left quarter. Only these modes are then used to construct the surface \gfnc{}
\begin{align}
\mat{g}^r_L = \mat{Q}_{llq}\mat{Q}_{ulq}^{-1}\mat{H}_{-}^{-1}\ .
\end{align}
With $\mat{Q}_{llq}$ and $\mat{Q}_{ulq}$ represent the lower-left and upper-left quarter of $\mat{Q}$ respectively. 

  \section{2DEG Properties from \cgfnc s in Tight-Binding}\label{sec:observables}
  To calculate the desired transport properties of the device in the discrete tight binding approximation the expressions for transmission eqn. (\ref{eqn:transcoeff}), electron denisty eqn. (\ref{eqn:analyticalelectrondensity}) and spin denisty eqn. (\ref{eqn:spindensity}) have to be recast in terms of steady-state tight-binding quantities, namely spin dependent \gfnc s.\par
The energy dependent local density of states (LDOS) is written in terms of the retarded \gfnc{}\cite{AnLunNik2008}:
\begin{align}
\text{LDOS}_{\v{m}}(E)=-\frac{1}{\pi} \text{Im}(\text{Tr}_S[\mat{G}_{\v{mm}}^r(E)])
\label{eqn:ldos}
\end{align}
To obtain real space iformation the spin degree of freedom has to be reducted. This is done by taking the \emph{partial trace}\,\cite{Jacobs} of the operator in question effectively adding up the density of state for spin up and spin down electrons in this case.\par
It is important to note that due to equation \ref{eqn:finitegreensfunctionwithselfenergy} the energy in the calculations becomes a free parameter.  Because the system under consideration is open there are no predetermined eigenenergies.\par
The simulation does not yield the energy of the system as one could suspect because of the relation to the \sdg{}. The \gfnc{} measures the \emph{response of the system} to electron states in one of the leads of a given energy $E$.\par
The index (\v{mm}) shows that the density of states can be calculated only from the diagonal of the \gfnc{}.
The electron density can easily be obtained from the lesser \gfnc{} given as an example in eqn. (\ref{eqn:edensfromgreensfnc}). For experimentally measurable quantities an integration over energy has to be performed, which can be divided into equilibrium and non-equilibrium parts if a distinction between total and conducting electrons is desired.
\begin{align}
	\expv{n_{\v{m}}} = \frac{1}{2\pi} \int_{-\infty}^{\infty}\text{d}E\text{Tr}_S[\mat{G}_{\v{mm}}^<(E)]
	\label{eqn:gfncdensity}
\end{align}
The lesser \gfnc{} also allows the definition and calculation  of a local spin density by multiplication with the \textsc{Pauli} matrix $\mat{\sigma}_i$, here for the out of plane polarisation $\expv{s^z_{\v{m}}}$\,\cite{Wimmer2009Thesis}:
\begin{align}
\expv{s^z_{\v{m}}} = \frac{\hbar}{4 \pi i} \int^{\infty}_{-\infty}\text{d}E\text{ Tr}_S\left[\mat{\sigma}_z\mat{G}^<_{\v{mm}}\right]
\label{eqn:gfncspindensity}
\end{align}
In the tight-binding picture one can also define a bond spin current operator \cite{EPL.80.47001}. The bond spin current is the amount of spin polariation flowing between two adjacent lattice-points.
\begin{align}
\expv{\v{j}_{\v{m}\v{m}'}^{s_z(tot)}} &=\expv{\v{j}_{\v{m}\v{m}'}^{s_z(eq)}}+ \expv{\v{j}_{\v{m}\v{m}'}^{s_z(neq)}}\\
	&=\frac{t_{SO}}{2} \int_{E_{\text{cut-off}}}^{E_F-eV/2} \frac{\text{d}E}{2 \pi} \text{Tr}_S \left[\mat{\sigma}_z\left(\mat{G}^<_{\v{m}'\v{m}}(E)- \mat{G}^<_{\v{m}\v{m}'}(E)\right)\right]\\
	&+\frac{t_{SO}}{2} \int_{E_F-eV/2}^{E_F-eV/2} \frac{\text{d}E}{2 \pi} \text{Tr}_S \left[\mat{\sigma}_z\left(\mat{G}^<_{\v{m}'\v{m}}(E)- \mat{G}^<_{\v{m}\v{m}'}(E)\right)\right]
	\label{eqn:gfnccurrent}
\end{align}
With the help of the the so called \textsc{Fisher-Lee} relation \cite{PhysRevB.23.6851} connecting the scattering formalism of \textsc{Landauer-B\"uttiker} with the \gfnc s the total transmission probability can be written as:
\begin{align}
T_{pq} = \text{Tr}(\Gamma_p G_{pq} \Gamma_q G^+_{pq}) = \sum_{n,m} \abs{t^{pq}_{ll'}}^2
\label{eqn:transmissionfunction}
\end{align}
This formula is known as the \textsc{Caroli} expression. In the linear response regime it is equivalent to \textsc{Landauer}'s formula\cite{PhysRevB.72.035450}.

\chapter{Implementation}
In experiments with 2DEGs in heterojunctions, only the conductance is available. In simulations, microscopic quantities like the electron and spin densities can additionally be obtained. These quantities allow the explanation of experimetal results from first principles. This chapter describes an implementation of the \gfnc{} formalism used to compute the microscopic quantities.
Mesoscopic devices range in geometry from simple wires to complex spin filter cascades, \textsc{Hall} bars or quantum point contacts and any combination of the above. The combination of the presented algorithms holds for all geometries.
\section{Natural Ordering}\label{sec:naruralordering}
The now finite dimension of the matrices obtained by inclusion of the self energy term is determined by the number of lattice points which make up the discretized system.
If a reasonable resolution i.e lattice spacing in the nanometer range is desired the number of lattice-points easily exceeds magnitudes of $10^5$ for a mesoscopic device in the range of micrometers which would lead. 
The number of lattice point neccessary for inclusion can be reduced if only the point are taken that constitute the device and discard the sourrounding environment effectively mapping the full rectangular lattice to a \emph{sparse grid}. Especially for devices of varying transverse extent this often reduces the size of the matrix by a factor of two or more.\par
The full inversion of matrices of this size has still high demands on memory and computational power due to its $\mathcal{O}(\text{Number of lattice-points})$ complexity and is therefore, even with specialised direct algorithms such as LU decomposition based inversion, a very time-consuming affair\cite{Datta2000.2.53} and \cite{Li2009Thesis}.\par
Therefore faster and more efficient techniques for the computation of the \gfnc{} are desireable.\par
For structures of serial nature i.e relatively narrow structures with colinear leads like a quantum wire the \hamil{} turns out to be in a block tri-diagonal form a \emph{natural ordering}. This natural ordering leads to so called \emph{layered devices} seen in fig. (\ref{fig:layered}) in which the device is cut into slices perpendicular to the longest dimension as shown in chapter \ref{sec:discretematrixrep}. There exist several techniques for example based on \textsc{Takahashi}'s observation \cite{Takahashi1973} or the \textsc{Dyson} equation that deliver much higher performance.
\begin{figure}[h!]
\centering
\includegraphics[trim=0cm 38cm 0cm 0cm,clip=true,width=0.5\textwidth]{images/layeredstructure}
\caption{Layered structure (Sample picture)}
\label{fig:layered}
\end{figure}

\section{Recursive \cgfnc{} Algorithm (RGA)} \label{sec:recursivegreenfncalgorithm}
\label{sec:rgapresentation}
The \emph{Recursive-\gfnc-Algorithm} (RGA) \cite{MacKinnon1985} is an optimized algorithm to compute the retarded \gfnc{}. It has been extended by the \textsc{Keldysh} equation to compute the lesser \gfnc{} $\mat{G}^<$ \cite{JApplPhys.91.2343}.
The RGA is based on the possibility that the \hamil{} can be written in a block tridiagonal form as described in \cref{sec:naruralordering}.
This form can be achieved for any system in the tight-binding approximation, each block representing a distinct section of the device \cite{Wimmer2009Thesis}.\par
% \todo{Complexity and implementation in Appendix \ref{app:RGA}, see Wimmer page 189}
Due to the fact that the electron and spin density only depend on certain elements of the \gfnc{} matrix the RGA computes those elements with only a little overhead. While still superfluous elements are computed the complexity is greatly reduced. The algorithm can be derived by the \textsc{Dyson} equation
\begin{align}
	\mat{G} = \mat{G}_0+\mat{G}_0\mat{U}\mat{G}\ .
	\label{eqn:DysonEquationG}
\end{align}
Here $\mat{G}_0$ denotes the \gfnc{} of the isolated block and $\mat{G}$ includes the remaining blocks and their interactions. The recursive calculation of connected blocks of the \gfnc{} matrix is obtained by the pertubation term $\mat{G}_0\mat{U}\mat{G}$. A more detailed derivation is outlined in Appendix \ref{app:RGA}.\par
For conduction, spin and electron densities, the retarded and lesser \gfnc s have to be computed. Because the computation of the lesser \gfnc{} is based on a prior computation of the retarded \gfnc{} the results of the former can be saved to obtain both \gfnc s.\par
In order to simplify the notation, the subscript $C$ will be dropped and the inverse of \cref{eqn:finitegreensfunctionwithselfenergy} can be written as
\begin{align}
(E\mathds{1}-\mat{H})\mat{G}_C = \mat{A}\mat{G}=\mathds{1}\ .
\label{eqn:amatrix}
\end{align}
with the \hamil{} $\mat{H}=\mat{H}_C +\mat{\Sigma}_{Ll}+\mat{\Sigma}_{Lr}+\mat{\Sigma}_{\text{scat}} $.
The central object in the recursive scheme to compute the retarded \gfnc{} is the left-connected \gfnc{} $\mat{G}^{r,i-1}$ containing all the blocks with index up to $i-1$ \cite{JApplPhys.91.2343}. Using the \textsc{Dyson} equation a link to $\mat{G}^{r,i}$ is established allowing the recursive evaluation of all blocks.\par
Let $\mat{G}=\mat{G}^{r,i-1}$ be the \gfnc{} for the connected blocks up to index $i-1$ (\cref{fig:rga1}) then $\mat{G}_0$ corresponds to the \gfnc{} for the blocks up to index $i-1$ plus the \emph{isolated} block $i$ (\cref{fig:rga2}). This system is described by the matrix
\begin{align}
\mat{A}_i = 
\begin{pmatrix}
\ddots 	& \vdots 	   & 		   & \\
\dots   & \mat{A}_{i-2,i-2}&\mat{A}_{i-2,i-1}& \\
     	& \mat{A}_{i-1,i-2}  &\mat{A}_{i-1,i-1}  & \mat{0} \\ 
	&  		   & \mat{0} 	   & \mat{A}_{i,i}
\end{pmatrix}\ .
\end{align}
Diagonal blocks $\mat{A}_{i,i} = E\mathds{1}-\mat{H}_{i,i}$ describe each isolated section and the off-diagonal blocks the interactions between neighboring sections. The isolated block is attached to the rest of the matrix via the \emph{hopping matrices} (\cref{fig:rga3})
\begin{align}
\mat{U} = 
\begin{pmatrix}
\mat{0}& -\mat{A}_{i-1,i}\\
-\mat{A}_{i,i-1}&\mat{0}
\end{pmatrix}\ .
\label{eqn:umatrix}
\end{align}
When inserting \cref{eqn:umatrix} into \cref{eqn:DysonEquationG} the recursive relation between block $i$ and $i+1$ reads 
\begin{align}
\mat{G}^{r,i}_{i,i} := \left[\mat{A}_{i,i}-\mat{A}_{i,i-1}\mat{G}^{r,i-1}_{i-1,i-1}\mat{A}_{i-1,i}\right]^{-1}\ .
\label{eqn:rga}
\end{align}
With analogous calculations the relations between diagonal and off-diagonal parts  of the \gfnc{} 
\begin{align}
\mat{G}^r_{i,i-1} &= -\mat{G}^r_{i,i}\mat{A}_{i,i-1}\mat{G}^{r,i-1}_{i-1,i-1}\\
\mat{G}^r_{i-1,i} &= -\mat{G}^{r,i-1}_{i-1,i-1}\mat{A}_{i-1,i} \mat{G}^{r}_{i,i}\\
\mat{G}^r_{i-1,i-1} &= \mat{G}^{r,i}_{i-1,i-1}-\mat{G}^{r,i-1}_{i-1,i-1}\mat{A}_{i-1,i}\mat{G}^{r}_{i,i-1}\ .
\end{align}
can be found enabling the computation of all matrix blocks.\par
It is important to note that not all matrix elements can be calculated during one pass of the algorithm. After calculation of the first left-connected \gfnc{} at the starting lead $i=0$ \cref{eqn:rga} is used successively until the other lead at index $i=N$ is reached. The $N$th element of the left connected \gfnc{} is already identical to the $N$th element of the full retarded \gfnc{}. The influence of the final lead needs to be propagated to all prior slices. This is done in a \emph{backward pass} during which all desired blocks can be computed, see \cref{alg:retardedrga}.\par
\begin{figure}
\subfloat[]{\label{fig:rga1}\begin{tikzpicture}\node at (0,0)[above] {\includegraphics[width=0.28\textwidth]{images/rga1.pdf}};
\node at (0,4) {$G^{r,i-1}$};
\node at (0,-0.3) {$i-1$};
\node at (-1.8,-0.3) {$i-3$};
\end{tikzpicture}}\quad
\subfloat[]{\label{fig:rga2}\begin{tikzpicture}\node at (0,0)[above]{\includegraphics[width=0.28\textwidth]{images/rga2.pdf}};
\node at (0,4) {$G_0$};
\node at (0,-0.3) {$i-1$};
\node at (1.8,-0.3) {$i$};
\node at (-1.8,-0.3) {$i-3$};
\node[font=\Large] at (1,1.8) {+};
\end{tikzpicture}}\quad
\subfloat[]{\label{fig:rga3}\begin{tikzpicture}\node at (0,0)[above]{\includegraphics[width=0.28\textwidth]{images/rga3.pdf}};
\node at (0,4) {$G^{r,i}$};
\node at (0,-0.3) {$i-1$};
\node at (1,-0.3) {$i$};
\node at (-1.8,-0.3) {$i-3$};
\end{tikzpicture}}
\caption{Schematic of the recursive \gfnc{} algorithm. (a) Slices connected up to index i-1. (b) The \gfnc{} $\mat{G}_0$ of the blocks up to $i-1$ plus the isolated \gfnc{} at index $i$. (c) The isolated block is connected and the \gfnc{} $\mat{G}^{r,i}$ is obtained.}
\label{fig:rgaschematic}
\end{figure}
\vskip 1em
\noindent\begin{minipage}{\textwidth}
\begin{algo}\label{alg:retardedrga}
\textit{Recursive Algorithm for $\mat{G}^r$}\\
\begin{tabularx}{\textwidth}{l X l}
\addlinespace \cmidrule(r{1.2cm}){1-1}\addlinespace
  $\mat{G}^{r,0}_{0,0} := \mat{A}_{0,0}^{-1}$&& $\blacktriangleright$ initialize first element\\\addlinespace[12pt]
  \textbf{for} $i = 1:N-1$ \textbf{do} && \multirow{3}{45mm}{$\blacktriangleright$ recursively compute\\\hspace{10pt} left-connected\\\hspace{10pt} \gfnc{}}\\
  \qquad$\mat{G}^{r,i}_{i,i} := \left[\mat{A}_{i,i}-\mat{A}_{i,i-1}\mat{G}^{r,i-1}_{i-1,i-1}\mat{A}_{i-1,i}\right]^{-1}$&& \\
  \textbf{end for} 			&& \\\addlinespace[12pt]
  $\mat{G}^r_{N,N} := \mat{G}^{r,N}_{N,N}$   	&&$\blacktriangleright N$-th element found\\ \addlinespace[12pt]
  \textbf{for} $i = N-1:1$ \textbf{do}&&$\blacktriangleright$ backward pass\\
  \qquad$\mat{G}^r_{i,i-1} := -\mat{G}^r_{i,i}\mat{A}_{i,i-1}\mat{G}^{r,i-1}_{i-1,i-1}$  \tikz \node[coordinate,yshift=1em,xshift=11.3em] (n1) {}; && \multirow{2}{45mm}{\hspace{10pt} off-diagonal elements}\\ \addlinespace
  \qquad$\mat{G}^r_{i-1,i} := -\mat{G}^{r,i-1}_{i-1,i-1}\mat{A}_{i-1,i} \mat{G}^{r}_{i,i}$ \tikz \node[coordinate,xshift=11.3em] (n2) {}; && \\ \addlinespace
  \qquad$\mat{G}^r_{i-1,i-1} := \mat{G}^{r,i}_{i-1,i-1}-\mat{G}^{r,i-1}_{i-1,i-1}\mat{A}_{i-1,i}\mat{G}^{r}_{i,i-1}$   & &$\blacktriangleright$ diagonal elements \\ 
  \textbf{end for}&& \\\addlinespace \bottomrule 
\end{tabularx}
\end{algo}
\begin{tikzpicture}[overlay]
      \path (n2) -| node[coordinate] (n3) {} (n1);
      \draw[line width=1.5pt,decorate,decoration={brace,amplitude=5pt}]
            (n1) -- (n3) node[midway, right=4pt] {};
\end{tikzpicture}
\end{minipage}
The recursive computation of the lesser \gfnc{} follows the same concept starting from the \textsc{Keldysh} equation (\ref{eqn:keldyshequation}) for non-interacting systems rewritten as
\begin{align}
\mat{G}^<_{i,j} = \mat{G}^r_{i,0}\mat{\Sigma}^<_L(\mat{G}^r_{j,0})^{\dagger}+\mat{G}^r_{i,N+1}\mat{\Sigma}^<_R(\mat{G}^r_{j,N+1})^{\dagger}\ .
\end{align}
Here $\mat{\Sigma}^<_L$ and $\mat{\Sigma}^<_R$ denote the lesser self-energy term for the left and right lead. For the computation of the lesser \gfnc{} it is sufficient to compute diagonal and one off-diagonal of the retarded \gfnc{} $\mat{G}^r$. The algorithm to compute the lesser \gfnc{} is illustrated in \cref{alg:lesserrga}, with $\mat{\sigma}^<_i=\mat{A}_i\mat{G}^{<,i-1}_{i-1,i-1}\mat{A}^{\dagger}_i$. It requires a forward and a backward pass like the algorithm for the retarded \gfnc{}.\par
\vskip 1em
\noindent\begin{minipage}{\textwidth}
\begin{algo}\label{alg:lesserrga} 
\textit{Recursive Algorithm for $\mat{G}^<$}\\
\begin{tabularx}{\textwidth}{l l}
\addlinespace\cmidrule(r{3.5cm}){1-1}\addlinespace
 $\mat{G}^{<,0}_{0,0} := \mat{G}^{r,0}_{0,0}\mat{\Sigma}^<_{0,0} \mat{G}^{a,0}_{0,0}$& $\blacktriangleright$ initialize first element\\\addlinespace[12pt]
 \textbf{for} $i = 1:N-1$ \textbf{do} & \multirow{3}{45mm}{$\blacktriangleright$ recursively compute\\\hspace{10pt} left-connected\\\hspace{10pt} \textsc{Green}'s function}\\
 \qquad$\mat{G}^{<,i}_{i,i} := \mat{G}^{r,i}_{i,i}( \mat{\Sigma}^<_{i,i}+\mat{\sigma}^<_{i})\mat{G}^{a,i}_{i,i}$& \\
 \textbf{end for} 				& \\\addlinespace[12pt]
 $\mat{G}^<_{N,N} := \mat{G}^{<,N}_{N,N}$   	&$\blacktriangleright N$-th element found \\ \addlinespace[12pt]
 \textbf{for} $i = N-1:1$ \textbf{do} 	  	&$\blacktriangleright$ backward pass\\
 \qquad$\mat{G}^<_{i,i-1} := \mat{G}^r_{i,i}\mat{A}_{i,i-1}\mat{G}^{<,i-1}_{i-1,i-1} +\mat{G}^<_{i,i}\mat{A}^{\dagger}_{i,i-1}\mat{G}^{a,i-1}_{i-1,i-1}$ &$\blacktriangleright$ off-diagonal elements\\ \addlinespace
 \qquad$\mat{G}^<_{i-1,i-1} := \mat{G}^{<,i-1}_{i-1,i-1} +\mat{G}^{r,i-1}_{i-1,i-1}(\mat{A}_{i-1,i}\mat{G}^{<}_{i,i}\mat{A}^{\dagger}_{i,i-1})\mat{G}^{a,i-1}_{i-1,i-1}$ &$\blacktriangleright$ diagonal elements \\ \addlinespace
 \hspace{4.5cm}$ + \mat{G}^{<,i-1}_{i-1,i-1}\mat{A}^{\dagger}_{i-1,i}\mat{G}^a_{i,i-1}$&\\\addlinespace
 \hspace{4.5cm}$+\mat{G}^r_{i-1,i}\mat{A}_{i,i-1}\mat{G}^{<,i-1}_{i-1,i-1}$&\\ 
 \textbf{end for}& \\\addlinespace \bottomrule 
\end{tabularx}
\end{algo}
\end{minipage}
\subsection{Computational Complexity of RGA}\label{sec:rgacomplexity}
The most demanding operations of the algorithms are the matrix multiplication and inversion. Both operations scale $\mathcal{O}(N^3)$ with the size of the involved $N \times N$ matrices. As the algorithms carry out a constant number of operations per block the overall RGA scales as $\mathcal{O}(N^3M)$ with $M$ being the number of blocks\,\cite{Li2009Thesis}.\par
For an elongated device that is much longer in the $x$ than in the $y$ direction, the number of lattice points in that direction is larger than the other ($N_y \ll N_x$). On a simple square lattice this amounts to a running time order of $\mathcal{O}(N_y^3N_x)$  instead of the direct approach $\mathcal{O}(N_y^3N_x^3=\text{Number of lattice-points})$.\par
The reduced complexity is based on the selective computation of matrix blocks if all entries of $\mat{G}$ were required the computational cost would also be $\mathcal{O}(N_y^3N_x^3)$.
\FloatBarrier

\section{Complex Geometries}
The application of the RGA requires a block tridiagonal \hamil{}. Layered systems naturally lead to a block tridiagonal \hamil{} but due to the serial construction of the RGA the choice of systems is limited to this geometry. For the computation of each \gfnc{} it traverses the blocks of the $A$-Matrix first from start to end and then vice versa. It can not take any detours or traverse each element more than once per path. Because of this restriction, given a device similar to the one shown in \cref{fig:noncolinear} one has to assign all remaining nodes to a large last layer once the angled output has been reached. This results in a very large last block. As can be seen in \cref{sec:rgacomplexity} large blocks in the \hamil{} are prohibitively expensive and would effectively render the algorithm useless.
\begin{figure}[!ht]
\centering
\begin{tikzpicture}
\node at (0,0)[above] {\includegraphics[scale=0.65]{images/noncolineardevice}};
\draw[style={latex-latex,line width=2pt}] (-6,1) node[right]{$y$} -- (-6,0) node [below left]{$z$} -- (-5,0) node[below]{$x$};
\draw (-6,0) circle (4pt) [fill=black];
\node at (2,3.5){$a$};
\end{tikzpicture}
\caption{Non-colinear device. Dots denote lattice points and slices, if treated as a layered structure, are indicated by alternating white and gray background.}
\label{fig:noncolinear}
\end{figure}
In essence one needs a \hamil{} with a particular block tri-diagonal structure that has a small bandwidth i.e. small blocks. This can in principle be achieved by a process called matrix reordering. Because the diagonal elements represent the on-site energy the reordering should only exchange one diagonal element with another. In order to do so, the same permutation is applied to the rows and columns. These \emph{symmetric permutations} can be realized by a permutation martix $P$ as follows \cite{saad2003iterative}:
\begin{align}
\mat{H}' = \mat{P} \mat{H} \mat{P}^{T}\ .
\label{eqn:permutation}
\end{align}
In order to find such a permutation matrix enabling the use of the RGA one needs an algorithm adhering to the requirements of the RGA.
A potent formulation of matrix reordering can be achieved in the form of graph theory. The tight-binding \hamil{} and therefore the matrix $A$ to invert is sparse as well as Hermitian. Thus one can use a natural one-to-one correspondence between sparse matrices and graphs.
\subsection{Graph Representation}
A graph is a way to represent binary relations between objects of a set. A common definition of a graph is given by
\begin{dfn}
A \emph{graph} \textpzc{G} consists of an ordered pair $\mathpzc{G} :=(\mathpzc{N},\mathpzc{E})$, where \textpzc{N} is a non-empty set of \emph{nodes} \textpzc{n} and \textpzc{E} a set of ordered pairs of nodes $(\mathpzc{n}_1,\mathpzc{n}_2) \in \mathpzc{N} \times \mathpzc{N}$ called \emph{edges}.
A graph is called \emph{directed} if the edges observe a direction of incidence  and it is called \emph{undirected} if for any two $(\mathpzc{n}_1,\mathpzc{n}_2) \in \mathpzc{E}$ also $(\mathpzc{n}_2,\mathpzc{n}_1) \in \mathpzc{E}$.
Two nodes $\mathpzc{n}_i$ and $\mathpzc{n}_j$ are \emph{adjacent} if $(\mathpzc{n}_i,\mathpzc{n}_j) \in \mathpzc{E}$. \cite{graham1995handbook}
\end{dfn}
The \emph{adjacency graph} of a sparse $n \times n$ matrix $\mat{A}$ is a graph $\mathpzc{G}$ whose $n$ nodes represent the $n$ unknowns of the equation system the matrix is based on and the edges represent the dependence on the unknowns. For each non-zero matrix entry $\mat{A}_{i,j}\neq 0$ there exists an edge $(\mathpzc{i},\mathpzc{j}) \in \mathpzc{E}$. The matrix corresponding to an undirected graph is the \emph{adjacency matrix}. Because the \hamil{} is often Hermitian, the non-zero matrix elements are usually symmetric. Hence an undirected adjacency graph represents the matrix \emph{structure}. It is important to note that by default the graph does not hold any information about the value of the matrix elements but only their relation. However it is a simple matter of storing the values of the matrix elements as a \emph{weight} of the edges.\par
A graph may be pictured as a collection of circles as nodes and connecting lines representing the edges. Incidentally for tight-binding models each circle corresponds to a site and the lines depict the hopping options. If the graph is weighted the thickness of a line may represent the hopping \emph{possibility}. An example for a small number of grid points can be seen in \cref{fig:adjacencygraph}.

\begin{figure}[!h]
\centering
\subfloat[]{\label{fig:discretizeddevice}\includegraphics[scale=1]{images/noncolineardevicelevelstructure}}\hspace{2.35em}
\subfloat[]{\label{fig:adjacencygraph}\includegraphics[scale=0.4]{images/7x6l_graph}}\hspace{2.35em}
\subfloat[]{\label{fig:hamilmatrix}
\begin{tikzpicture}
\def\a{0.51}
\def\b{1.28}
\def\c{0.88}
\node at (0,0){\includegraphics[scale=1]{images/7x6lhamil}};
\draw[dashed,thick] (-1.6,1.2) -- ++(\a,0)--++(0,-\a)--++(-\a,0) -- cycle;
\draw[dashed,thick] ($(-1.6+\a,1.2)$) -- ++(\b,0)--++(0,-\a)--++(-\b,0) -- cycle;
\draw[dashed,thick] (-1.099,0.69) -- ++(\b,0)--++(0,-\b)--++(-\b,0) -- cycle;
\draw[dashed,thick] ($(-1.099+\b,0.69)$) -- ++(\c,0)--++(0,-\b)--++(-\c,0) -- cycle;
\draw[dashed,thick] ($(-1.099-\a,0.69)$) -- ++(\a,0)--++(0,-\b)--++(-\a,0) -- cycle;
\draw[dashed,thick] (0.19,-0.59) -- ++(\c,0)--++(0,-\c)--++(-\c,0) -- cycle;
\draw[dashed,thick] ($(0.19+\c,-0.59)$) -- ++(\a,0)--++(0,-\c)--++(-\a,0) -- cycle;
\draw[dashed,thick] ($(0.19-\b,-0.59)$) -- ++(\b,0)--++(0,-\c)--++(-\b,0) -- cycle;
\draw[dashed,thick] (1.08,-1.49) -- ++(\a,0)--++(0,-\a)--++(-\a,0) -- cycle;
\draw[dashed,thick] ($(1.08-\c,-1.49)$) -- ++(\c,0)--++(0,-\a)--++(-\c,0) -- cycle;
\end{tikzpicture}}
\begin{tikzpicture}[overlay]
\draw[-latex,line width=2pt] (-12.5,3)--++(1,0);
\draw[-latex,line width=2pt] (-6.25,3)--++(1,0);
\end{tikzpicture}
\caption{(a) Schematic of a simple discretized non-colinear device. The dots denote lattice points. The levels are indicated by alternating white and gray background. (b) The relations between nodes are stored in the edges of the depicted graph. Nodes are represented as red circles. (c) \hamil{} of the discretized nanowire obtained by matrix reordering. The blocks are highlighted by dashed lines. The arrows indicate the matrix reordering procedure.}
\label{fig:matrixtograph}
\end{figure}
\subsection{Matrix Reordering as a Graph Partitioning Problem}
The process of matrix reordering with symmetric permutations corresponds to renaming of the nodes of a graph without altering the edges. In order to find the permutation matrix $P$ one has to sort the nodes of the graph.\par
There exist numerous algorithms that can reduce the bandwidth of the adjacency matrix like the well known reverse \textsc{Cuthill-McKee} algorithm but the application of the RGA imposes an additional constraint. It requires that the last block of the matrix (the end nodes of the graph) has to be the endpoint of the walk of the algorithm i.e. the sites adjacent to a contact have to be in the upper-left and lower-right corners of the matrix. That means the nodes representing these sites may not be renamed. A \emph{partition} of the graph yields a so-called \emph{level structure} \cite{gibbs.Siam.13.236}.
\begin{dfn}
A level structure $\mathpzc{L}(\mathpzc{G}) = (\mathpzc{l}_0,\mathpzc{l}_1,\mathpzc{l}_2,\dotsc,\mathpzc{l}_N)$, of a graph \textpzc{G} is a partition of the set of nodes \textpzc{V} into $N$ \emph{levels} $\mathpzc{l}_i$ such that
\begin{enumerate}
\item all nodes adjacent to nodes in level $\mathpzc{l}_1$ are in either level $\mathpzc{l}_1$ or $\mathpzc{l}_2$
\item all nodes adjacent to nodes in level $\mathpzc{l}_N$ are in either level $\mathpzc{l}_{N}$ or $\mathpzc{l}_{N-1}$
\item for $1 < i < N$, all nodes adjacent to nodes in level $\mathpzc{l}_{i}$ are either in level $\mathpzc{l}_{i-1}$, $\mathpzc{l}_{i}$ or $\mathpzc{l}_{i+1}$
\end{enumerate}
\end{dfn}
For a simple colinear geometry a level is usually identical to a slice. For more complex geometries a level and a slice may differ. A possible choice to fulfill these requirements is the \textsc{Gibbs-Poole-Stockmeyer} (GPS) algorithm \cite{gibbs.Siam.13.236}. Two conditions are of crucial importance for the reordering in order to achieve high RGA performance. As the complexity of the RGA is of order $\mathcal{O}(N^3)$(\cref{sec:rgacomplexity}) one has to avoid large blocks. In addition one seeks a balanced level structure that is all the levels ideally have the same number of nodes.
An algorithm tailored to the demands of high RGA performance has been introduced by \textsc{Wimmer}. As it outperforms the GPS algorithm and offers greater versatility \cite{Wimmer2009JComPhys} it is applied to the adjacency graph of the \hamil{} prior to the execution of the RGA.
\subsection{RGA Tailored Block-Tridiagonalization Algorithm}
The process of graph partitioning is known to be NP-Complete \cite{GareyTCS.1.237}. Hence one cannot compute the optimal level structure from first principles but has to employ heuristics. The algorithm used by \textsc{Wimmer} incorporates a global and a local approach. To find the maximum number of levels for a given geometry a \emph{Breadth-First-Search} (BFS) as described in \cref{alg:bfs} is performed as it yields a level structure by construction as well as the maximum number of levels $N_0$.
\begin{algo} \label{alg:bfs}
\textit{Breadth-First-Search Algorithm}\\
\begin{tabularx}{\textwidth}{l}
  \addlinespace \cmidrule(r{7.6cm}){1-1}
\begin{minipage}{\textwidth}
    \vskip 4pt
    \begin{enumerate}[1]
   \item \textbf{Start} with $i=1$ then $\mathpzc{l}_i = \mathpzc{l}_1$ because start and end nodes are fixed
   \item If any node $\mathpzc{n} \in \mathpzc{l}_i$ is adjacent to a $\mathpzc{n} \in \mathpzc{l}_N$ distribute all nodes not in $\mathpzc{l}_i$ to $\mathpzc{l}_N$ and \textbf{end}
   \item Assign all nodes not in levels $\mathpzc{l}_0 - \mathpzc{l}_i$ to level $\mathpzc{l}_{i+1}$
   \item \textbf{Continue} at 2 with $i=i+1$
   \end{enumerate}
   \vskip 4pt
 \end{minipage}
\\
 \bottomrule 
\end{tabularx}
\end{algo}
With the maximum number of levels $N_i$ in $\mathpzc{l}_i$ known the graph is cut into sections $\mathpzc{l}_{i_x}$ with approximately equal number of levels to ensure a balanced level structure i.e.
\begin{align}\label{eqn:balancecriterion}
\abs{\mathpzc{l}_{i_1}} \approx \frac{N_{i_1}}{N_i}\abs{\mathpzc{l}_{i_1}} \qquad\text{and}\qquad \abs{\mathpzc{l}_{i_2}} \approx \frac{N_{i_2}}{N_i}\abs{\mathpzc{l}_{i_1}}\ .
\end{align}
With $\abs{\mathpzc{l}_i}$ denoting the number of nodes in that level.  All the nodes in each of those sections are distributed into levels by a BFS and optional further optimization starting from both ends where the graph has been cut. This section is in turn cut into ideally equal number of levels and the process is recursively repeated until only one level remains. The ordered collection of these final levels of all recursive paths constitutes the sought level structure.\par
If the nodes have been named from 1 to $k$ with $k$ being the size of the square \hamil{} the matrix reordering is achieved by simply rearranging the rows and columns symmetrically according to the sequence of the nodes in the level structure. The ordering of the nodes within one level is of minor importance. 
A modification of this algorithm omitting detailed optimizations as used in this work is presented in algorithm \ref{alg:wimmeralgorithm}.\par
% \noindent\begin{minipage}{\textwidth}
\begin{algo}\label{alg:wimmeralgorithm} 
\textit{Block-Tridiagonalization Algorithm}\\
\begin{tabularx}{\textwidth}{l}
 \addlinespace\cmidrule(r{7cm}){1-1}
\begin{minipage}{\textwidth}
    \vskip 4pt
    \begin{enumerate}[1]
 \item Generate graph \textpzc{G} corresponding to the \hamil{} and the starting level $\mathpzc{l}_0$ and ending  level $\mathpzc{l}_N$.
   \item Use BFS to determine maximum number of levels $N$.
   \item Bisect remainder of graph: $\mathpzc{G}_1 = \mathpzc{G} \setminus (\mathpzc{l}_0 \cup \mathpzc{l}_N)$ containing $N_i = N-2$ levels. With left-adjacent level $\mathpzc{l}_l = \mathpzc{l}_0$ and right-adjacent level $\mathpzc{l}_r = \mathpzc{l}_N$.\label{alg:wimmeralgorithm3}
   \item[] \begin{enumerate}[a]
   \item  \textbf{Stop}, if $N_i$ = 1.\label{alg:wimmeralgorithm3a}
   \item Do a BFS starting from level $\mathpzc{l}_l$ up to level $N_{i_1} = \text{Floor}(N_i/2)$ and a BFS starting from level $\mathpzc{l}_r$ up to level $N_{i_2}= N_i - \text{Floor}(N_i/2)$. The nodes found are assigned to $\mathpzc{l}_{i_1}$ and $\mathpzc{l}_{i_2}$ respectively and marked as visited.
   \item Distribute remaining nodes in level. Continue BFSs from previous step from both sides and assign nodes according to the balance criterion (\ref{eqn:balancecriterion}).
   \item Recursively apply step (\ref{alg:wimmeralgorithm3}) for levels $\mathpzc{l}_{i_1}$ and $\mathpzc{l}_{i_2}$ until in (\ref{alg:wimmeralgorithm3a}) final level is reached.
   \end{enumerate}
   \item Collect all levels from (\ref{alg:wimmeralgorithm3}) into level structure \textpzc{L} and reorder matrix rows and columns according to the sequence of nodes found  in \textpzc{L}
   \end{enumerate}
   \vskip 4pt
 \end{minipage}
\\
 \bottomrule 
\end{tabularx}
\end{algo}
% \end{minipage}\vskip 1em\par
The steps involved in the matrix reordering are sketched in \cref{fig:discretizeddevice,fig:adjacencygraph,fig:hamilmatrix}.
By reordering the matrix one can use the unaltered RGA even for more complex geometries but also gains the ability to this conventional two-terminal algorithm for multiple terminals.
\FloatBarrier

\section{Validation}\label{sec:validation}
Due to the lack of proper standard problems regarding the validity of a quantum mechanical simulator for transport properties the outcome of the numerical calculations via the \gfnc{} method will be compared to a selection of simple problems and experimental results.\par
\subsubsection{Electron Density}
For the case of a quantum wire the probability disrtibution can be calculated analytically. The quantum wire is simply a one-dimensional potential well in $y$ direction which ideally extends infinitely in the $x$ direction.
Because of this the wavefunction can be separated into transverse and longitudinal parts i.e
\begin{align}
\Psi(x,y) = \sqrt{2/L}\text{ sin}(n\pi/L \cdot y) \cdot e^{ik_xx}\,.
\end{align}
Because of the transverse confinement the eigenenergies become discrete similar to the case of the $z$ confinement illustrated in \cref{fig:potentialwell}.
\todo[noline]{average electro density for 100nm wire is about $10^17$ so that should fit}
The analytically calculated probability distribution is renormalized to experimentally measured electron densities $n \sim 10^{17}/m^2$ typical for a 2DEG embedded in heterojunctions containing InAs\,\cite{gelfand2006}\,\cite{JJAP.26.L59}
In figures (\dots ) the analytical calculation for the first three modes is superimposed with the simulated results for a nano-wire of 100nm. 
The parameters for the effective mass of an electron in a InAs 2DEG are taken to be $0.026\times m_0$ with $m_0$ being the electron rest mass\,\cite{PhysRev.105.460}.\par
\begin{figure}[h!]
  \begin{center}
    % \showthe\columnwidth % Use this to determine the width of the figure.
\subfloat[Analytical wavefunction and probability distribution]{\label{fig:analytical1}\includegraphics[width=210pt]{images/analytical1}} \qquad
    \subfloat[Simulated and re-normalized analytical transverse electron density cut]{\label{fig:overlay1}\includegraphics[width=210pt]{images/overlay1}}\\
    \subfloat[Relative error $\frac{n1_{ana}-n1_{sim}}{n1_{sim}}$]{\label{fig:relerror1}\includegraphics[width=210pt]{images/error1}}\qquad
    \subfloat[Simulated two dimensional electron density]{\label{fig:dens1}\includegraphics[width=210pt]{images/dens1}}
    \caption{Comparison of analytical and simulated electron densities for the first mode. Analytical probability amplitude \ref{fig:analytical1}, cut of electron density and analytical calculation overlay perpendicular to the length of the wire \ref{fig:overlay1}, relative error \ref{fig:relerror1} and two-dimensional electron density \ref{fig:dens1}}\label{fig:mode1}
  \end{center}
\end{figure}
The electrons populate only the first mode at the calculated energy using the dispersion relation of electrons on a lattice in transverse direction
\begin{align}
E_{n} = 2t_0(1-\text{cos}(n\pi a/L)).
\label{eqn:dispersionlattice}
\end{align}
Here $L=100$nm is the transverse extent of the nano-wire and $n=1$ the number of the mode. Figure~\ref{fig:analytical1} shows the normalized wavefunction and probability density for the first mode.\par
In \cref{fig:overlay1} of the tableau of \cref{fig:mode1} the overlay of re-normalized analytical wavefunction and simulated electron density shows the excellent agreement of theory and simulation. This becomes especially clear noting the relative error in \cref{fig:relerror1} of magnitude $10^{-11}$ normalized to \emph{mean electron density}.\par
\Cref{fig:dens1} displays a top view of the two-dimensional electron density. Red denoting high and blue low density. The electron density is uniform in $x$-direction as expected.\par
\begin{figure}[h]
  \begin{center}
    % \showthe\columnwidth % Use this to determine the width of the figure.
\subfloat[Analytical wavefunction and probability distribution]{\label{fig:analytical2}\includegraphics[width=210pt]{images/analytical2}} \qquad
    \subfloat[Simulated and re-normalized analytical transverse electron density cut]{\label{fig:overlay2}\includegraphics[width=210pt]{images/overlay2}}\\
    \subfloat[Relative error $\frac{n1_{ana}-n1_{sim}}{n1_{sim}}$]{\label{fig:relerror2}\includegraphics[width=210pt]{images/error2}}\qquad
    \subfloat[Simulated two dimensional electron density]{\label{fig:dens2}\includegraphics[width=210pt]{images/dens2}}
    \caption{Comparison of analytical and simulated electron densities for the first mode. Analytical probability amplitude \ref{fig:analytical2}, cut of electron density and analytical calculation overlay perpendicular to the length of the wire \ref{fig:overlay2}, relative error \ref{fig:relerror2} and two-dimensional electron density \ref{fig:dens2}}\label{fig:mode2}
  \end{center}
\end{figure}
The second mode is also appropriately populated for $n=2$. The analytical calculation does not consider the existence of lower modes at a certain energy, in contrast the model employed in the simulations does. The electron densities for the first and second modes have therefore to be superposed what is shown in \cref{fig:overlay2}. The total energy independent electron density is obtained by integrating over energy space as is shown in \cref{eqn:gfncdensity}.For transport properties it suffices to focus on energies $E=E_F\pm k_bT$ because the conductance is a \textsc{Fermi} surface property c.f Sec.~\ref{sec:landauerbuettiker}. $E_F$ denotes the \textsc{Fermi} energy and $k_bT$ the thermal energy contribution $k_b$ being the \textsc{Boltzmann} constant and $T$ the temperature.\par
The relative error increases considerably but stays within acceptable bounds considering the rather simple analytical model.
Also the uniform distribution along the wire remains and the two maxima in electron density can clearly be seen in \cref{fig:dens2}.\par
\begin{figure}[h]
  \begin{center}
\subfloat[Analytical wavefunction and probability distribution]{\label{fig:analytical3}\includegraphics[width=210pt]{images/analytical3}} \qquad
    \subfloat[Simulated and re-normalized analytical transverse electron density cut]{\label{fig:overlay3}\includegraphics[width=210pt]{images/overlay3}}\\
    \subfloat[Relative error $\frac{n1_{ana}-n1_{sim}}{n1_{sim}}$]{\label{fig:relerror3}\includegraphics[width=210pt]{images/error3}}\qquad
    \subfloat[Simulated two dimensional electron density]{\label{fig:dens3}\includegraphics[width=210pt]{images/dens3}}
    \caption{Comparison of analytical and simulated electron densities for the first mode. Analytical probability amplitude \ref{fig:analytical3}, cut of electron density and analytical calculation overlay perpendicular to the length of the wire \ref{fig:overlay3}, relative error \ref{fig:relerror3} and two-dimensional electron density \ref{fig:dens3}}\label{fig:mode3}
  \end{center}
\end{figure}
The results of simulation and simple analytical model begin to derivate stronger for the case of the third mode. Only the previous lower mode is considered in the renormalization of the analytical result in the overlay of transversal electron densities but the simulated results are for still within acceptable bounds for as \cref{fig:relerror3} shows. Derivations in the predictions of electron density for the compared models continue to exist as is expected because the analytical model does not incorporate the many-particle nature of the 2DEG like the \gfnc{} theory does.
The electron density calculation show throughout satisfactory and acceptable results.\par
\FloatBarrier
\subsubsection{Spin densities}
Validating the results of the spin density calculations is a rather intricate process because of the number of parameters and physical influences involved.\par
The spin density of flowing electrons in a conductor with spin-orbit coupling exhibits a characteristic spin split in the local magnetic field due to the \textsc{Rashba} interaction.
The spin-split has been measured through optical means by \textsc{Kato} et. al.\,\cite{Kato2004Science} and is shown in \cref{fig:spinplitkato}. It shows the split of electron spin flowing along the wire. There is an accumulation of upward spin in the upper half and of downward spin in the lower half.\par
\begin{figure}[h]
  \begin{center}
    \subfloat[Experimentally measured spin density $s_z$\,\cite{Kato2004Science} in arbitrary units]{\label{fig:spinplitkato}\includegraphics[width=\textwidth]{images/spinsplitkato}}\qquad
    \subfloat[Simulated spin density $s_z$ in arbitrary units]{\label{fig:spinsplitme}\includegraphics[width=\textwidth]{images/spinsplitme}}\qquad
    \caption{Spin-split. Comparison of experimental and simulated spin densities in a nanowire.}
  \end{center}
\end{figure}
\todo[noline]{Spin oscillation??}The spin split is also shown in the results of the simulations in the \gfnc{} model. Due to the approximative and simplifying nature of the model considered only an idealized system is described therefore lacking common physical features like temperature induced noise although they can be implemented in an efficient fashion.\par
\FloatBarrier
\subsubsection{Transmission}
Validating the results from the transmission calculations can be done in a very straightforward manner. For quantum point contacts of varying constriction or if the energy parameter is altered the conductance will vary only in discrete steps.\par
This is a direct quantum mechanical consequence as from a classical point of view a linear relation between conductance and device width is expected.
Is has been found out that the conductance is quantized in steps of the conductance quantum $e^2/h$ c.f. \cref{eqn:conductance}.
It is due to the finite transverse size of the modes within the conductor that only if space permits another mode is established. The transmission probability function \ref{eqn:transmissionfunction} will reach a plateau when each new mode is established and ideally performs a unit step per mode when the next higher or lower state begins to exist in the conductor. The shape and height of the steps is subject to a multitude of influences but because of its discrete step height under certain conditions a valuable tool for comparison.\par
\begin{figure}[h]
  \begin{center}
    \subfloat[Experimentally measured conductance for varying gate voltage, effectivly "opening" the quantum point contact\,\cite{PhysRevLett.60.848}] {\includegraphics[]{images/qpcwees}}
    \subfloat[Simulated conductance from opening a QPC constriction geometrically]{\includegraphics[]{images/qpcme}}
    \caption{Comparison of experimental and simulated conductances of an QPC constriction of variable size. Plotted is the conductance in terms of the conductance quantum $e^2/h$ against two different but proportional measures of width of the QPC.}
\end{center}
\end{figure}
The comparison of experimentally measured and simulated conductance quantization show a very good agreement in step height as new propagating modes are established.\par
The simulated results show in summary acceptable results to merit its application in the theoretical analysis of quantum devices embedded in a 2DEG.
The roughness and shape of the steps is subject to investigation of the following chapter.\par
\FloatBarrier

\section{Computational Framework}
The algorithms and logic neccessary for the simulation are mainly implemented in an object oriented programming language called \emph{python}.
As an interpreted scripting language python suffers in comparison to low level languages like FORTRAN or C and its derivatives a performance penalty. However the ease of use, powerful numerical modules and large scientific userbase make it a suitable choice if one writes a simulator from ground up.\par
Especially because of pythons object-orientation and the existance of a multitude of interfaces to lower level programming languages and libraries computing intensive parts of the program are easily modulized.\par
It exists a large suite of numerical modules wrapping fast numerical algebra routines like BLAS, ATLAS or MKL called \emph{numpy} \cite{numpy} which is heavily  used for the simulations in this work for high performance computations.\par
Higher level mathmatical functions and operations like the \textsc{Schur} decomposition are part \emph{scipy} \cite{scipy} which is a superpackage of numpy.\par
The RGA is implemented in pure python on top of the numeric algebra package with the addition of custom routines.\par
For the handling of graph based calculations and reoderings \emph{networkx} \cite{networkx} is employed. The networkx package supplies a graph class upon which the custom BFS and graph partitioning algorithm used by \textsc{Wimmer} are also written in pure python to ensure interoperability.

\chapter{2DEG Transport Simulations}
An implementation of the \gfnc{} formalism is used to investigate the transport properties of a 2DEG in an InAs heterojunction. A lattice spacing of $a=1$~nm and an effective mass of $m^*=0.026 \cdot m_0$ as in \cref{sec:validation} are assumed. Here, $m_0$ denotes the electron rest mass. The \textsc{Rashba} parameter is set to a value of $\alpha = 20 \cdot 10^{-12}$~eVm as derived from experiments \cite{Jacob2009Thesis}.\par
In the fist part of this chapter, the influence of the geometry of quantum point contacts on the conductance, and the electron and spin densities in a nanowire is examined. In the second part, the electron and spin densities of a 2DEG of varying geometry is presented. The simulated results are used to explain the interference in an \textsc{Aharonov-Bohm} ring qualitatively by an analytical model.
\section{Collinear Quantum Point Contacts}
In chapter \ref{sec:conductancefromtransmission} the ``\emph{relation of conductance and transmission}'' \cite{landauer1996} is shown. In experiments of semiconductor heterojunctions \cite{vanHoutenBeenakker2005} and metals \cite{PhysRevB.36.1284} it was found that the transmission of a device with a narrow constriction is \emph{quantized}. A two-dimensional electron gas in a semiconductor heterojunction has a \textsc{Fermi} wavelength a hundred times larger than electrons in metal. Due to the large \textsc{Fermi} wavelength the electron transport in mesoscopic devices shows quantization. \emph{Quantum point contacts} are mesoscopic constrictions with a width comparable to the \textsc{Fermi} wavelength of $\lambda_F \approx 30\text{~nm}$ allowing the direct control of the quantization \cite{vanHoutenBeenakker2005}.\par
The quantization of the conductance measurements depends on the width of the constriction. Additionally, the \emph{potential landscape}, if smooth or abrupt, influences the conductance \cite{PhysRevB.44.8017}. The potential landscapes of the quantum point contacts used in the simulations are illustrated in \cref{fig:recthardwalled,fig:trihardwalled,fig:sphericalsemihardwalled,fig:pointsoftwalled,fig:variationalwalled}.
For the simulated quantum point contacts the width of the constriction is varied for electrons with a fixed energy of $E=0.06\cdot t_0$, with the hopping parameter $t_0$, see \cref{sec:discretematrixrep}. At this energy 15 modes accur within the nanowire of 200~nm width when no constriction is present. All simulations are performed at an energy between the 15th and the 16th mode of the nanowire. This assures a constant initial conduction of $G =30$~$e^2/h$ and a stable initilal mode configuration.
\subsection{Hard Walled Quantum Point Contacts}
The geometry of quantum devices in 2DEGs, for example, prepared by etching via reactive gases or the cleaved edge technique commonly exhibits steep walls of the confining potential \cite{ApplPhysLett.66.323}. This motivates the investigation of the influences of geometrical constrictions with abruptly varying potential landscapes on the transport properties of the 2DEG.
\Cref{fig:recthardwalled} shows a potential constriction with abrupt walls also addressed as a \emph{hard walled} potential.\par
\begin{figure}[h!]
  \begin{minipage}[c]{0.5\textwidth}
    \begin{tikzpicture}%[every text node part/.style={align=center}]
	\node at (0,0)[above] {\includegraphics[width=0.9\textwidth]{images/qpcrect-crop.png}};
	\axes
  \draw[style={<->,thin}] ($1.125*(-1.3,3.6247)$) -- ($1.125*(-0.55,3.25)$) node[midway,left,yshift=-0.5em] {$d$};
  \draw[style={<->,thin}] ($1.125*(0,2)$) -- ($1.125*(1,2.5)$) node[midway,above] {$w$};
      \end{tikzpicture}
      \end{minipage}
  \begin{minipage}[c]{0.5\textwidth}
   \begin{flalign}\quad\text{Pot}(x,y) =\ &\Theta\left(\frac{d}{2}-\abs{x}\right)&\notag\\
   \cdot\ &\Theta\left(\abs{y}-\frac{w}{2}\right)&\end{flalign}
   \end{minipage}
  \caption{Potential landscape of a hard walled quantum point contact. The longitudinal direction of electron flow $x$ and the transverse direction $y$ is denoted. The depth of the wall $d$ and the width of the confinement $w$ are illustrated by the arrows. The dimensionless analytical expression of the potential is presented on the right. The \textsc{Heaviside} step function is denoted by $\Theta$.}\label{fig:recthardwalled}
\end{figure}
The simulations are performed in the regime of \emph{linear transport}, i.e. the electrons enter and exit the conductor approximately at the same potential~\cite{Nikolic2010}. The lateral electron propagation is distinctly disturbed if the maximum potential energy of the constriction is comparable to the \textsc{Fermi} energy of the 2DEG. The computed electron and spin density is shown in \cref{fig:rectedens} and \cref{fig:rectspindens}, respectively. All figures of densities presented in this chapter represent a view parallel to the $z$ axis on the 2DEG under the influence of lateral confinement.\par
In the linear-transport model of the \gfnc{} formalism at steady state, the electrons are injected at a fixed energy $E$, confer \cref{eqn:finitegreensfunctionwithselfenergy}. The wave vectors of the incoming electrons initially point along the positive $x$ axis, see \cref{fig:rectedens} for example. If no constriction were present 15 modes would form in the nanowire which are translationaly invariant in respect to the $x$ axis. The electron density in \cref{fig:rectedens} indicates that only eight modes are able to form within the quantum point contact with $w=105$~nm as is evident from the eight continuous electron density channels along the $x$ axis. This agrees with the conductance simulation of $G=16$~$e^2/h$ at $w=105$~nm, including a factor of 2 for the spin, see \cref{fig:recttrans}.\par
\begin{figure}[t]
  \subfloat[]{\label{fig:rectedens}\includegraphics[]{images/238916-wire400x200-Nov-28-2011-17.41edens}}
  \hspace{14pt}
    \begin{tikzpicture}[overlay]%[every text node part/.style={align=center}]
	\node[thick,shape=rectangle,draw,inner sep=1pt] at (0,-8) {\includegraphics[width=0.13\textwidth]{images/qpcrect-crop.png}};
    \end{tikzpicture}
  \hspace{4pt}
  \subfloat[]{\label{fig:rectspindens}\includegraphics[]{images/238916-wire400x200-Nov-28-2011-17.41spindens}}
  \caption{Simulated electron and spin density of a rectangular quantum point contact of  width $w$=105~nm, and depth $d$=60~nm. (a) Two-dimensional electron density. Color code denotes the electron density $n$. (b) Two-dimensional spin density. Note the well defined spin split from about 250~nm to 350~nm. Color code denotes the spin density $s_z$. The corresponding potential landscape is depicted in the inset.}
\end{figure}
\begin{figure}[h]
  \centering
  \subfloat[]{\label{fig:recttrans} \includegraphics[]{images/238916-wire400x200-Nov-28-2011-17.41trans}}
  \subfloat[]{\label{fig:rectdetailtrans} \includegraphics[]{images/238916-wire400x200-Nov-28-2011-17.41detailtrans}}
    \begin{tikzpicture}[overlay]%[every text node part/.style={align=center}]
	\node[thick,shape=rectangle,draw,inner sep=1pt] at (-13.1,-1.3) {\includegraphics[width=0.13\textwidth]{images/qpcrect-crop.png}};
	\node[thick,shape=rectangle,draw,inner sep=1pt] at (-5.1,-1.3) {\includegraphics[width=0.13\textwidth]{images/qpcrect-crop.png}};
    \end{tikzpicture}
  \caption{(a) Simulated conductance profile of rectangular quantum point contact in dependence on the width of the constriction. Note the irregularities in the conductance at each new step. (b) The first three conductance steps in more detail. Note the separation visible in the first and third step. The corresponding potential landscape is depicted in the inset.}
\end{figure}
In the areas towards the leads strong interference disturbs the formation of continuous electron channels despite the well pronounced channels within the quantum point contact. The figures show the total electron densities in steady state. The densities include both conducting and non-conducting electrons regardless if stationary or in motion. Thus, the densities offer only an incomplete picture of the kinetics of the electrons. The interference pattern visible in the upper and lower half of \cref{fig:rectedens} can be explained by scattering of the propagating electrons on the confining potential. The reflection of the wave functions of the electrons are source of secondary spherical wave functions, located at the confinement boundaries, whose superposition results in the observed interference pattern. Such an interference can be observed at the points of high electron density both in the corners of the quantum point contact and on a central line parallel to the $x$ axis.\par
The spin density in \cref{fig:rectspindens} and the electron density show a similar interference pattern. Interestingly, two areas of dominant upward polarization to the left and downward polarization to the right side of the constriction are present. This effect could be due to the diffusion of spin injection from the quantum point contact into the lead. The spin scatters from the narrow constriction, in which fewer modes are available, into the wide nanowire, where the available modes almost double.\par
In the conductance profile depicted in \cref{fig:recttrans} distinct steps of height 2~$e^2/h$ are visible. At each step, the conductance increases by one conductance quantum $e^2/h$ per mode per spin-up and spin-down electron. Upon close inspection, an irregularity in the conductance at the transition between modes can be observed. A separation of the conductance step of 2~$e^2/h$  into two smaller steps is expected due to the \emph{zero field spin split} induced by the \textsc{Rashba} interaction \cite{PhysRevB.41.8278}. Due to this interaction the spin degeneracy is lifted and two electrons in the same mode possess different energy whether in the spin-up or spin-down state. An increase of the width of the quantum point contact leads to a decrease of the energy necessary to enter the constriction. Thus, the electron with the higher energy is already able to populate a mode in the constriction at a smaller width.\par
\FloatBarrier
To reduce the abrupt potential change of the rectangular quantum point contact a \emph{triangular} potential is introduced. This geometry is similar to the physical shape of the constriction used by \textsc{van Wees} et. al \cite{PhysRevLett.60.848}. The triangular constriction presented in \cref{fig:trihardwalled} is of variable width in dependence on the $x$ direction, in contrast to \cref{fig:recthardwalled}. The width varies linearly with $x$ from the maximum of $w_{\text{max}} = 2t+w$ to the minimum of $w_{\text{min}} = w$ towards the center of the constriction. In the simulations the tip size is set to $t=100$~nm and depth to $d=50$~nm. Although the walls of the triangular quantum point contact are of infinite slope the potential changes less abruptly in dependence to the $y$ axis compared to the rectangular case. Hence, the electron and spin densities are expected to exhibit a more continuous spatial distribution.\par
\begin{figure}[t!]
  \begin{minipage}[c]{0.5\textwidth}
    \begin{tikzpicture}%[every text node part/.style={align=center}]
	\node at (0,0)[above] {\includegraphics[width=0.9\textwidth]{images/qpctriangular-crop.png}};
	\axes
    \draw[style={<->,thin}] ($1.125*(-1.3,3.6247)$) -- ($1.125*(-0.55,3.25)$) node[midway,left,yshift=-0.5em] {$d$};
    \draw[style={dashed,thin}] (0.3,4.85) -- node (m1){}(1.48,4.43);
    \draw[style={<->,thin}] (m1) -- (0.37,4.45) node[left] {$t$};
    \draw[style={<->,thin}] (-0.25,2.25) -- (0.25,2.53) node[midway,above] {$w$};
  % \draw[help lines,step=0.25cm] (0,0) grid (4,4);
    \end{tikzpicture}
  \end{minipage}
  \begin{minipage}[c]{0.5\textwidth}
    \begin{flalign}
      \quad\text{Pot}(x,y) =\ &\Theta\left(\frac{d}{2}-\abs{x}\right)&\notag\\
      \cdot\ &\Theta\left(\abs{y}-\frac{w}{2}+tx\right)&\notag\\
      \cdot\ &\Theta\left(\abs{y}-\frac{w}{2}-tx\right)&
    \end{flalign}
  \end{minipage}
  \caption{Potential landscape of a hard walled quantum point contact with triangular edges. The longitudinal direction of electron flow $x$ and the transverse direction $y$ is denoted. The depth of the wall $d$ and the width of the confinement $w$ are illustrated by the arrows. The parameter $t$ controls the extent of the triangle tip. The dimensionless analytical expression of the potential is presented on the right. The \textsc{Heaviside} step function is denoted by $\Theta$.}\label{fig:trihardwalled}
\end{figure}
Due to the less abrupt changes of the triangular potential in the $x$ and $y$ direction the electron flow shows more continuous densities in \cref{fig:triedens} as can be observed on the edges of the triangular tips of the constriction. Within the constriction eight peaks in the electron density can be observed corresponding to a conductance of $G=16$~$e^2/h$ at a width of $w=102$~nm. The deviation in the necessary width of $w=105$~nm to obtain the same conductance in the rectangular case lies in the coarse discretization of the triangle tip. The electron density within the constriction exhibits less pronounced minima and is spread out over the central region of the quantum point contact. This spread out electron density indicates an increased interaction that hinders the formation of well pronounced modes. Similar to the rectangular quantum point contact there appear a number of points of high electron density in the upper and lower part. The maxima in the electron density are due to the reflection of the electron wave functions on the boundaries of the nanowire and the triangular quantum point contact. The localized peaks in electron density near $(x,y) = (45,100)$ and $(x,y) = (345,100)$ are due to the constructive interference of the electrons scattering from both the sides of the nanowire at $y=0,200$ and the diagonal sides of the triangular constriction. Interestingly, the maximum remains localized and moves toward the quantum point contact when the width of the constriction is increased.\par
\begin{figure}[h!]
  \subfloat[]{\label{fig:triedens}\includegraphics[]{images/wire400x200-Nov-28-2011-00.47edens}}
  \hspace{14pt}
    \begin{tikzpicture}[overlay]%[every text node part/.style={align=center}]
	\node[thick,shape=rectangle,draw,inner sep=1pt] at (0,-8) {\includegraphics[width=0.13\textwidth]{images/qpctriangular-crop.png}};
    \end{tikzpicture}
  \hspace{4pt}
  \subfloat[]{\label{fig:trispindens}\includegraphics[]{images/wire400x200-Nov-28-2011-00.47spindens}}
  \caption{Simulated electron and spin density for a triangle shaped constriction tip with a width of $w=102$~nm and a depth $d=50$~nm. (a) Two-dimensional electron density. Note the point of high density along the $y$ axis at $y=100$~nm. Color code denotes the electron density $n$. (b) Two-dimensional spin density.  Color code denotes the spin density $s_z$. The corresponding potential landscape is depicted in the inset.}
\end{figure}
In the spin density plot in \cref{fig:trispindens} a spin-split pattern very similar to the rectangular case can be observed. The spin-down electrons and spin-up electrons are separated after passing the constriction in positive $x$ direction. The interference pattern along the central axis in $y$ direction of the wire is more complex than in the rectangular case in \cref{fig:rectspindens}. Additionally, the almost parallel spin split at the very ends of the device and the maximum spin split throughout the device are diminished.\par
It is interesting to note that there are almost no steps visible in \cref{fig:tritrans} although the potential is more adiabatic because of the less abrupt change in $x$ direction. For low intrusion of the quantum point contact tips, smooth steps of height 2 $e^2/h$ are observed. A smoother shape of the steps is expected for devices with less abrupt changes but the walls of the triangular potential are still of \emph{infinite} slope. Here, the error might be due to the  discretization. Because the device is mapped on a square lattice the angled walls of the constriction become highly ragged. This series of longitudinal and transverse wall segments might give rise to a steep increase of scattering effectively smoothing out the conductance steps due to a higher spread in the kinetic energy distribution of the propagating electrons. These effects might explain the unexpected result of the conductance quantization or the lack thereof and must be investigated further.\par
\begin{figure}[h]
  \centering
  \includegraphics[]{images/wire400x200-Nov-28-2011-00.47trans}
    \begin{tikzpicture}[overlay]%[every text node part/.style={align=center}]
	\node[thick,shape=rectangle,draw,inner sep=1pt] at (-5.2,3.55) {\includegraphics[width=0.13\textwidth]{images/qpctriangular-crop.png}};
    \end{tikzpicture}
  \caption{Simulated conduction profile of a triangular quantum point contact in dependence on the width of the constriction. The corresponding potential landscape is depicted in the inset.}\label{fig:tritrans}
\end{figure}
\FloatBarrier
\subsection{Soft Walled Quantum Point Contacts}
The geometrical shape of the constriction influences the transport properties of the device as presented in the previous section. Changing the geometry perpendicular to the $z$ axis smoothed out most features related to quantized conductance either physically or by introducing a numerical error due the discretization not found in the actual continuous system. An alternative approach is to alter the slope of the confining potential landscape perpendicular to the $x$ and $y$. A potential that varies only slowly in all directions will be called a \emph{soft walled} potential, in contrast to the hard walled potentials considered before. A potential with \emph{spherical} tips illustrated in \cref{fig:sphericalsemihardwalled} has a shape with characteristics of both the abrupt and adiabatically varying potential walls. The hard walls perpendicular to each lateral axis are gradually smoothed towards higher potential. The rectangular or triangular tip of the gate electrodes are replaced by a quarter circle of radius $r=30$~nm, which mimics the potential introduced by a top gate above the 2DEG for small intrusions.\par
\begin{figure}[!h]
  \begin{minipage}[c]{0.5\textwidth}
    \begin{tikzpicture}%[every text node part/.style={align=center}]
	\node at (0,0)[above] {\includegraphics[width=0.9\textwidth]{images/qpcspherical-crop.png}};
	\axes
  \draw[style={dashed,thin}] (-2,1.325) coordinate (r1) -- (-2,2.5);
  \draw[style={<->,thin}] (-2.75,1.825) -- (r1) node[midway,left,yshift=-0.5em] {$r$};
  % \draw[style={<->,thin}] (m1) -- (0.37,4.45) node[left] {$t$};
  \draw[style={->,thin}] (-0.16,2.34) -- (0.25,2.53) node[midway,above] {$w$};
  % \draw[help lines,step=0.25cm] (-4,0) grid (4,4);
      \end{tikzpicture}
      \end{minipage}
  \begin{minipage}[c]{0.5\textwidth}
   \begin{flalign}\quad\text{Pot}(x,y) =&\sqrt{r^2-x^2-Y(y)^2\,\Theta(Y(y))}&\notag\\
   % \cdot\ &\Theta\left(\frac{\abs{y}-w/2}{t}+x\right)&\notag\\
   \intertext{\hspace{4.7em}with}
    Y(y)=&\left(r+w-\abs{y}\right)&
   \end{flalign}
   \end{minipage}
  \caption{Potential landscape of a quantum point contact with spherical tips. It exhibits both steep and smooth walls. The longitudinal direction of electron flow $x$ and the transverse direction $y$ is denoted. The radius of the wall $r$ and the width of the confinement $w$ are illustrated by the arrows. The dimensionless analytical expression of the potential is presented on the right. The \textsc{Heaviside} step function is denoted by $\Theta$.}\label{fig:sphericalsemihardwalled}
\vskip -2em
\end{figure}
\begin{figure}[h!]
  \subfloat[]{\label{fig:sphericaledens}\includegraphics[]{images/wire400x200-Nov-27-2011-21.41edens}}
  \hspace{14pt}
    \begin{tikzpicture}[overlay]%[every text node part/.style={align=center}]
	\node[thick,shape=rectangle,draw,inner sep=1pt] at (0,-8) {\includegraphics[width=0.13\textwidth]{images/qpcspherical-crop.png}};
    \end{tikzpicture}
  \hspace{4pt}
  \subfloat[]{\label{fig:sphericalspindens}\includegraphics[]{images/wire400x200-Nov-27-2011-21.41spindens}}
  \caption{Simulated electron and spin density of a quantum point contact with spherical constriction of width $w$=100 nm, and radius of the tip $r$=60 nm. (a) Two-dimensional electron density. Note the well defined continuous areas of electron density within the constriction. Color code denotes the electron density $n$. (b) Two-dimensional spin density. Color code denotes the spin density $s_z$. The corresponding potential landscape is depicted in the inset.}
\end{figure}
In \cref{fig:sphericaledens} a resemblance to the electron distribution of prior quantum point contacts can be observed. The 2DEG under influence of the spherical constriction exhibits increased continuous electron densities as well as points of high electron densities symmetrically arranged in the upper and lower half. This interference pattern is similar to the electron density under the influence of the quantum point contact with triangular tips. The potential which more adiabatic in comparison to the hard walled potentials leads to extended continuous electron densities both in the center and at the edges of the constriction. Within the constriction eight well pronounced maxima can be observed which indicates the establishment of distinct modes and steps also indicated by the conductance profile in \cref{fig:sphericaltrans}. Despite the similarities to the triangular case clearly pronounced steps are seen in the conductance profile. The variations near the steps are smoothed out without the loss of discrete steps of height 2~$e^2/h$.  The spherical potential is used to validate the results of the conductance calculations in \cref{sec:validation}.\par
\begin{figure}[h]
  \centering
  \includegraphics[]{images/wire400x200-Nov-27-2011-21.41trans}
    \begin{tikzpicture}[overlay]%[every text node part/.style={align=center}]
	\node[thick,shape=rectangle,draw,inner sep=1pt] at (-5.2,3.7) {\includegraphics[width=0.13\textwidth]{images/qpcspherical-crop.png}};
    \end{tikzpicture}
  \caption{Simulated conduction profile of a quantum point contact with spherical constriction in dependence on the width of the constriction. The corresponding potential landscape is depicted in the inset.}\label{fig:sphericaltrans}
\end{figure}
The spin densities pictured in \cref{fig:sphericalspindens} differ only slightly from the triangular and rectangular case. The spin split in the lower half of the figure is still the dominant feature due to the reduced irregularity of the interference pattern. Additionally, the maximum spin split increases in respect to the triangular quantum point contact. \par
Electric point charges are introduced in the vicinity of the 2DEG to model a constriction formed by side gate electrodes more closely. A quantum point contact of this kind exhibits a very slow varying potential throughout large parts with almost infinite slopes near the location $r_q$ of the point charges due to the $\sim 1/r_q$ dependency. \Cref{fig:pointsoftwalled} illustrates such a constriction formed by two electric point charges.
\begin{figure}[b!]
  \begin{minipage}[c]{0.5\textwidth}
    \begin{tikzpicture}%[every text node part/.style={align=center}]
	\node at (0,0)[above] {\includegraphics[width=0.9\textwidth]{images/qpcpoint-crop.png}};
	\axes
  \draw[style={<->,thin}] ($1.125*(-1.6,3)$) -- ($1.125*(1.3,4.3)$) node[midway,above] {$w$};
      \end{tikzpicture}
      \end{minipage}
  \begin{minipage}[c]{0.5\textwidth}
   \begin{flalign}\quad\text{Pot}(x,y) &= \frac{q}{\sqrt{x^2+(y+w/2)^2}}&\notag\\
   &+\frac{q}{\sqrt{x^2+(y-w/2)^2}}&\end{flalign}
      \end{minipage}
  \caption{Potential landscape of a soft walled quantum point contact of two point charges of charge $q$ and dislocation $w$. The longitudinal direction of electron flow $x$ and the transverse direction $y$ is denoted. The dimensionless analytical expression of the potential is presented on the right.}\label{fig:pointsoftwalled}
\end{figure}
In the simulations, either the geometry of the constriction or the charge of the point charges is varied. Modifying the width of the quantum point contact is usually experimentally achieved by the placement of several gate electrodes either on or beside the 2DEG. A negative potential is applied to the electrodes which effectively results in a depopulation of the conductance band in their vicinity. The electrodes can experimetally be realized by oxidizing the surface above the 2DEG to form depleted isolating regions \cite{JApplPhys73.262} \cite{ElDevLett31.1227}. The use of isolated areas of 2DEG as electrodes is illustrated in \cref{fig:sidegatesketch}.\par
\begin{figure}[t!]
  \subfloat[]{\label{fig:sidegatesketch}
      \begin{tikzpicture}[circuit ee IEC,thick]
	\node[above] at (0,0) {\includegraphics[scale=0.9]{images/sidegates}};
	\draw (-3.2,2.7) -- ++(0,-2.4) -- (0,0.3) node[contact] (contact1){} -- ++(3.2,0) -- ++(0,2.2);
	\draw (0,-0.8) node[rotate=-90,ground]{} to [battery={info={$\mp$}}] (contact1);
	\node at (0,4) (2deg) {2DEG};
	\draw[-latex] (2deg)-- (-1.5,2.45);
	\draw[-latex] (2deg)-- (0,3);
	\draw[-latex] (2deg)-- (1.5,2.45);
	\node at (-3,4) (contact1) {Contact};
	\draw[-latex] (contact1)-- ++(0,-1);
	\node at (3.1,4) (contact2) {Contact};
	\draw[-latex] (contact2)-- ++(0,-1.2);
	\node at (-2.5,-0.5) (substrate) {Substrate};
	\draw[-latex] (substrate)-- ++(0.8,1.5);
	\node at (-0.9,0.95) {GaAs};
      \end{tikzpicture}}
  \subfloat[]{\label{fig:differentq}\includegraphics{images/differentq}}
  \caption{(a) Illustration of the realization of sidegates in a 2DEG. The (blue) 2DEG is sketched on a (green) substrate. The application of a voltage via two (golden) leads is outlined. Note the lack of conducting substance in the trenches effectively creating three isolated 2DEGs. (b) Illustration of confining potential at two different charges. Transverse potential profile in Volt at charges $q_1 = 6.5 \cdot 10^{-18}$~C (blue line) and $q_2= 0.5 \cdot 10^{-18}$~C. (dotted green line). A fixed electron energy $E_{e^-}$ is depicted by a red line.} 
\end{figure}
It is important to note that by changing the voltage on the electrodes, and therefore the charge, not only the width of the potential is changed but also the steepness of the confining potential for a given energy, see \cref{fig:differentq}. Hence, the transfer from the adiabatic to the non-adiabatic, i.e. steep walled domain is possible.\par
\begin{figure}[h]
  \subfloat[]{\label{fig:pointedens}\includegraphics[]{images/242255-wire400x200-Dec-02-2011-23.08edens}}
  \hspace{14pt}
    \begin{tikzpicture}[overlay]%[every text node part/.style={align=center}]
	\node[thick,shape=rectangle,draw,inner sep=1pt] at (0,-8) {\includegraphics[width=0.13\textwidth]{images/qpcpoint-crop.png}};
    \end{tikzpicture}
  \hspace{4pt}
  \subfloat[]{\label{fig:pointspindens}\includegraphics[]{images/242255-wire400x200-Dec-02-2011-23.08spindens}}
  \caption{Simulated electron and spin density for a constriction formed by two point charges. The point charges are located at the transverse boundary of the wire 200 nm apart. The charge of the point charges is $q\approx 1.5\cdot 10^{-18}$~C. (a) Two-dimensional electron density. Note the lack of pronounced electron density in the center of the figure and the symmetrical reflection interference in the bottom and top part. Color code denotes the electron density $n$. (b) Two-dimensional spin density. Color code denotes the spin density $s_z$. The corresponding potential landscape is depicted in the inset.}
\vskip -1em
\end{figure}
The electron density illustrated in \cref{fig:pointedens} shows continuous density channels except for the center region which shows strong interference. An interference pattern can also be observed before and after the quantum point contact. The pattern is much simpler in comparison to all prior quantum point contacts. The lateral offset near the upper and lower lead are due to the necessity to cut off the potential at some point which would otherwise extend to infinity.\par
Although the maximum spin split is lower compared to all prior quantum point contacts it exhibits a much simplified interference pattern with localized areas of maximum polarization and otherwise almost negligible polarization, see \cref{fig:pointspindens}. The simplicity of the pattern is due to the reduced number of modes. The probability distribution of the electrons is laterally confined to the same area. Due to the slow decrease of the potential there exists a finite potential throughout the device.\par
\begin{figure}[h]
  \centering
  \subfloat[]{\label{fig:qpcpointtransspin}\includegraphics[]{images/242255-wire400x200-Dec-02-2011-23.08trans}}
  \subfloat[]{\label{fig:qpcpointtransnospin}\includegraphics[]{images/242368-wire400x200-Dec-03-2011-04.35trans}}
    \begin{tikzpicture}[overlay]%[every text node part/.style={align=center}]
	\node[thick,shape=rectangle,draw,inner sep=1pt] at (-13,-1.3) {\includegraphics[width=0.13\textwidth]{images/qpcpoint-crop.png}};
	\node[thick,shape=rectangle,draw,inner sep=1pt] at (-5.1,-1.3) {\includegraphics[width=0.13\textwidth]{images/qpcpoint-crop.png}};
    \end{tikzpicture}
  \caption{Simulated conductance profile of quantum point contact of two point charges. The conductance is given in dependence on the charge of the point charges for $0 < q <7\cdot 10^{-18}$ C. (a) Spin resolved conductance in $e^2/h$. (b) Spin degenerate conductance in 2~$e^2/h$. Note the similarity to \cref{fig:qpcpointtransspin}. Note also the disappearance of conductance quantization below $\approx 2\cdot10^{-18}$~C and the reappearance close to $q=0$. The corresponding potential landscape is depicted in the inset.}
\end{figure}
Both, the spin resolved conductance profile in \cref{fig:qpcpointtransspin} and the spin degenerate conductance profile in \cref{fig:qpcpointtransnospin} show both, the hard and the soft walled contributions to the potential. A well pronounced first and second conductance step can be observed but the steps smooth out more with decreasing charge. Interestingly the steps become well pronounced again when the charge approaches zero. This pattern is also observed for spin degenerate systems as can be observed in \cref{fig:qpcpointtransnospin}. Hence, a relation of this effect to a spin-orbit coupling seems unlikely. The disappearance of steps can be explained by the onset of \emph{inter-subband scattering} which has been found to have significant effect starting at the fourth mode \cite{Lehmann2011}. If inter-subband scattering is indeed the cause for the smoothing of the conductance profile the reappearance of noticeable steps would correspond to the reduction of inter-subband scattering due to some change in the device at lower charge. The only observed change in the environment of the 2DEG at lower charge is the height and steepness of the potential landscape.
Note that as the charge decreases the steepness of the potential increases for a given electron energy $E_{e^-}$ as illustrated in \cref{fig:differentq}. Due to the higher potential gradient the subband mode spacing increases similar to the abrupt wall scenario and the interactions between modes is therefore decreased.
% \FloatBarrier
\subsection{Variational Walled Quantum Point Contacts}
To investigate the relation of the steepness of the potential landscape to the diminishing conductance quantization a parametrized  quantum point contact is introduced into the 2DEG, see \cref{fig:variationalwalled}.\par
\begin{figure}[h!]
  \vskip -1em
    \begin{minipage}[c]{0.5\textwidth}
    \begin{tikzpicture}%[every text node part/.style={align=center}]
	\node at (0,0)[above] {\includegraphics[width=.9\textwidth]{images/qpcvariational2-crop.png}};
  \draw[style={<->,thin}] ($1.125*(-0.205,2)$) -- ($1.125*(0.27,2.25)$) node[midway,above] {$w$};
	\axes
    \coordinate (a) at (1.65,2.3);
    \coordinate (b) at (2.6,2.87825);
    % \coordinate (b) at (2.505,2.81325);
    \coordinate (c) at (2.6,3.982);
    \coordinate (e) at (0.53,2.9);
    \coordinate (f) at (1.45,4.5);
    \draw[-] (a) node[above,xshift=0.5em,yshift=2em]{$m$} -- (b) node[midway,below,yshift=0.3em,xshift=0.8em]{$\Delta y$} -- (c) --cycle;
    \draw[dashed] (a) -- (e);
    \draw[-] (e) -- (f);
    \draw[dashed] (c) -- (f);
      \end{tikzpicture}
  \end{minipage}
  \begin{minipage}[c]{0.5\textwidth}
  \begin{flalign}\quad\text{Pot}(x,y) &= e^{-x^2/\xi^2}\cdot m\left(\abs{y}- \frac{w}{2}\right)&\end{flalign} 
  \end{minipage}
  % \caption{hallo}
  \caption{Example potential landscape of a variational walled quantum point contact used in the analysis. The longitudinal direction of electron flow $x$ and the transverse direction $y$ is denoted. The width at the bottom of the potential $w$ is indicated by arrows. The slope of the transverse confinement $m$ is illustrated by a gradient triangle. The parameter $\xi$ controls the steepness of the walls in $x$ direction. The dimensionless analytical expression of the potential is presented on the right.}\label{fig:variationalwalled}
  \vskip -2em
\end{figure}
A potential which can be varied between hard and soft walled will be called \emph{variational potential}. The variational potential is designed to investigate the dependence of the conduction on the slope of the constricting potential. Here the walls can be symmetrically modified to achieve different but constant slopes $m$ of the confinement. The slope is taken to be the $y$ gradient at the center of the dimensionless potential landscape presented in \cref{fig:variationalwalled} and thus is given in units of nm$^{-1}$. The parameter $\xi$ controls the steepness of the walls in $x$ direction and is set to $\xi=10$~nm to ensure an adiabatic increase of the potential in the $x$ direction.\par
The effects of the wall slope on the smoothness of the conductance profile and thus possibly on the inter-subband scattering are investigated by increasing the width of the constriction for many different slopes $m$. The conductance quantization varies noticeably only in the interval $0 < m < 0.4$, see \cref{fig:slopes}. Thus for slopes $m > 0.4$ the potential could be considered hard walled.  Quantized steps in the conductance can already be observed above a slope of $m \approx 0.03$, with only slight changes if the slope increases further in that interval. For slowly varying potentials with slopes $m<0.032$ a noticeable change in conduction quantization can be observed, see \cref{fig:slope_cuts}. A significant change in quantization is only observed in the slope interval $0.008 < m < 0.032$. The offset in the conduction of quantum point contacts with varying slope is due to the relative width of the quantum point contact at a specific energy. For smaller slopes the constriction is wider and therefore a higher conductance compared to larger slopes is obtained. \Cref{fig:slopes} shows conductance profiles for constant slopes in the interval $0 < m < 0.4$.\par
\begin{figure}[h] 
  \subfloat[]{\label{fig:slope_cuts}\includegraphics[]{images/slope_cuts.pdf}}
  \subfloat[]{\label{fig:slopes}\begin{tikzpicture}[font=\footnotesize]
  \node[above] at (0,0){\includegraphics[width=0.46\textwidth]{images/slopesrainbow.png}};
  \draw[style={<->,thin}] (-3.8,2.5) -- node[below,align=center,rotate=-44] {Slope [nm$^{-1}$]} (-1.15,0) node[left,yshift=-0.2em] {0} --   node[below,yshift=-0.5em,xshift=-1em,rotate=25] {Width [nm]} (3.75,2.2);
  \draw[thin] (-3.7,2.4)--++(0.05,0.05) node[below,xshift=0.2em,yshift=-0.5em,rotate=-44] {0.4};
  \draw[thin] (3.65,2.16)--++(-0.03,0.065) node[below,xshift=0em,yshift=-0.3em,rotate=25] {200};
  \node[above] at (2.6,-0.3){\includegraphics[width=0.25cm,height=3cm]{images/slopescolorbar.png}};
  % \draw[step=0.25cm,gray,very thin] (0,0) grid (3.5,3);
  \node[right] at (2.7,0) {0};
  \node[right] at (2.7,2.65) {30};
  \node[below, rotate=90] at (2.7,1.17) {$G$ $[e^2/h]$};
  \end{tikzpicture}}
    \begin{tikzpicture}[overlay]%[every text node part/.style={align=center}]
	\node[thick,shape=rectangle,draw,inner sep=1pt] at (-13.5,-1.3) {\includegraphics[width=0.13\textwidth]{images/qpcvariational-crop.png}};
    \end{tikzpicture}
  \caption{(a) Conductance profiles at different slopes $m$. Note the disappearance of quantization for small $m$. (b) Full conduction profile in dependence on slope $0 < m < 0.4$ and width $0 < w < 200$ nm. Note the increased smoothing towards small slopes $m$. Color code denotes the conductance $G$ in $e^2/h$. The corresponding potential landscape for one $m$ is depicted in the inset.}\label{fig:conductances}
\end{figure}
The decrease of the conductance quantization for small slopes is likely due to the increase of inter-subband interaction. If the energy separation of the modes decreases comparable to the decrease of energy separation of the eigenvalues of the \textsc{Airy} functions in the triangular potential well the interaction between the eigenstates becomes significant. The electron densities presented in \cref{fig:variedens1} and \cref{fig:variedens2} provide further indications of inter-subband scattering. These figures show that the electron density for the softer quantum point contact is much less localized in discrete channels. This indicates an increase in interaction of the modes passing through the quantum point contact.\par
Interestingly, the spin split is increased for the smoother walled quantum point contact in \cref{fig:varispindens1} in comparison to \cref{fig:varispindens2,fig:trispindens,fig:rectspindens,fig:pointspindens}. Additionally the spin split exhibits less dominant interference than any hard walled potential. This might be caused by the smoother potential landscape and thus reduced possibility of a scattering event that alters the electron propagation significantly.\par
\begin{figure}[h]
  \subfloat[]{\label{fig:variedens1} \includegraphics[]{images/1edens-crop}}
  \hspace{14pt}
    \begin{tikzpicture}[overlay]%[every text node part/.style={align=center}]
	\node[thick,shape=rectangle,draw,inner sep=1pt] at (0,-8) {\includegraphics[width=0.13\textwidth]{images/qpcvariational-crop.png}};
    \end{tikzpicture}
  \hspace{4pt}
  \subfloat[]{\label{fig:varispindens1} \includegraphics[]{images/1spindens}}
  \caption{Simulated electron and spin density for a constriction formed by a variational walled constriction of slope $m=0.008$. Densities correspond to a conductance of 16 $e^2/h$ as is indicated by the eight maxima within the constriction. (a) Two-dimensional electron density. Note the smoothed out electron density within the constriction. Color code denotes the electron density $n$. (b) Two-dimensional spin density. Color code denotes the spin density $s_z$. The corresponding potential landscape is depicted in the inset.}
\end{figure}
\begin{figure}[h]
  \subfloat[]{\label{fig:variedens2}\includegraphics[]{images/4edens-crop}}
  \hspace{14pt}
    \begin{tikzpicture}[overlay]%[every text node part/.style={align=center}]
	\node[thick,shape=rectangle,draw,inner sep=1pt] at (0,-8) {\includegraphics[width=0.13\textwidth]{images/qpcvariational-crop.png}};
    \end{tikzpicture}
  \hspace{4pt}
  \subfloat[]{\label{fig:varispindens2}\includegraphics[]{images/4spindens}}
  \caption{Simulated electron and spin density for a constriction formed by a variational walled constriction of slope $m=0.032$. Densities correspond to a conductance of 16 $e^2/h$ as is indicated by the eight maxima within the constriction. (a) Two-dimensional electron density. Note the distinct areas of electron density within the constriction. Color code denotes the electron density $n$. (b) Two-dimensional spin density. Note the diminished maximal spin split in comparison to \cref{fig:varispindens1}. Color code denotes the spin density $s_z$. The corresponding potential landscape is depicted in the inset.}
\end{figure}
The results present strong indications for the interaction of wall slope and inter-subband scattering similar to the mixing of intra-subband eigenstates due to higher order effects of the \textsc{Rashba} spin-orbit interaction \cite{Wolfgang2003PhysicaE.18.337}. The influence of spin-orbit coupling by the shape and slope of the constriction could also explain the difference in spin polarization and could be subject to further analysis.
\FloatBarrier
\section{Non-Collinear devices}
The simulated electron and spin densities in the prior chapters show that both the electron density and the spin split induced by the \textsc{Rashba} interaction in the 2DEG are symmetric in regard to the $x$ axis. The electron densities exhibit reflection symmetry to a plane parallel to the $x$ axis in the center of the conductor whereas the spin densities show a rotational symmetry in respect to the $x$ axis. The geometry of the device can be changed to induce a break of the symmetry of the spin split. Therefore, asymmetric geometries in respect to the $x$ axis can be used to analyze the changes in spin polarization. To that end the assumption of collinear leads has to be dropped.
\subsection{Curved Nanowire}
A bend in the nanowire already introduces complex interactions both in real and in spin-space. This is demonstrated for the electron density in \cref{fig:curvedens1,fig:curvedens2,fig:curvedens3} and for the spin density in \cref{fig:curvspindens1,fig:curvspindens2,fig:curvspindens3}. The first three modes correspond to the energies 0.003~eV, 0.036~eV and 0.081~eV, respectively, of a straight nanowire of constant width.\par
\begin{figure}[h!]
  \centering
  \subfloat[]{\label{fig:curvedens1}\begin{tikzpicture}
  \node[above] at (0,0){\includegraphics[]{images/curve_1modes_edens-crop}};
  \curveoutline
  \end{tikzpicture}}
  \subfloat[]{\label{fig:curvedens2}\begin{tikzpicture}
  \node[above] at (0,0){\includegraphics[]{images/curve_2modes_edens-crop}};
  \curveoutline
  \end{tikzpicture}}
  \subfloat[]{\label{fig:curvedens3}\begin{tikzpicture}
  \node[above] at (0,0){\includegraphics[]{images/curve_3modes_edens-crop}};
  \curveoutline
  \end{tikzpicture}}
  \caption{Electron densities of a curved nanowire of 40 nm width. The wire width is marginally reduced in the curve as outined by the gray lines. (a) First mode of energy $E=0.003$ eV. (b) Second mode of energy $E=0.036$ eV. (c) Third mode of energy $E=0.081$ eV. Color code indicates electron density $n$.}
\end{figure}
The electron densities are also slightly elevated due to the singularities in the LDOS at the band edge of these modes. If compared to \cref{fig:dens1,fig:dens2,fig:dens3}, for example, an apparent change is immediately seen. The spacial distribution of electrons along the wire is not constant and the spin densities exhibit a preference for spin-up electrons, see \cref{fig:curvspindens2}, for example. Note that the scale changes about the order of two in the \cref{fig:curvedens1,fig:curvedens2,fig:curvedens3} to prevent the loss of details for smaller electron densities. Likewise, the scale of the spin density in \cref{fig:curvspindens2} is due to the same reasons about twice as large compared to \cref{fig:curvspindens1,fig:curvspindens3}. The features of the curved 2DEG are energy dependent. For all three modes the spin-down density is suppressed. For the third mode the spin density shows an irregular interference pattern. This shows that already a slight departure from a symmetrical geometry alters the scale and the distribution of the spin density significantly.\par
\begin{figure}[h!]
  \centering
  \subfloat[]{\label{fig:curvspindens1}\begin{tikzpicture}
  \node[above] at (0,0){\includegraphics[]{images/curve_1modes_spindens-crop}};
  \curveoutlinedark;
  \end{tikzpicture}}
  \subfloat[]{\label{fig:curvspindens2}\begin{tikzpicture}
  \node[above] at (0,0){\includegraphics[]{images/curve_2modes_spindens-crop}};
  \curveoutlinedark;
  \end{tikzpicture}}
  \subfloat[]{\label{fig:curvspindens3}\begin{tikzpicture}
  \node[above] at (0,0){\includegraphics[]{images/curve_3modes_spindens-crop}};
  \curveoutlinedark;
  \end{tikzpicture}}
  \caption{Spin densities of a curved nanowire of 40 nm width. The wire width is marginally reduced in the curve as outlined by the gray lines. (a) First mode of energy $E=0.003$ eV. (b) Second mode of energy $E=0.036$ eV. (c) Third mode of energy $E=0.081$ eV. Note the increasing irregularity of the spin split and the diminished spin-down density in (a) and (b). Color code indicates spin density $s_z$.}
\end{figure}
\subsection{\textsc{Aharonov-Bohm} Ring Interferometer}
The changes in spin density under the influence of an asymmetrical confinement can be analyzed in a ring interferometer with rotatable leads. An interferometer is a common approach to the manipulation of properties of wave objects. A realization of such a ring interferometer is given by a modification of the \textsc{Arahonov-Bohm} ring. The \textsc{Arahonov-Bohm} ring is a two-dimensional annulus where the difference of inner and outer radius is smaller than the inner radius $\abs{r_o- r_i} \ll r_i$, see \cref{fig:aharonovbohmring}.
Spin interference has been achieved via the attachment of multiple leads \cite{PhysRevB.75.035304} or the application of an electro-magnetic field \cite{PhysRevB.69.155335} to the \textsc{Arahonov-Bohm} ring.\par
\begin{figure}[!h]
  \centering
  \begin{tikzpicture}
    % \draw[step=.5cm,gray,very thin] (-2,-2) grid (2,2);
    % \draw[dashed] (1.5,0) -- (1.5,0);
    \draw[dashed] (0,-2) -- (0,2);
    \draw[dashed] (0,-2) -- (0,2);
    \draw[dashed] (0,0) -- (-2,0);
    % \draw[dashed] (1,0) -- ++(0,1.5);
    \draw[dashed] (1.5,0) -- ++(0,2);
    \draw[<->] (0,1.75) -- node[above] {$r_o$}++(1.5,0);
    \draw[<->] (0,0) -- node[above] {$r_i$}++(1,0);
    \draw[thick] (0,0) circle (1);
    \draw[thick] (1.5,0) arc (0:80:1.5) coordinate (n1);
    \draw[thick] (-2,0.25)-- ++(0.51,0) coordinate(leftupper);
    \draw[thick] (leftupper) arc (-10:-80:-1.5) coordinate (n2);
    \draw[thick] (n1)-- ++(0,0.5);
    \draw[thick] (n2)-- ++(0,0.5);
    \draw[thick] (1.5,0) arc (0:-80:1.5) coordinate (n3);
    \draw[thick,dashed] (n3)-- ++(0,-0.5);
    \draw[thick] (-2,-0.25)-- ++(0.51,0) coordinate(left1);
    \draw[thick] (left1) arc (10:80:-1.5) coordinate (n4);
    \draw[thick,dashed] (n4)-- ++(0,-0.5);
    \node (a) at (-1.5,-1.5) {$\alpha$};
    \begin{scope}[dashed,decoration={
      markings,
      mark=at position 0.5 with {\arrow{latex}}}
      ] 
      \draw[postaction={decorate}] (0,-1.75) arc (270:180:1.75);
  \end{scope}
  \end{tikzpicture}
  \caption{\textsc{Aharonov-Bohm} ring based interferometer of inner radius $r_i=65$~nm and outer radius $r_o=75$~nm. One of the leads is rotated. The angle of rotation is denoted by $\alpha$. Here $\alpha=\pi/2$.}\label{fig:aharonovbohmring}
\end{figure}
The most basic interferometer is obtained by the attachment of two leads to the \textsc{Aharonov-Bohm} ring \cite{PhysRevLett.79.273}. To investigate changes in spin properties of the 2DEG one of the leads is then rotated against the other to obtain different enclosed angles, see \cref{fig:aharonovbohmring}. The ring segments which are separated by the leads represent the arms of the interferometer. The symmetry of the interferometer is broken by rotating the lower lead and the distriution of electron and spin densities is altered significantly.\par
The energy is chosen such that only the first mode is formed in the ring and leads. The electron density for collinear leads does not exhibit indications of interference in either lead, see \cref{fig:ring0edens}. When one lead is rotated an interference pattern becomes apparent in both arms of the interferometer, see \cref{fig:ring90edens,fig:ringneg90edens}. The peaks and troughs are due to the scattering of the electron wave functions at each intersection with the lead. Note that \cref{fig:ring90edens} is identical to \cref{fig:ringneg90edens} if rotated 90$^{\circ}$ clockwise. Again, the electron density is elevated due to its energy dependence near the band edges.\par
\begin{figure}[h!]
  \subfloat[]{\label{fig:ring0edens}\includegraphics[trim=0 0 0 6pt,clip]{images/ring0edens}}
  \subfloat[]{\label{fig:ring90edens}\includegraphics[trim=0 0 0 6pt,clip]{images/ring90edens}}
  \subfloat[]{\label{fig:ringneg90edens}\includegraphics[trim=0 0 0 6pt,clip]{images/ringneg90edens}}
  \caption{Electron densities in \textsc{Aharonov-Bohm} type ring for three different enclosing angles. (a) Enclosing angle $\alpha=0$. (b) Enclosing angle $\alpha=\pi/2$. (c) Enclosing angle $\alpha=-\pi/2$. Color code denotes electron density $n$.} 
\end{figure}
\begin{figure}[h!]
  \subfloat[]{\label{fig:ring0spindens}\includegraphics[]{images/ring0spindens}}
  \subfloat[]{\label{fig:ring90spindens}\includegraphics[]{images/ring90spindens}}
  \subfloat[]{\label{fig:ringneg90spindens}\includegraphics[]{images/ringneg90spindens}}
  \caption{Spin densities in \textsc{Aharonov-Bohm} type ring for three different enclosing angles. (a) Enclosing angle $\alpha=0$. (b) Enclosing angle $\alpha=\pi/2$. (c) Enclosing angle $\alpha=-\pi/2$. Color code denotes the spin density $s_z$. Note the dependence of the spatial spin distribution and spin-up to spin-down ratio on enclosed angle.}
\end{figure}
The interference of spin is already for collinear leads, i.e $\alpha=0$, well pronounced, as can be observed in \cref{fig:ring0spindens}. In contrast to a wire or closed ring \cite{PhysRevB.82.165322}, the spin split depends on the angular coordinate $\phi$.
The spins interact to form a standing wave in steady state. The circumference of a concentric circle in the center of inner and outer circle is $c\approx 408$~nm. For the energy $E=0.068$~eV 23 peaks and troughs of the standing wave can be observed which corresponds to a wavelength of $\lambda_{\text{spin}}=17.75$~nm. This length is well below the spin-precession length in undisturbed geometries of $L_{SO}\approx230$~nm due to the strength of the \textsc{Rashba} spin-orbit interaction $\alpha_{RSO}$ considered.\par
Upon rotation of the lower lead, the interference pattern changes considerably in amplitude and spatial profile. The phase of the standing wave appears to be fixed to the bottom lead and rotates the peaks and troughs according to the angle the lead has been rotated. Whereas the polarization in the leads is almost negligible throughout \cref{fig:ring0edens,fig:ring90edens,fig:ringneg90edens,fig:ring0spindens,fig:ring90spindens,fig:ringneg90spindens} the amplitude increases sharply within the ring. The amplitudes of spin polarization within the \textsc{Aharonov-Bohm} ring with rotated lead decrease gradually with increasing rotation about one order of magnitude.\par
For the presented angles of 0, $\pi/2$ and $-\pi/2$ three very distinct situations occur. While the maximum spin-up to spin-down amplitude ratio is unity for the symmetric device in \cref{fig:ring0spindens} it reaches $s_{\uparrow}/s_{\downarrow}=5$ for $\alpha=\pi/2$ rotation in \cref{fig:ring90spindens} and $s_{\uparrow}/s_{\downarrow}=0.2$ for $\alpha=-\pi/2$ rotation in \cref{fig:ringneg90spindens}. The electron densities in the ring with rotated leads are about twice as large in the longer arms than the shorter arms which is, at least in part, correlated to the increase in spin density.\par
Qualitatively, this result is reproduced analytically for a one dimensional ring for the presented rotations. The diagonalization of the \hamil{} \begin{align}
\mat{H}_{1D}=\hbar\omega_0 \left[\left(-i\pd{}{\phi}+\frac{\omega_{SO}}{2\omega_{0}} (\mat{\sigma}_x\text{cos}\phi+\mat{\sigma}_y\text{sin}\phi)\right)^2-\frac{\omega_{SO}^2}{4\omega_0^2}\right]\ ,
\end{align}
for a charged particle of mass $m^*$ in the presence of spin-orbit interaction in a closed ring of radius $a$ in polar coordinates \cite{PhysRevB.73.155325} 
leads to the four eigenstates in terms of the \textsc{Pauli} spin states ($\begin{pmatrix}1\\0\end{pmatrix}$ and $\begin{pmatrix}0\\1\end{pmatrix}$) in each arm \cite{nitta1999.695}
\begin{align}
  \Psi^{\pm}_{\uparrow} = e^{i\phi}\begin{pmatrix}\text{cos}(\theta/2)\\\pm\text{sin}(\theta/2)e^{i\phi}\end{pmatrix} 
  \qquad\text{and}\qquad
  \Psi^{\pm}_{\downarrow} = e^{\pm i\phi}\begin{pmatrix}\pm\text{sin}(\theta/2)e^{-i\phi}\\-\text{cos}(\theta/2)\end{pmatrix}\ ,
\end{align}
with the dimensionless kinetic energy of the particle $\hbar\omega_0=\hbar^2/2m^*a^2$, and the frequency associated to the spin-orbit interaction $\omega_{SO} = \alpha/\hbar a$. Here, only the \textsc{Rashba} spin-orbit interaction is considered. Thus, the parameter for the strength of the interaction is $\alpha = \alpha_{RSO}$, confer \cref{eqn:alpharashba}. Plus and minus denote the direction of travel and $\uparrow,\downarrow$ spin up and spin down respectively. The spin tilt angle is constant and given by \cite{PhysRevB.71.033309}
\begin{align}
\theta = \text{arctan}(-\omega_{SO}/\omega_0)\ .
\end{align}
Expanding each wave function in terms of the eigenstates, the interference pattern in \cref{fig:ring90spindens} can be reproduced, see \cref{fig:spinors}. Here, partial reflection of the spin-up wave function and total transmission of the spin-down wave function at the junctions in the right arm and partial reflection of both spin states at the junctions in the left arm is assumed. The the spin density for collinear leads is reproduced in \cref{fig:spinors0}. Here, no reflections are assumed. The analytical spin amplitudes do not exhibit the fivefold increase for the rotated leads because the spin split in the leads is not considered.\par
\begin{figure}[!h]
  \centering
  \includegraphics[trim=0 2mm 0 3mm,clip]{images/spinors}
  \caption{Expansion of spin state $s_z$ in eigenstates of the closed ring for an enclosing angle of $\alpha = \pi/2$. Negative $\phi$ corresponds to the left arm and positive $\phi$ to the right. The interference pattern is only shown for the first three peaks. Left arm state (dashed green line) corresponds to partial reflection ($\approx$ 20\%) of $\uparrow$ and $\downarrow$ states equally. Right arm state (blue line) corresponds to no reflection of $\downarrow$ and strong reflection of $\uparrow$ state ($\approx$ 60\%).}\label{fig:spinors}
\end{figure}
\begin{figure}[t]
  \centering
  \includegraphics[trim=0 0 0 1cm, clip]{images/spinors0}
  \caption{Expansion of spin state $s_z$ in eigenstates of the closed ring for an enclosing angle of $\alpha = 0$. Negative $\phi$ corresponds to the left arm and positive $\phi$ to the right. The interference pattern is only shown for the first three peaks. Both the left arm state (dashed green line) and the right arm state (blue line) correspond to no reflection at the leads.}\label{fig:spinors0}
\end{figure}
\begin{figure}[b!]
  \subfloat[]{\label{fig:ring15spindens}\includegraphics[trim=0 0 0 3pt,clip]{images/ring3spindens}}%\hspace{-0.5em}
  \subfloat[]{\label{fig:ring25spindens}\includegraphics[trim=0 0 0 3pt,clip]{images/ring5spindens}}%\hspace{-0.5em}
  \subfloat[]{\label{fig:ring30spindens}\includegraphics[trim=0 0 0 3pt,clip]{images/ring6spindens}}
  \caption{Spin densities in \textsc{Aharonov-Bohm} type ring for three different enclosing angles. (a) Enclosing angle $\alpha\approx 1/12 \pi$. (b) Enclosing angle $\alpha=1/7\pi$. (c) Spin density $s_z$. Enclosing angle $\alpha=1/6\pi$. Color code denotes the spin density $s_z$. Note the dependence of the spatial spin distribution and spin-up to spin-down ratio on enclosed angle.}
\end{figure}
The model of pure eigenstates of the isolated ring breaks down for arbitrary enclosing angles as the coupling to the leads becomes significant. The coupling to the leads can be observed in \cref{fig:ring15spindens,fig:ring25spindens,fig:ring30spindens}. The spin density in the upper lead is dominated by the spin-up state at an enclosing angle of $\alpha\approx 1/12\pi$ and $\alpha= 1/6\pi$ whereas the spin-down state is dominant for a rotation of $\alpha\approx 1/7\pi$. Due to the coupling of the spin states to the leads, the eigenstates of the isolated ring do not give an accurate model of the physical situation. The corresponding electron densities are shown in \cref{fig:ring15edens,fig:ring25edens,fig:ring30edens}. They illustrate the significant changes of the interference pattern in the arms already with small differences between the enclosing angles.\par 
The application of the analytical model is also limited by the assumptions made in the \hamil{}. Outside of the \gfnc{} theory, it describes only a single particle which is insufficient for typical carrier densities found in 2DEGs. Additionally, the inclusion of further interactions such as the \textsc{Dresselhaus} effect \cite{PhysRev.100.580} is necessary to model a 2DEG in heterostructures with significant bulk inversion asymmetry. For this, the model would have to be extended to two dimensions to take the anisotropy of the \textsc{Dresselhaus} interaction into account. Furthermore, the leads influence on the states in the \textsc{Aharonov-Bohm} ring is not negligible and needs to be included within a theory of open quantum systems.\par The approach of a numerical simulation based on the \gfnc{} formalism is limited by similar assumptions. Besides the errors introduced by the discretization, interactions like the \textsc{Dresselhaus} effect or electric and magnetic fields~\cite{Nitta2002PhysicaE.12.753} are a necessary addition to the \hamil{} for a more realistic simulation of the spin interference. 
A \hamil{} with included electron-electron interaction would lead to both a spread out electron density due to \textsc{Coulomb} repulsion and electron correlations effects in the interference pattern. The omission of these interactions is not due to a limitation of the \gfnc{} formalism and can be included by adding an appropriate term to the \hamil{}.
\begin{figure}[t!]
    \subfloat[]{\label{fig:ring15edens}\includegraphics[trim=0 0 0 6pt,clip]{images/ring3edens}}%\hspace{-0.5em}
    \subfloat[]{\label{fig:ring25edens}\includegraphics[trim=0 0 0 6pt,clip]{images/ring5edens}}%\hspace{-0.5em}
    \subfloat[]{\label{fig:ring30edens}\includegraphics[trim=0 0 0 6pt,clip]{images/ring6edens}}
    \caption{Electron densities in \textsc{Aharonov-Bohm} type ring for three different enclosing angles. (a) Enclosing angle $\alpha\approx 1/12 \pi$. (b) Enclosing angle $\alpha\approx 1/7\pi$. (c) Enclosing angle $\alpha=1/6\pi$. Color code denotes electron density $n$.} 
\end{figure}

\chapter{Conclusion and Outlook}
In this thesis the non-equilibrium \gfnc{} formalism is implemented to investigate the electron density, spin density and conductance of two-dimensional electron systems.\par
In chapter 1 the \gfnc{} formalism for the quantum mechanical discription of two dimensional electron gases is presented.\par
In chapter 2 the discretized \hamil{} in the finite differences approximation and the reduction to a finite system of equations via the self-energy is discussed.\par
In chapter 3 the implementation of a simulator based on the recursive \gfnc{} algorithm is presented. A graph based matrix reordering algorithm is described allowing the application of the recursive \gfnc{} algorithm to arbitrary geometry.
Furthermore the computed electron density is compared to analytical results and the spin density and conductance are compared to experimental data.\par
In chapter 4 the results of spin-dependent transport simulations for different colinear quantum point contacts and a non-colinear \textsc{Aharonov-Bohm} interferometer are examined. The influence of the potential landscape which is induced by the quantum point contact on the 2DEG is analyzed.
The electron density is found to mirror the geometrical constriction closely. An interference of spin polarization normal to the 2DEG can be observed. The conductance profile shows dependence on the spatial gradient of the potential landscape. Abrupt potential changes show well defined quantization while more gradual changes in potential smooth out the discrete steps in the conductance.
The spin density in an \textsc{Aharonov-Bohm} interferometer with two arms exhibits strong dependence on the enclosing angle between the arms. An analytical explanation for enclosing angles with negligible interaction of spin states in the ring and the arms in terms of pure spin states of the isolated ring is presented.\par
The numerical code can be used to study devices within non-interacting two-dimensional electron systems of complex geometry. When treating multiple leads as one virtual lead even \emph{multi terminal devices} like \emph{spin-filter cascades} \cite{jacob:093714} and \textsc{Hall} bars \cite{Wunderlich24122010} can be investigated. The addition of a magnetic field via a \textsc{Peierls} phase allows the investigation of \emph{spin hall effects} and \emph{spin field effect transistors}. The generality of the applied concepts permits the use of different bases for discretization, e.g. the simulation of \emph{graphene} on a hexagonal lattice. Conceivable future uses lie in the extension of the code to enable the simulation of interfaces in \emph{hybrid structures} to allow the study of \emph{spin injection} through semiconductor metal interfaces \cite{holz:431}.

\appendix
\chapter{Derivation of RGA}
\label{app:RGA}
The outline of the derivation of the recursive algorithms for the retarded and the lesser \gfnc{} will be presented. The discussion closely follows \cite{JApplPhys.91.2343}, \cite{JApplPhys.81.7845} and \cite{Wimmer2009Thesis} which obtain equivalent results with differences in notation.
\section{Recursive Algorithm for the retarded \cgfnc{}}
Only the retarded \gfnc{} of the conductor will be discussed in this section, hence, all superscripts and subscripts will be dropped, i.e. $\mat{G} \equiv \mat{G}^r\equiv \mat{G}_C$. The recursive algorithm to calculate all diagonal and off-diagonal blocks of the retarded \gfnc{} can be obtained by starting from \cref{eqn:amatrix}
\begin{align}
  (E\mathds{1}-\mat{H})\mat{G}_C = \mat{A}\mat{G}=\mathds{1}\ .
  \label{eqn:amatrixppendix}
\end{align}
The solution to the partitioning
\begin{align}
  \begin{pmatrix} \mat{A}_{Z,Z} & \mat{A}_{Z,Z'}\\
		  \mat{A}_{Z',Z} & \mat{A}_{Z',Z'}
  \end{pmatrix}
  \begin{pmatrix} \mat{G}_{Z,Z} & \mat{G}_{Z,Z'}\\
		  \mat{G}_{Z',Z} & \mat{G}_{Z',Z'}
  \end{pmatrix} = 
  \begin{pmatrix} \mathds{1} & \mat{0}\\
		  \mat{0} & \mathds{1}
  \end{pmatrix}
\end{align}
is
\begin{align}
  \mat{G}=\mat{G}_0+\mat{G}_0\mat{U}\mat{G}
  \label{eqn:dysonequationappendix}
\end{align}
where 
\begin{align}
  \mat{G} = 
  \begin{pmatrix} \mat{G}_{Z,Z} & \mat{G}_{Z,Z'}\\
		  \mat{G}_{Z',Z} & \mat{G}_{Z',Z'}
  \end{pmatrix}\ ,\quad
  \mat{G}_0 &= 
  \begin{pmatrix} \mat{G}_{0 Z,Z} & \mat{0}\\
		  \mat{0} & \mat{G}_{0 Z',Z'}
  \end{pmatrix}=
  \begin{pmatrix} \mat{A}^{-1}_{Z,Z} & \mat{0}\\
		  \mat{0} & \mat{A}^{-1}_{Z',Z'}
  \end{pmatrix}\ ,\notag\\[3mm] 
  \text{and}\hspace{2em}
  \mat{U} &= 
  \begin{pmatrix} \mat{0} & -\mat{A}_{Z,Z'}\\
		  -\mat{A}_{Z',Z} & \mat{0}
  \end{pmatrix}\ .
\end{align}
The index $Z$ denotes a range of blocks, i.e. $Z = 1:i-1$ and $Z'=i:N$. Where $N$ is the maximum number of blocks and $i$ denotes the individual block. The $\mat{A}_{Z,Z}$ are submatrices of $\mat{A}$. \Cref{eqn:dysonequationappendix} is the \textsc{Dyson} equation for the retarded \gfnc{}. The matrix $\mat{G}_{0 Z,Z}$ represents the \gfnc{} of the system of size $Z$ and $\mat{G}_{0 Z',Z'}$ denotes the \gfnc{} for the isolated blocks $Z'$.
The left connected \gfnc{} $\mat{G}^i$ is defined by the first $i$ blocks of the \cref{eqn:amatrixppendix} 
\begin{align}
\mat{A}_{1:i,1:i}\mat{G}^{i}=\mathds{1}_{1:i,1:i}\ .
\end{align}
Here $\mat{G}^i$ represents the \gfnc{} of the system up to block $i$.
If inserted into the \textsc{Dyson} equation, \cref{eqn:dysonequationappendix}, the recursive relation for the diagonal elements of the left connected \gfnc{} reads
\begin{align}
  \mat{G}^{i}_{i,i} = \left[\mat{A}_{i,i}-\mat{A}_{i,i-1}\mat{G}^{i-1}_{i-1,i-1}\mat{A}_{i-1,i}\right]^{-1}\ .
  \label{eqn:recrel1}
\end{align}
In a similar calculation, using the fact that the only nonzero element of $\mat{A}_{1:i,i+1:N}$ is $\mat{A}_{i,i+1}$ all other relations necessary for the algorithm are obtained
\begin{align}
  \mat{G}_{i,i-1} &= -\mat{G}_{i,i}\mat{A}_{i,i-1}\mat{G}^{i-1}_{i-1,i-1}\label{eqn:recrel2}\\\addlinespace
  \mat{G}_{i-1,i} &= -\mat{G}^{,i-1}_{i-1,i-1}\mat{A}_{i-1,i} \mat{G}_{i,i}\label{eqn:recrel3}\\\addlinespace
  \mat{G}_{i-1,i-1} &= \mat{G}^{i}_{i-1,i-1}-\mat{G}^{i-1}_{i-1,i-1}\mat{A}_{i-1,i}\mat{G}_{i,i-1}\label{eqn:recrel4}\ .
\end{align}
After one iteration from $i=0$ to $N$, the last element calculated by \cref{eqn:recrel1} is already the last element of the final \gfnc{}. Using this element as a starting point for a backward iteration using \cref{eqn:recrel2,eqn:recrel3,eqn:recrel4} the full retarded \gfnc{} $\mat{G}$ can be obtained. The off-diagonal elements $\mat{G}_{i+1,i}$ need to be saved for the computation of the lesser \gfnc{} $\mat{G}^<$.
\section{Recursive Algorithm for the lesser \cgfnc{}}
Starting from the \textsc{Keldysh} equation (\cref{eqn:keldyshequation}), the equation for the lesser \gfnc{} is written as
\begin{align}
\mat{A}\mat{G}^< = \mat{\Sigma}^< \mat{G}^a\ .
\end{align}
Here $\mat{G}^<$ is the lesser \gfnc{}, and $\mat{\Sigma}^<$ and $\mat{G}^a$ are the lesser self energy (\cref{eqn:lesserselfenergy}) and the advanced \gfnc{} (\cref{eqn:retardedandadvancedgfnc}), respectively. The \textsc{Dyson} equation for $\mat{G}^<$ is the solution to
\begin{align}
  \begin{pmatrix} \mat{A}_{Z,Z} & \mat{A}_{Z,Z'}\\
		  \mat{A}_{Z',Z} & \mat{A}_{Z',Z'}
  \end{pmatrix}
  \begin{pmatrix} \mat{G}^<_{Z,Z} & \mat{G}^<_{Z,Z'}\\
		  \mat{G}^<_{Z',Z} & \mat{G}^<_{Z',Z'}
  \end{pmatrix} = 
  \begin{pmatrix} \mat{\Sigma}^<_{Z,Z} & \mat{\Sigma}^<_{Z,Z'}\\
		  \mat{\Sigma}^<_{Z',Z} & \mat{\Sigma}^<_{Z',Z'}
  \end{pmatrix}
  \begin{pmatrix} \mat{G}^a_{Z,Z} & \mat{G}^a_{Z,Z'}\\
		  \mat{G}^a_{Z',Z} & \mat{G}^a_{Z',Z'}
  \end{pmatrix}\ .
\end{align}
and reads
\begin{align}
\mat{G}^< = \mat{G}^0\mat{U}\mat{G}^< + \mat{G}^0\mat{\Sigma}^<\mat{G}^a\ .
\end{align}
The recursive algorithm for the lesser \gfnc{} follows the same approach as the algorithm for the retarded \gfnc{}, thus the left-connected lesser \gfnc{} $\mat{G}^{<,i}$ is defined by
\begin{align}
\mat{A}_{1:i,1:i}\mat{G}^{<,i}=\mat{\Sigma}^<_{1:i,1:i}\mat{G}^a_{1:i,1:i}\ .
\end{align}
Again, by setting $Z=1:i$ and $Z'=i+1$ and using the \textsc{Dyson} equation for $\mat{G}$ and $\mat{G}^<$ the recursive relations for the left-connected lesser \gfnc{} is obtained
\begin{align}
\mat{G}^{<,i}_{i,i} = \mat{G}^{i}_{i,i}( \mat{\Sigma}^<_{i,i}+\mat{\sigma}^<_{i})\mat{G}^{a,i}_{i,i}
\end{align}
with $\mat{\sigma}^<_i=\mat{A}_i\mat{G}^{<,i-1}_{i-1,i-1}\mat{A}^{\dagger}_i$. The recursive relations for the diagonal and off-diagonal elements of the full lesser \gfnc{} can be obtained by noting again that the only nonzero element of $\mat{A}_{1:i,i+1:N}$ is $\mat{A}_{i,i+1}$. The off-diagonal elements can be calculated by
\begin{align}
\mat{G}^<_{i,i-1} = \mat{G}_{i,i}\mat{A}_{i,i-1}\mat{G}^{<,i-1}_{i-1,i-1} +\mat{G}^<_{i,i}\mat{A}^{\dagger}_{i,i-1}\mat{G}^{a,i-1}_{i-1,i-1}\ .
\end{align}
Using the \textsc{Dyson} equation for the lesser \gfnc{} again, the expression for the diagonal elements is found to be
\begin{align}
\mat{G}^<_{i-1,i-1} = \mat{G}^{<,i-1}_{i-1,i-1} &+\mat{G}^{i-1}_{i-1,i-1}(\mat{A}_{i-1,i}\mat{G}^{<}_{i,i}\mat{A}^{\dagger}_{i,i-1})\mat{G}^{a,i-1}_{i-1,i-1}\notag \\
&+ \mat{G}^{<,i-1}_{i-1,i-1}\mat{A}^{\dagger}_{i-1,i}\mat{G}^a_{i,i-1}\notag \\
&+\mat{G}_{i-1,i}\mat{A}_{i,i-1}\mat{G}^{<,i-1}_{i-1,i-1}\ .
\end{align}
For further details and a deeper physical discussion of the individual terms see \cite{JApplPhys.91.2343}.

\clearpage
\pagestyle{bib}
\bibliographystyle{amsalphaabbrv}
\bibliography{thesis}
\clearpage
\chapter*{Danksagungen}
An dieser Stelle ....

\makeatletter\@openrightfalse
\chapter*{Erkl\"arung}
Hiermit versichere ich, diese Diplomarbeit selbstst\"andig und ausschlie\ss lich mit den angegebenen Quellen und Hilfsmitteln angefertigt zu haben. Mit der Ver\"offentlichung meiner Diplomarbeit in der Bibliothek bin ich einverstanden.\par
\vspace{3em}
\noindent Hamburg, den 17. Januar 2012\par
\vspace{6em}
\noindent \line(1,0){150}\par
\noindent Jonas Siegl
\@openrighttrue\makeatother

\end{document}
