\chapter{Arbitrary Angle}\label{app:rings}
\begin{figure}[h!]
\subfloat[]{\label{fig:ring15edens}\includegraphics[]{images/ring3edens}}\hspace{-0.5em}
\subfloat[]{\label{fig:ring25edens}\includegraphics[]{images/ring5edens}}\hspace{-0.5em}
\subfloat[]{\label{fig:ring30edens}\includegraphics[]{images/ring6edens}}
\caption{Electron densities in \textsc{Aharonov-Bohm} type ring for three different enclosing angles. (a) Enclosing angle $\alpha\approx 1/12 \pi$. (b) Enclosing angle $\alpha\approx 1/7\pi$. (c) Enclosing angle $\alpha=1/6\pi$. Color code denotes electron density $n$.} 
\end{figure}
\begin{figure}[h!]
\subfloat[]{\label{fig:ring15spindens}\includegraphics[]{images/ring3spindens}}\hspace{-0.5em}
\subfloat[]{\label{fig:ring25spindens}\includegraphics[]{images/ring5spindens}}\hspace{-0.5em}
\subfloat[]{\label{fig:ringneg30spindens}\includegraphics[]{images/ring6spindens}}
\caption{Spin densities in \textsc{Aharonov-Bohm} type ring for three different enclosing angles. (a) Enclosing angle $\alpha\approx 1/12 \pi$. (b) Enclosing angle $\alpha=1/7\pi$. (c) Spin density $s_z$. Enclosing angle $\alpha=1/6\pi$ . Color code denotes the spin density $s_z$. Note the dependence of the maximum spin split and spin-up to spin-down ratio on enclosed angle.}
\end{figure}
