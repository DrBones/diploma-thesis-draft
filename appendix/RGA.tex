\chapter{RGA}
\label{app:RGA}
The derivation of the recursive \gfnc{} algorithm will be presented. The discussion follows \cite{JApplPhys.91.2343}, \cite{JApplPhys.81.7845} and \cite{Wimmer2009Thesis} which obtain equivalent results with differences in notation.
\section{Recursive Algorithm for the retarded \gfnc{}}
Only the retarded \gfnc{} of the conductor will be discussed in this section, hence, all superscripts and subscripts will be dropped, i.e. $\mat{G} \equiv \mat{G}^r\equiv \mat{G}_C$. The recursive algorithm to calculate all diagonal and off-diagonal blocks of the retarded \gfnc{} can be obtained by starting from \cref{eqn:amatrix}
\begin{align}
  (E\mathds{1}-\mat{H})\mat{G}_C = \mat{A}\mat{G}=\mathds{1}\ .
  \label{eqn:amatrixppendix}
\end{align}
The solution to the partitioning
\begin{align}
  \begin{pmatrix} \mat{A}_{Z,Z} & \mat{A}_{Z,Z'}\\
		  \mat{A}_{Z',Z} & \mat{A}_{Z',Z'}
  \end{pmatrix}
  \begin{pmatrix} \mat{G}_{Z,Z} & \mat{G}_{Z,Z'}\\
		  \mat{G}_{Z',Z} & \mat{G}_{Z',Z'}
  \end{pmatrix} = 
  \begin{pmatrix} \mathds{1} & \mat{0}\\
		  \mat{0} & \mathds{1}
  \end{pmatrix}
\end{align}
is
\begin{align}
  \mat{G}=\mat{G}_0+\mat{G}_0\mat{U}\mat{G}
  \label{eqn:dysonequationappendix}
\end{align}
where 
\begin{align}
  \mat{G} = 
  \begin{pmatrix} \mat{G}_{Z,Z} & \mat{G}_{Z,Z'}\\
		  \mat{G}_{Z',Z} & \mat{G}_{Z',Z'}
  \end{pmatrix}\ ,\quad
  \mat{G}_0 &= 
  \begin{pmatrix} \mat{G}_{0 Z,Z} & \mat{0}\\
		  \mat{0} & \mat{G}_{0 Z',Z'}
  \end{pmatrix}=
  \begin{pmatrix} \mat{A}^{-1}_{Z,Z} & \mat{0}\\
		  \mat{0} & \mat{A}^{-1}_{Z',Z'}
  \end{pmatrix}\ ,\notag\\[3mm] 
  \text{and}\hspace{2em}
  \mat{U} &= 
  \begin{pmatrix} \mat{0} & -\mat{A}_{Z,Z'}\\
		  -\mat{A}_{Z',Z} & \mat{0}
  \end{pmatrix}\ .
\end{align}
The index $Z$ denotes a range of blocks, i.e. $Z = 1:i-1$ and $Z'=i:N$. Where $N$ is the maximum number of blocks and $i$ denotes the individual block. The $\mat{A}_{Z,Z}$ are submatrices of $\mat{A}$. \Cref{eqn:dysonequationappendix} is the \textsc{Dyson} equation for the retarded \gfnc{}. The matrix $\mat{G}_{0 Z,Z}$ represents the \gfnc{} of the system of size $Z$ and $\mat{G}_{0 Z',Z'}$ denotes the \gfnc{} for the isolated blocks $Z'$.
The left connected \gfnc{} $\mat{G}^i$ is defined by the first $i$ blocks of the \cref{eqn:amatrixppendix} 
\begin{align}
\mat{A}_{1:i,1:i}\mat{G}^{i}=\mathds{1}_{1:i,1:i}\ .
\end{align}
Here $\mat{G}^i$ represents the \gfnc{} of the system up to block $i$.
If inserted into the \textsc{Dyson} equation, \cref{eqn:dysonequationappendix}, the recursive relation for the diagonal elements of the left connected \gfnc{} reads
\begin{align}
  \mat{G}^{i}_{i,i} = \left[\mat{A}_{i,i}-\mat{A}_{i,i-1}\mat{G}^{i-1}_{i-1,i-1}\mat{A}_{i-1,i}\right]^{-1}\ .
  \label{eqn:recrel1}
\end{align}
In a similar calculation, using the fact that the only nonzero element of $\mat{A}_{1:i,i+1:N}$ is $\mat{A}_{i,i+1}$ all other relations necessary for the algorithm are obtained
\begin{align}
  \mat{G}_{i,i-1} &= -\mat{G}_{i,i}\mat{A}_{i,i-1}\mat{G}^{i-1}_{i-1,i-1}\label{eqn:recrel2}\\\addlinespace
  \mat{G}_{i-1,i} &= -\mat{G}^{,i-1}_{i-1,i-1}\mat{A}_{i-1,i} \mat{G}_{i,i}\label{eqn:recrel3}\\\addlinespace
  \mat{G}_{i-1,i-1} &= \mat{G}^{i}_{i-1,i-1}-\mat{G}^{i-1}_{i-1,i-1}\mat{A}_{i-1,i}\mat{G}_{i,i-1}\label{eqn:recrel4}\ .
\end{align}
After one iteration from $i=0$ to $N$, the last element calculated by \cref{eqn:recrel1} is already the last element of the final \gfnc{}. Using this element as a starting point for a backward iteration using \cref{eqn:recrel2,eqn:recrel3,eqn:recrel4} the full retarded \gfnc{} $\mat{G}$ can be obtained. The off-diagonal elements $\mat{G}_{i+1,i}$ need to be saved for the computation of the lesser \gfnc{} $\mat{G}^<$.
\section{Recursive Algorithm for the lesser \gfnc{}}
Starting from the \textsc{Keldysh} equation (\cref{eqn:keldyshequation}), the equation for the lesser \gfnc{} is written as
\begin{align}
\mat{A}\mat{G}^< = \mat{\Sigma}^< \mat{G}^a\ .
\end{align}
Here $\mat{G}^<$ is the lesser \gfnc{}, and $\mat{\Sigma}^<$ and $\mat{G}^a$ are the lesser self energy (\cref{eqn:lesserselfenergy}) and the advanced \gfnc{} (\cref{eqn:retardedandadvancedgfnc}), respectively. The \textsc{Dyson} equation for $\mat{G}^<$ is the solution to
\begin{align}
  \begin{pmatrix} \mat{A}_{Z,Z} & \mat{A}_{Z,Z'}\\
		  \mat{A}_{Z',Z} & \mat{A}_{Z',Z'}
  \end{pmatrix}
  \begin{pmatrix} \mat{G}^<_{Z,Z} & \mat{G}^<_{Z,Z'}\\
		  \mat{G}^<_{Z',Z} & \mat{G}^<_{Z',Z'}
  \end{pmatrix} = 
  \begin{pmatrix} \mat{\Sigma}^<_{Z,Z} & \mat{\Sigma}^<_{Z,Z'}\\
		  \mat{\Sigma}^<_{Z',Z} & \mat{\Sigma}^<_{Z',Z'}
  \end{pmatrix}
  \begin{pmatrix} \mat{G}^a_{Z,Z} & \mat{G}^a_{Z,Z'}\\
		  \mat{G}^a_{Z',Z} & \mat{G}^a_{Z',Z'}
  \end{pmatrix}\ .
\end{align}
and reads
\begin{align}
\mat{G}^< = \mat{G}^0\mat{U}\mat{G}^< + \mat{G}^0\mat{\Sigma}^<\mat{G}^a\ .
\end{align}
The recursive algorithm for the lesser \gfnc{} follows the same approach as the algorithm for the retarded \gfnc{}, thus the left-connected lesser \gfnc{} $\mat{G}^{<,i}$ is defined by
\begin{align}
\mat{A}_{1:i,1:i}\mat{G}^{<,i}=\mat{\Sigma}^<_{1:i,1:i}\mat{G}^a_{1:i,1:i}\ .
\end{align}
Again, by setting $Z=1:i$ and $Z'=i+1$ and using the \textsc{Dyson} equation for $\mat{G}$ and $\mat{G}^<$ the recursive relations for the left-connected lesser \gfnc{} is obtained
\begin{align}
\mat{G}^{<,i}_{i,i} = \mat{G}^{i}_{i,i}( \mat{\Sigma}^<_{i,i}+\mat{\sigma}^<_{i})\mat{G}^{a,i}_{i,i}
\end{align}
with $\mat{\sigma}^<_i=\mat{A}_i\mat{G}^{<,i-1}_{i-1,i-1}\mat{A}^{\dagger}_i$. The recursive relations for the diagonal and off-diagonal elements of the full lesser \gfnc{} can be obtained by noting again that the only nonzero element of $\mat{A}_{1:i,i+1:N}$ is $\mat{A}_{i,i+1}$. The off-diagonal elements can be calculated by
\begin{align}
\mat{G}^<_{i,i-1} = \mat{G}_{i,i}\mat{A}_{i,i-1}\mat{G}^{<,i-1}_{i-1,i-1} +\mat{G}^<_{i,i}\mat{A}^{\dagger}_{i,i-1}\mat{G}^{a,i-1}_{i-1,i-1}\ .
\end{align}
Using the \textsc{Dyson} equation for the lesser \gfnc{} again, the expression for the diagonal elements is found to be
\begin{align}
\mat{G}^<_{i-1,i-1} = \mat{G}^{<,i-1}_{i-1,i-1} &+\mat{G}^{i-1}_{i-1,i-1}(\mat{A}_{i-1,i}\mat{G}^{<}_{i,i}\mat{A}^{\dagger}_{i,i-1})\mat{G}^{a,i-1}_{i-1,i-1}\notag \\
&+ \mat{G}^{<,i-1}_{i-1,i-1}\mat{A}^{\dagger}_{i-1,i}\mat{G}^a_{i,i-1}\notag \\
&+\mat{G}_{i-1,i}\mat{A}_{i,i-1}\mat{G}^{<,i-1}_{i-1,i-1}\ .
\end{align}
For further details and a deeper physical discussion of the individual terms see \cite{JApplPhys.91.2343}.
