In this thesis the non-equilibrium \gfnc{} formalism is implemented to investigate the electron density, spin density and conductance of two-dimensional electron systems.\par
In chapter 1 the \gfnc{} formalism for the quantum mechanical discription of two dimensional electron gases is presented.\par
In chapter 2 the discretized \hamil{} in the finite differences approximation and the reduction to a finite system of equations via the self-energy is discussed.\par
In chapter 3 the implementation of a simulator based on the recursive \gfnc{} algorithm is presented. A graph based matrix reordering algorithm is described allowing the application of the recursive \gfnc{} algorithm to arbitrary geometry.
Furthermore the computed electron density is compared to analytical results and the spin density and conductance are compared to experimental data.\par
In chapter 4 the results of spin-dependent transport simulations for different colinear quantum point contacts and a non-colinear \textsc{Aharonov-Bohm} interferometer are examined. The influence of the potential landscape which is induced by the quantum point contact on the 2DEG is analyzed.
The electron density is found to mirror the geometrical constriction closely. An interference of spin polarization normal to the 2DEG can be observed. The conductance profile shows dependence on the spatial gradient of the potential landscape. Abrupt potential changes show well defined quantization while more gradual changes in potential smooth out the discrete steps in the conductance.
The spin density in an \textsc{Aharonov-Bohm} interferometer with two arms exhibits strong dependence on the enclosing angle between the arms. An analytical explanation for enclosing angles with negligible interaction of spin states in the ring and the arms in terms of pure spin states of the isolated ring is presented.\par
The numerical code can be used to study devices within non-interacting two-dimensional electron systems of complex geometry. When treating multiple leads as one virtual lead even \emph{multi terminal devices} like \emph{spin-filter cascades} \cite{jacob:093714} and \textsc{Hall} bars \cite{Wunderlich24122010} can be investigated. The addition of a magnetic field via a \textsc{Peierls} phase allows the investigation of \emph{spin hall effects} and \emph{spin field effect transistors}. The generality of the applied concepts permits the use of different bases for discretization, e.g. the simulation of \emph{graphene} on a hexagonal lattice. Conceivable future uses lie in the extension of the code to enable the simulation of interfaces in \emph{hybrid structures} to allow the study of \emph{spin injection} through semiconductor metal interfaces \cite{holz:431}.
