In der Nano-Spintronik steht die Manipulation und Kontrolle von Electronen und ihres Spins im Mittelpunkt.\par
Mit der Endeckung des \emph{GMR-Effekts} durch \textsc{Albert Fert} und \textsc{Peter Gr\"unberg} in 1988 wurde die Manipulation des Elektronentransports durch die Nutzung des Spinfreiheitsgrads offenkundig. Dieser wissenschaftliche Durchbruch, f\"ur den der \textsc{Nobel}preis in 2007 verliehen wurde, markiert den Anfang des Studiums und der Manipulation des Spins in elektronischen Bauelementen~\cite{evegeny2010spin}. Der \textsc{Datta-Das} \emph{Spin-Feldeffekttransistor} geh\"ort zu den bekanntesten zu realisiernenden Bauelementen der Spintronik~\cite{datta:665}. Es stellte sich heraus, dass die Leitf\"ahigkeit eines solchen mesoskopischen Bauelements in starken magnetischen Feldern unter niedrigen Temperaturen quantisiert ist \cite{PhysRevLett.45.494}. Hierdurch wurde der quantenmechanischen Ursprung der Leitf\"ahigkeit deutlich. F\"ur endliche Temperaturen wurde der Spin-Feldeffekttransistor erst k\"urzlich experimentell umgesetzt~\cite{Wunderlich24122010}.\par
Der zentrale Bestandteil des Spin-Feldeffekttransistors ist das \emph{zweidimensionale Elektronengas} (2DEG). Das 2DEG beschreibt ein Sysem nicht wechselwirkender Elektronen, die durch ein stark lokalisiertes Potential in ihrer Bewegung auf einen zweidimensionalen Raumbereich beschr\"ankt sind. Die Ladungstr\"agerdichte und der Elektronenspin im 2DEG k\"onnen durch Quantenpunktkontakte \"uber die Effekte des Quanten-Confinements beeinflusst werden. Quantenpunktkontakte sind sowohl Gegenstand aktiver Grundlagenforschung als auch von Bedeutung in experimentellen Umsetzungen von Detektoren~\cite{PhysRevB.67.161308} und Quantenbits~\cite{PhysRevA.57.120}.\par
Bestandteil dieser Arbeit ist die Entwicklung eines quantenmechanischen Simulators auf Basis des Formalismus der \textsc{Green}schen Funktionen im Nichtgleichgewicht. Die numerischen Berechnungen werden durch die Anwendung des \emph{rekursiven \textsc{Green}sche Funktionen Algorithmus} beschleunigt. Desweiteren erm\"oglicht die Anwendung einer graphbasierten Methode zur Matrix Umordnung die Simulation von Quantenpunktkontakten beliebiger Geometrie. Die Elektronendichte $n$, die Spindichte $s_z$, und die Leitf\"ahigkeit $G$ werden sowohl f\"ur Quantenpunktkontakte unterschiedlicher Geometrie als auch f\"ur 
