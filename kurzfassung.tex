In der Nano-Spintronik steht die Manipulation und Kontrolle von Elektronen und ihres Spins im Mittelpunkt.\par
Mit der Entdeckung des \emph{GMR-Effekts} durch \textsc{Albert Fert} und \textsc{Peter Gr\"unberg} im Jahr 1988 wurde die Manipulation des Elektronentransports durch die Nutzung des Spinfreiheitsgrades offenkundig. Dieser wissenschaftliche Durchbruch, f\"ur den der \textsc{Nobel}preis f\"ur Physik des Jahres 2007 verliehen wurde, markiert den Anfang des Studiums und der Manipulation des Spins in elektronischen Bauelementen~\cite{evegeny2010spin}. Der \textsc{Datta-Das} \emph{Spin-Feldeffekttransistor} geh\"ort zu den bekanntesten Bauelementen der Spintronik die es zu realisieren gilt~\cite{datta:665}. Es stellte sich heraus, dass die Leitf\"ahigkeit eines solchen mesoskopischen Bauelements in starken magnetischen Feldern und unter niedrigen Temperaturen quantisiert ist \cite{PhysRevLett.45.494}. Hierdurch wurde der quantenmechanische Ursprung der Leitf\"ahigkeit deutlich. F\"ur endliche Temperaturen wurde der Spin-Feldeffekttransistor erst k\"urzlich experimentell umgesetzt~\cite{Wunderlich24122010}.\par
Der zentrale Bestandteil des Spin-Feldeffekttransistors ist das \emph{zweidimensionale Elektronengas} (2DEG). Das 2DEG beschreibt ein System nicht wechselwirkender Elektronen, die durch ein stark lokalisiertes Potential in ihrer Bewegung auf einen zweidimensionalen Raumbereich beschr\"ankt sind. Die Ladungstr\"agerdichte und der Elektronenspin im 2DEG k\"onnen durch Quantenpunktkontakte \"uber die Effekte des Quanten-Confinements beeinflusst werden. Quantenpunktkontakte sind Gegenstand aktiver Grund\-la\-gen\-for\-schung und dar\"uber hinaus von Bedeutung in der experimentellen Umsetzung von Detektoren~\cite{PhysRevB.67.161308} und Quantenbits~\cite{PhysRevA.57.120}.\par
Bestandteil dieser Arbeit ist die Entwicklung eines quantenmechanischen Simulators auf Basis des Formalismus der \textsc{Green}schen Funktionen im Nichtgleichgewicht. Die numerischen Berechnungen werden durch die Anwendung des \emph{rekursiven \textsc{Green}sche Funktionen Algorithmus} beschleunigt. Desweiteren erm\"oglicht die Anwendung einer graphbasierten Methode zur Matrix Umordnung die Simulation von 2DEGs beliebiger Geometrie. Die Elektronendichte $n$, die Spindichte $s_z$ und die Leitf\"ahigkeit $G$ werden sowohl f\"ur Quantenpunktkontakte unterschiedlicher Geometrie als auch f\"ur nicht kollineare Systeme mit gegeneinander rotierten Zuleitungen berechnet. Aus den Ergebnissen der Simulation ist ersichtlich, dass durch \"Anderungen in der Geometrie Einfluss auf die lokalen Elektronen- und Spindichten genommen werden kann. Au\ss erdem hat die Form des Quantenpunktkontaktes erheblichen Einfluss auf die Quantisierung des Leitf\"ahigkeitprofils. In dem untersuchten \textsc{Aharonov-Bohm} Interferometer ist die Spinausrichtung der Elektronen im Ring durch die Position der Zuleitungen gegeben. Der Zugang zu mikroskopischen Gr\"o\ss en durch den Simulator erm\"oglicht die Erforschung von Methoden zur Spinmanipulation in zweidimensionalen Elektronengasen.
