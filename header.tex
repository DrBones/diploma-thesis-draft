% ***********************************************************
% ******************* PHYSICS HEADER ************************
% ***********************************************************
%\documentclass[11pt]{article} 
% \documentclass[12pt,twoside,a4paper]{article}
% \documentclass[12pt,twoside,a4paper]{scrbook}
\documentclass[12pt,twoside,a4paper]{scrbook}
% \setkomafont{sectioning}{\normalcolor\bfseries}
%%%%%%%%%%%%%% PACKAGE INCLUDES %%%%%%%%%%%%%%%
% \usepackage{scrhack}
\usepackage[utf8]{inputenc} %allows input of utf8 characters
\usepackage[american,ngerman]{babel}
\usepackage[T1]{fontenc} %more complete but lower quality than ComMod fontpacke
% \usepackage{ae,aecompl}
\usepackage[tbtags]{mathtools}
%\usepackage{amsmath} % AMS Math Package
\usepackage{amsthm} % Theorem Formatting
\usepackage{amssymb}	% Math symbols such as \mathbb
\usepackage{graphicx} % Allows for eps images
\usepackage{multirow}
\usepackage{multicol} % Allows for multiple columns
\usepackage{color} % well, supplies colors
\usepackage{todonotes}
\usepackage{booktabs}
\usepackage{url}
\usepackage{grffile}
\usepackage{dsfont}%supplies 1 and lowercase k in blackboard bold
\usepackage{tikz}
\newcommand{\axes}{\draw[style={<->,thin}] (-1,0.65) node[left,align=center] {$x$} -- (0,0) --  (1,0.65) node[right] {$y$};}
\newcommand{\curveoutline}{
\draw[lightgray] (-0.85,6) .. controls (-0.9,1.75) and (0.7,1.7) .. (1.23,1.35);
\draw[lightgray] (0.18,6) .. controls(0.25,3.3) and (0.58,2.8) .. (1.23,2.35);}
\newcommand{\curveoutlinedark}{
\draw[lightgray] (-0.8,6) .. controls (-0.9,1.75) and (0.7,1.7) .. (1.23,1.35);
\draw[lightgray] (0.18,6) .. controls(0.25,3.3) and (0.58,2.8) .. (1.23,2.35);}
\usetikzlibrary{matrix,decorations.pathreplacing,shapes.multipart,calc}
\usepackage[section]{placeins} %supplies \FloatBarrier to stop floats
\usepackage{setspace} %Zeilenabstand
\usepackage[pagestyles,raggedright]{titlesec}
\usepackage{float}
%--------------------------------------
% Loosen latex float to text ratio and other float related things. Explanation
%on pages 199–200, section C.9 of the LaTeX manual.
\renewcommand{\topfraction}{.85}
\renewcommand{\bottomfraction}{.7}
\renewcommand{\textfraction}{.15}
\renewcommand{\floatpagefraction}{.66}
\renewcommand{\dbltopfraction}{.66}
\renewcommand{\dblfloatpagefraction}{.66}
\setcounter{topnumber}{9}
\setcounter{bottomnumber}{9}
\setcounter{totalnumber}{20}
\setcounter{dbltopnumber}{9}
%----------------------------------------
% %define custom float
% \floatstyle{plain}
% \newfloat{algo}{thp}{lop}[chapter]
% \floatname{algo}{Algorithm}

% Page setup (titlesec)
\newpagestyle{main}{%
  \headrule
  \sethead[\chaptername~\thechapter\quad\chaptertitle][][]{}{}{\thesection\quad\sectiontitle}
  % \footrule
\renewcommand{\makefootrule}{%
  % \makebox[0pt][l]{\rule[.7\baselineskip]{\linewidth}{0.8pt}}%
  % \color{red}%
  \rule[1.3\baselineskip]{\linewidth}{0.4pt}}
  \setfoot[][\thepage][]{}{\thepage}{}
}
% \usepackage{calligra}
\DeclareMathAlphabet{\mathpzc}{OT1}{pzc}{m}{it}
% \DeclareMathAlphabet{\mathcalligra}{T1}{calligra}{m}{n}
% \DeclareFontShape{T1}{calligra}{m}{n}{<->s*[2.2]callig15}{}
\newenvironment{pzcseries}{\fontfamily{pzc}\selectfont}{}
\newcommand{\textpzc}[1]{{\pzcseries#1}}
\newenvironment{frcseries}{\fontfamily{frc}\selectfont}{}
\newcommand{\textfrc}[1]{{\frcseries#1}}
\newcommand{\mathfrc}[1]{\text{\textfrc{#1}}}
\assignpagestyle{\chapter}{empty} %pages on which chapters start have now {} layout
%\onehalfspacing
% \usepackage{mathptmx}
% \usepackage{bbold}
% \usepackage{bold-extra} % Ugly bitmap fonts, but holds bold glyphs!
% \usepackage{kpfonts} % Pretty complete fontpackage but looks 'different'
\usepackage[
    pdftitle={Spin-dependend transport simulations in two-dimensionsal electron systems},    % title
    pdfauthor={Jonas Siegl},     % author
    pdfsubject={Quantum Transport},   % subject of the document
    pdfcreator={Jonas Siegl},   % creator of the document
    pdfproducer={Jonas Siegl}, % producer of the document
    pdfkeywords={Green's Function } {Numerical } {Transport}, % list of keywords
    % linktocpage=true,
    linkcolor=red,          % color of internal links
    citecolor=green,        % color of links to bibliography
    filecolor=magenta,      % color of file links
    urlcolor=cyan           % color of external links
    % pagebackref=true
    ]{hyperref}
\usepackage[all]{hypcap} % shifts the link point to the top of tables or figures so it is visible when clicked
\usepackage{array}
%\usepackage[final]{svninfo}
\usepackage{tabulary}
\usepackage{tabularx}
%\usepackage{ifthen}
%\usepackage[dvips]{graphicx}
%\usepackage[small,bf]{caption}
\usepackage[margin=10pt]{subfig} 		%enabled the use of subfloats within a float
\usepackage[capitalize]{cleveref}
\crefname{table}{Tab.}{Tabs.}
\crefname{section}{Sec.}{Secs.}
\crefname{algo}{Alg.}{Algs.}
\crefname{equation}{Eqn.}{Eqns.}
\crefformat{equation}{Eqn.~#2#1#3}
\crefrangeformat{equation}{Eqns.~#3#1#4 to~#5#2#6}
\Crefformat{equation}{Equation~#2#1#3}
\Crefrangeformat{equation}{Equations.~#3#1#4 to~#5#2#6}
%\usepackage{textcomp}
%\usepackage{pstricks}
% % Fix latex
% \def\smallskip{\vskip\smallskipamount}
% \def\medskip{\vskip\medskipamount}
% \def\bigskip{\vskip\bigskipamount}
%  
%  % Hand made theorem
%  \newcounter{thm}
%  \renewcommand{\thethm}{\thechapter.\arabic{thm}}
%  \def\claim#1{\par\medskip\noindent\refstepcounter{thm}\hbox{\bf #1 \arabic{thm}.}
%  %\ignorespaces
%  }
%  \def\endclaim{
%  \par\medskip}
%  \newenvironment{thm}{\claim}{\endclaim}
%\usepackage{courier}
%\usepackage{geometry}
\usepackage{caption}
% \usepackage[algo2e,algoruled,algochapter]{algorithm2e}
%\usepackage{url}
\usepackage{tikz}
\tikzstyle{every picture}+=[remember picture]
\usetikzlibrary{decorations.pathreplacing,decorations.markings,circuits.ee.IEC}
%\usepackage{paralist}
%\usepackage{xspace}
%\usepackage[dvips,letterpaper,margin=0.75in,bottom=0.5in]{geometry}
 % Sets margins and page size
%\pagestyle{empty} % Removes page numbers
%\makeatletter % Need for anything that contains an @ command 
%\renewcommand{\maketitle} % Redefine maketitle to conserve space
%{ \begingroup \vskip 10pt \begin{center} \large {\bf \@title}
%	\vskip 10pt \large \@author \hskip 20pt \@date \end{center}
%  \vskip 10pt \endgroup \setcounter{footnote}{0} }
%\makeatother % End of region containing @ commands
\usepackage{enumerate}
\renewcommand{\labelenumi}{(\alph{enumi})} 		% Use letters for enumerate

% Shortcuts to Named Objects
\newcommand{\gfnc}{\textsc{Green}'s function} 		% call with {} like \gfnc{} to ensure following whitespace
\newcommand{\gfncs}{\textsc{Green}'s functions} 		% call with {} like \gfnc{} to ensure following whitespace
\newcommand{\cgfnc}{\textsc{Green}'s Function} 		% call with {} like \gfnc{} to ensure following whitespace
\newcommand{\cgfncs}{\textsc{Green}'s Functions} 		% call with {} like \gfnc{} to ensure following whitespace
\newcommand{\hamil}{\textsc{Hamilton}ian}
\newcommand{\rash}{\textsc{Rashba}}
\newcommand{\sdg}{\textsc{Schr\"odinger} equation}
\newcommand{\lanbform}{\textsc{Landauer-B\"uttiker} formalism}
\newcommand{\clanbform}{\textsc{Landauer-B\"uttiker} Formalism}

% Vectors and matrices
\let\vaccent=\v 					% rename builtin command \v{} to \vaccent{}
\renewcommand{\v}[1]{\ensuremath{\boldsymbol{#1}}} 	% for vectors
\newcommand{\mat}[1]{\ensuremath{\boldsymbol{#1}}} 	% for matrices
\newcommand{\uv}[1]{\ensuremath{\mathbf{\hat{#1}}}} 	% for unit vector
\newcommand{\abs}[1]{\left| #1 \right|} 		% for absolute value
\newcommand{\avg}[1]{\left< #1 \right>} 		% for average
\providecommand{\abs}[1]{\lvert#1\rvert}
\providecommand{\norm}[1]{\lVert#1\rVert}

% Derivatives: 
\let\underdot=\d 					% rename builtin command \d{} to \underdot{}
\renewcommand{\d}[2]{\frac{d #1}{d #2}} 		% for derivatives
\newcommand{\dd}[2]{\frac{d^2 #1}{d #2^2}} 		% for double derivatives
\newcommand{\pd}[2]{\frac{\partial #1}{\partial #2}} 	% for partial derivatives
\newcommand{\pdd}[2]{\frac{\partial^2 #1}{\partial #2^2}} 	% for double partial derivatives
\newcommand{\pdc}[3]{\left( \frac{\partial #1}{\partial #2} \right)_{#3}} % for thermodynamic partial derivatives

%Dirac style
\newcommand{\ket}[1]{\left| #1 \right>} 	% for Dirac kets
\newcommand{\bra}[1]{\left< #1 \right|} 	% for Dirac bras
\newcommand{\braket}[2]{\left< #1 \vphantom{#2} \right| \left. #2 \vphantom{#1} \right>} % for Dirac brackets
\newcommand{\expv}[1]{\left< #1 \right>} 	% for expectation value
\newcommand{\matrixel}[3]{\left< #1 \vphantom{#2#3} \right| #2 \left| #3 \vphantom{#1#2} \right>} % for Dirac matrix elements

%Gradient and curls
\newcommand{\grad}[1]{\v{\nabla} #1} 		% for gradient
\let\divsymb=\div 				% rename builtin command \div to \divsymb
\renewcommand{\div}[1]{\v{\nabla} \cdot #1} 	% for divergence
\newcommand{\curl}[1]{\v{\nabla} \times #1}	% for curl

\let\baraccent=\= 				% rename builtin command \= to \baraccent
\renewcommand{\=}[1]{\stackrel{#1}{=}} 		% for putting numbers above =
% \newtheorem{prop}{Proposition}
% \newtheorem{thm}{Theorem}[section]
% \newtheorem{lem}[thm]{Lemma}
\theoremstyle{definition}
\newtheorem{dfn}{Definition}
\newtheorem{algo}{Algorithm}
\theoremstyle{remark}
\newtheorem*{rmk}{Remark}

%%%%%%%%%%%%%% STYLE %%%%%%%%%%%%%%%
%\geometry{hmargin={1in,1in},vmargin={2.0in,1.5in}}
% \usepackage{fancyhdr}
% \pagestyle{fancy}
% % \renewcommand{\chaptermark}[1]{\markboth{\MakeUppercase{\chaptername} \ \thechapter. \ \textbf{#1}}{}}
% % \renewcommand{\sectionmark}[1]{\markright{\textbf{\thesection \ #1}}}
% \fancyhead[RE,LO]{}
% % \fancyhead[RO]{\rightmark}
% % \fancyhead[LE]{\leftmark}
% \fancyfoot{}
% % \fancyfoot[RO,LE]{\thepage}
% \renewcommand{\footrulewidth}{0.4pt}
% % \renewcommand{\headrulewidth}{0.4pt}
% 
% % \fancypagestyle{plain}{%
% % \fancyhf{}
% \fancyfoot[C]{\thepage}
% % \renewcommand{\headrulewidth}{0pt}
% % \renewcommand{\footrulewidth}{0.4pt}}


% configure subfig
\captionsetup[subfloat]{position=top,labelformat=simple,listofformat=subsimple,singlelinecheck=false}
\def\thesubfigure{(\alph{subfigure})} 
%
%% url
%\makeatletter
%\def\url@leostyle{%
%  \@ifundefined{selectfont}{\def\UrlFont{\sf}}{\def\UrlFont{\small\ttfamily}}}
%\makeatother
%\urlstyle{leo}
%
%
\numberwithin{equation}{chapter}
%
%%%%%%%%%%%%%%%% CODE LISTING SETTINGS %%%%%%%%%%%%%%%
% \usepackage{listings}
%
%\lstset{
%frame=tb,
%framesep=5pt,
%tabsize=2,
%basicstyle=\scriptsize\ttfamily,
%showstringspaces=false,
%stringstyle=\color{red},
%commentstyle=\color{gray},
%breaklines=true,
%numbers=left,
%numbersep=5pt
%%numberstyle=\ttfamily,
%%keywordstyle=\color{green},
%%identifierstyle=\color{blue},
%%xleftmargin=5pt,
%%xrightmargin=5pt,
%%aboveskip=\bigskipamount,
%%belowskip=\bigskipamount,
%%rulecolor=\color{Gray},
%%numberstyle=\color{blue},
%}

%\lstdefinelanguage{JavaScript} {
%morekeywords={
%break,const,continue,delete,do,while,export,for,in,function,
%if,else,import,in,instanceOf,label,let,new,return,switch,this,
%throw,try,catch,typeof,var,void,with,yield
%},
%sensitive=false,
%morecomment=[l]{//},
%morecomment=[s]{/*}{*/},
%morestring=[b]",
%morestring=[d]'
%}
%
%% MACROS
%\include{macros}

% ***********************************************************
% ********************** END HEADER *************************
% ***********************************************************
