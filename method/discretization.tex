A practical way to perform the necessary inversion of the \hamil{} to obtain the \gfnc{} is to leverage numerics to obtain an approximation to the solution for
\begin{equation}
	\hat{G} = \left[E-\hat{H} \right]^{-1}\text{.}
  \label{eqn:greensfnccompmethods}
\end{equation}
Spin-interactions are neglected to outline the discretization method. They are however included in an analogous way. To use numerical methods to find the inverse in \cref{eqn:greensfnccompmethods} the system has to be discretized. The energy $E$ is easily discretized by multiplication with the appropriate identity matrix. The choice of space and basis to discretize the \hamil{} is in principle arbitrary but significant advantages can be achieved if a discretization method suitable to the system is used. Due to the cubic zincblende crystal structure of the 2DEG host materials, like GaAs or InAs, it is convenient to discretize on a square lattice.
The procedure will be outlined for the \hamil{} in the effective mass $m^*$ approximation
\begin{equation}
  \hat{H}=-\frac{\hbar^{2}}{2m^*}\left[\pdd{}{x}+\pdd{}{y}\right]+U(x,y)\ .
    \label{eqn:Hamil}
\end{equation}
To achieve a discrete matrix-representation of the \hamil{}, self-energies and \gfnc{} one can expand the equations in any localized basis functions like Muffin-Tin orbitals, \textsc{Wannier} functions, or \textsc{Dirac} distributions giving rise to a so-called tight-binding \hamil{} limiting the interaction e.g. to nearest neighbours. In the following a coordinate in discrete space will be refered to as a lattice point regardless of the nature and space of the localized function.
Using a \textsc{Dirac} delta distribution basis amounts to replacing the continuous function with it's values at the respective points in space \cite{JApplPhys.92.3730}.
\begin{equation}
  \int \delta (\v{r}-\v{r}') f(\v{r}) \mathrm{d}\v{r} = f(\v{r}')\ .
  \label{eqn:deltabasis}
\end{equation}
In order to study the influence of geometry on the transport properties of a 2DEG a real-space approach with \textsc{Dirac} delta distributions is chosen. Real-space discretization leads to the replacement of continuous real space $x$ and $y$ directions by an infinite net of lattice points. Here the same spacing $a$ is assumed in both directions. The lattice points are located in the continuous space at the coordinates $x=i*a$ and $y=j*a$ with $i,j$ being integers and $a$ the lattice spacing. The \hamil{} now operates on functions defined on discrete points in space obtained by
\begin{equation}
  T_{i,j} \leftrightarrow T(x=ia,y=ja) \mbox{ and } U_{i,j}\leftrightarrow U(x=ia,y=ja)\ .
  \label{FunctionDescrete}
\end{equation}
With $T_{ij}$ describing the state of the system and $U_{ij}$ being the discrete version of the lateral potential.
The \hamil{} reads
\begin{equation}
  \left[\v{H}T\right]_{x=ia,y=ja}=-\frac{\hbar^{2}}{2m^*}\left[\pdd{T}{x}+\pdd{T}{y}\right]_{x=ia,y=ja}+U_{i,j}T_{i,j}\ .
  \label{DiscreteHamil}
\end{equation}
Using the finite-differences scheme and assuming small $a$ the first derivative in each direction is approximated 
\begin{equation}
  \left[\pd{T}{x}\right]_{x=(i+1/2)a} \sim  \frac{1}{a}\left(T_{i+1}-T_{i}\right)\ .
  \label{ApproximateFirstDerivative}
\end{equation}
In this case the subscript denotes evaluation at a point between two lattice points $x=(i+1/2)$. By applying the concept of finite differeneces a second time for the second derivative one arrives at
\begin{align}
  \left[\pdd{T}{x}\right]_{x=(i+1/2)a} &\sim \frac{1}{a}\left( \left[\pd{T}{x}\right]_{x=(i+1/2)a}-\left[\pd{T}{x}\right]_{x=(i-1/2)a} \right) \notag \\
  &\sim \frac{1}{a^2}\left(T_{i+1}-2T_{i}+T_{i-1}\right)\ .
  \label{ApproximateSecondDerivative}
\end{align}
Following Nicoli\'c \cite{Nikolic2010} the bras $\bra{m}$ and kets $\ket{n}$ denote localized states at the lattice-points. The finite-difference aproximation is  expressend in the second quantization formalism with a point $(i,j) = \v{m}$ as
\begin{align}
	\bra{\v{m}} \pd{}{x} \ket{\v{n}} &= \frac{\braket{\v{m}+1}{\v{n}}-\braket{m-1}{n}}{2a}\notag\\ 
				&= \frac{\delta_{\v{n},\v{m}+1}-\delta_{\v{n},\v{m}-1}}{2a}\ .
	\label{eqn:finitedifffirstderivative}
\end{align}
and
\begin{align}
	\bra{\v{m}} \dd{}{x} \ket{\v{n}} &= \frac{\braket{\v{m}+1}{\v{n}}-2\braket{\v{m}}{\v{n}}+\braket{\v{m}-1}{\v{n}}}{a^2} \\
				 &= \frac{\delta_{\v{n},\v{m}+1}-2\delta_{\v{n},\v{m}}+\delta_{\v{n},\v{m}-1}}{a^2}\ .
	\label{eqn:finitediffsecondderivative}
\end{align}
With the help of these relations any discrete one-particle operators can be expressed in second quantization
\begin{align}
	\hat{A} = \sum_{\v{m},\v{n}} \bra{\v{m}}\hat{A}\ket{\v{n}} \hat{c}^{\dagger}_\v{m} \hat{c}_\v{n}.
 \label{eqn:singleparticleopinsecondquantization}
\end{align}
with $\hat{c}^{\dagger}_\v{m} \hat{c}_\v{n} $ being creation and annihilation operators in chosen localized basis.
Inserting \cref{eqn:finitedifffirstderivative}, \ref{eqn:finitediffsecondderivative} and \ref{eqn:singleparticleopinsecondquantization} leads to the tight-binding version of the \hamil{}
\begin{align}
	\hat{H} = \sum_{\v{m},\sigma} \epsilon_{\v{m}} \hat{c}^{\dagger}_{\v{m}\sigma} \hat{c}_{\v{m},\sigma} +
	\sum_{\v{m},\v{m}',\sigma} t_{\v{m},\v{m}'} \hat{c}^{\dagger}_{\v{m}\sigma} \hat{c}_{\v{m}',\sigma}\ .
	\label{eqn:discretizedeffectivemassHamil}
\end{align}
The $\epsilon$ account for local potentials in the from of $U(x,y)$ like lateral confinement or atomic disorder.
The spin independend hopping parameter $t_{\v{m}\v{m'}}$ is given as
\begin{align}
t_{\v{m}\v{m'}} = \left\{ \begin{array}[c]{cl} -t_0 & \text{if } \v{m} = \v{m} \pm a \cdot \v{e}_{x,y} \\ 0 & \text{else} \end{array} \right.\ .
	\label{eqn:t0hopping}
\end{align}
The hopping parameter $t_0 = \hbar^2/2m^*a^2$ describes the probability of one electron propagating to a neighbouring lattice point. Introducing the discretized \rash{} \hamil{}
\begin{align}
	\hat{H} = \sum_{\v{m},\sigma} \epsilon_{\v{m}} \hat{c}^{\dagger}_{\v{m}\sigma} \hat{c}_{\v{m},\sigma} +
	\sum_{\v{m},\v{m'},\sigma}  \hat{c}^{\dagger}_{\v{m}\sigma} \mat{t}_{\v{m},\v{m'},\sigma,\sigma '} \hat{c}_{\v{m'},\sigma}\ .
	\label{eqn:discretizedrashbahamil}
\end{align}
Now the hopping parameter $\mat{t}_{\v{m},\v{m'},\sigma,\sigma '}$ is a non-trivial $2 \times 2$ matrix with distinct coefficients depending on hopping direction:
\begin{align}
	\mat{t}_{\v{m}\v{m'},\sigma,\sigma '} = \left\{ \begin{array}[c]{cl} -t_0\mat{1}_S - i t_{SO} \mat{\sigma}_y & \text{if } \v{m} = \v{m} \pm a \v{e}_{x} \\
		-t_0\mat{1}_S + i t_{SO} \mat{\sigma}_x & \text{if } \v{m} = \v{m} \pm a \v{e}_{y} \end{array} \right. 
	\label{eqn:tsohopping}
\end{align}
In the matrix representation each element of the spin-less \hamil{} becomes a $2 \times 2$ matrix of itself increasing the number of matrix-elements four-fold.
\begin{figure}[h!]
\centering
\begin{align*}
\mat{H}=
\renewcommand*{\arraystretch}{1.5}
\begin{pmatrix} 
\ddots& & & & & & & & & \\
& \mat{H}_p& \mat{V}_p& & & && &&\\
& \mat{V}^{\dagger}_p& \mat{H}_p& \mat{H}_{0,1}& &\ddots& & &\\
& &\mat{H}_{1,0}&\mat{H}_{1,1} &\mat{H}_{1,2} & &0 & &\\
& & &\mat{H}_{2,1} & &\ddots & &\ddots &\\
& & & &\ddots &\mat{H}_{N-2,N-2} &\mat{H}_{N-2,N-1} & &\\
& & \ddots& &\mat{H}_{N-1,N-2} &\mat{H}_{N-1,N-1} &\mat{H}_{N-1,N} &\\
& & &0 & & &\mat{H}_{N,N-1} &\mat{H}_q &\mat{V}_q\\
& & & &\ddots & & &\mat{V}^{\dagger}_q &\mat{H}_q\\
& & & & & & && &\ddots
\end{pmatrix}
% \includegraphics[width=0.9\textwidth]{images/blocktridiagonal}
\end{align*}
\caption{Block tridiagonal matrix of the tight-binding \hamil{}, $\mat{H}_{i,i}$ denotes the \hamil{} of layer $i$ and $\mat{H}_{i,i+1},\mat{H}_{i+1,i}$the interaction between layers. The matrices $\mat{H}_p,\mat{V}_p,\mat{H}_q,\mat{V}_q$ denote the \hamil s and coupling matrices of leads $p$ and $q$ respectively.)}
\label{fig:blocktridiagonal}
\end{figure}
The on-site energies $\epsilon_{\v{m}}$ in this \hamil{} are
\begin{equation}
\epsilon_{\v{m}} = 4t_0 + U_{\v{m}}\ .
\end{equation}
The potential offset $4t_0$ arises naturally in the process of approximation with finite differences. Because of the offset the dispersion relation for free electrons ($V_{\v{m}}=0$) in an infinite empty lattice reads
\begin{equation}
\epsilon = 4t_0(1-cos(\abs{\v{k}}a))\ .
\end{equation}
The tight-binding model is a good approximation if the lattice spacing is below the \textsc{Fermi}-wavelength
\begin{equation}
\abs{\v{k}}a \ll  2\pi \Leftrightarrow a \ll \lambda_F = \frac{2\pi}{\abs{\v{k}}} \ .
\end{equation}
In this regime the dispersion relation is approximately parabolic as in the continuum case \cite{Metalidis2007Thesis}.
For simple systems with colinear leads the matrix representaion in the given approximation of the \hamil{} takes a natural block tridiagonal form. Each block on the main diagonal belongs to a set of lattice points called a layer. All interactions within the layer correspond to matrix elements within the block. Interactions across layers are found in blocks on the first off-diagonal, see \cref{fig:blocktridiagonal} \cite{AnLunNik2008}.
