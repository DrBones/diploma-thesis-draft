A practical way to actually perform the necessary inversion of of the \hamil{} to obtain the Green's Function is to leverage numerics to obtain an approximation to the solution.
Consider the following \hamil{}:

\begin{equation}
	\hat{H} = -\frac{\hbar^{2}}{2m^*}\left(\pdd{}{x}+\pdd{}{y}\right) \otimes \mat{I}_S + U(x,y) \otimes \mat{I}_S
  \label{eqn:effectivemasshamiltonian}
\end{equation}

To use numerical methods the system has to be discretized in some way. The choice of basis is in principle arbitrary but significant advantages can be achieved if a discretization suitable to the system is used.
Due to the cubic zincblende crystal structure of the materials as GaAs and InAs which constitute the environment of the 2DEG it is convenient to discretize on a square lattice.
To achieve a discrete matrix-representation of the \hamil{}, Self-Energies and Green's function one can expand the equations in any localized basis functions like Muffin-Tin orbitals, Wannier functions, or \textsc{Dirac} distributions which gives rise to a so called ``Tight Binding'' \hamil{}. Discretization means that the continuous real space in X and Y directions will be replaced by an infinite net of lattice points. The same spacing $a$ is assumed in both directions. The lattice points are located in the pre-discrete space at the coordinates $x=i*a$ and $y=j*a$ with $i,j$ being integers and $a$ the lattice spacing. The \hamil{} now operates on functions defined on discrete points in space obtained by:
\begin{equation}
  T_{i,j} \leftrightarrow T(x=ia,y=ja) \mbox{ and } U_{i,j}\leftrightarrow U(x=ia,y=ja)
  \label{FunctionDescrete}
\end{equation}
With $T_{ij}$ describing the state of the system and $U_{ij}$ being the discrete version of the lateral potential.
The \hamil{} is expressed accordingly:
\begin{equation}
  \left[\v{H}T\right]_{x=ia,y=ja}=-\frac{\hbar^{2}}{2m^*}\left[\pdd{T}{x}+\pdd{T}{y}\right]_{x=ia,y=ja}+U_{i,j}T_{i,j}
  \label{DiscreteHamil}
\end{equation}
Using finite differences scheme and assuming small $a$ the first derivative in each direction is approximated by:
\begin{equation}
  \left[\pd{T}{x}\right]_{x=(i+1/2)a} \sim  \frac{1}{a}\left(T_{i+1}-T_{i}\right)
  \label{ApproximateFirstDerivative}
\end{equation}
In this case the subscript denotes evaulation at that point. By applying the concept a second time for the second derivative:
\begin{align}
  \left[\pdd{T}{x}\right]_{x=(i+1/2)a} &\sim \frac{1}{a}\left( \left[\pd{T}{x}\right]_{x=(i+1/2)a}-\left[\pd{T}{x}\right]_{x=(i-1/2)a} \right) \\
  &\sim \frac{1}{a^2}\left(T_{i+1}-2T_{i}+T_{i-1}\right)
  \label{ApproximateSecondDerivative}
\end{align}
If executed for both dimensions individually the approximated \hamil{} is written as:
\begin{equation}
  \left[\v{H}T\right]_{x=ia,y=ja}= \left(U_{i}\right)
  \label{<++>}
\end{equation}<++>

move to second discretization following Nikloic
\begin{align}
	\bra{m} \d{}{x} \ket{n} = \frac{\braket{m+1}{n}-\braket{m-1}{n}}{2a} = \frac{\delta_{n,m+1}-\delta_{n,m-1}}{2a}
	\label{eqn:finitedifffirstderivative}
\end{align}
and
\begin{align}
	\bra{m} \dd{}{x} \ket{n} = \frac{\braket{m+1}{n}-2\braket{m}{n}+\braket{m-1}{n}}{a^2} = \frac{\delta_{n,m+1}-2\delta_{n,m}+\delta_{n,m-1}}{a^2}
	\label{eqn:finitediffsecondderivative}
\end{align}
discrete one particle operators in second quantization:
\begin{align}
	\hat{A} = \sum_{m,n} \bra{m}\hat{A}\ket{n} \hat{c}^+_m \hat{c}_n
 \label{eqn:singleparticleopinsecondquantization}
\end{align}

with $\hat{c}^+_m \hat{c}_n $ being creation and annihilation operators in chosen localized basis.
taking \ref{eqn:finitedifffirstderivative}, \ref{eqn:finitediffsecondderivative} and \ref{eqn:singleparticleopinsecondquantization} one arrives at:

\begin{align}
	\hat{H} = \sum_{\v{m},\sigma} \epsilon_{\v{m}} \hat{c}^+_{\v{m}\sigma} \hat{c}_{\v{m},\sigma} +
	\sum_{\v{m},\v{m'},\sigma} t_{\v{m},\v{m'}} \hat{c}^+_{\v{m}\sigma} \hat{c}_{\v{m'},\sigma}
	\label{eqn:discretizedeffectivemassHamil}
\end{align}
The $\epsilon$ account for any localized potential in the from of $U(x,y)$ like lateral confinement or atomic disorder.
The spin independend hopping parameter $t_{\v{m}\v{m'}}$ is with $t_0 = \hbar^2/2m^*a^2$  given as:
\begin{align}
t_{\v{m}\v{m'}} = \left\{ \begin{array}[c]{cl} -t_0 & \text{if } \v{m} = \v{m} \pm a \cdot \v{e}_{x,y} \\ 0 & \text{else} \end{array} \right.
	\label{eqn:t0hopping}
\end{align}

if extended to include spin-orbit interactions one finds the discretized Rashba \hamil{} as:
\begin{align}
	\hat{H} = \sum_{\v{m},\sigma} \epsilon_{\v{m}} \hat{c}^+_{\v{m}\sigma} \hat{c}_{\v{m},\sigma} +
	\sum_{\v{m},\v{m'},\sigma}  \hat{c}^+_{\v{m}\sigma} \mat{t}_{\v{m},\v{m'},\sigma,\sigma '} \hat{c}_{\v{m'},\sigma}
	\label{eqn:discretizedrashbahamil}
\end{align}

Now the hopping parameter $\mat{t}_{\v{m},\v{m'},\sigma,\sigma '}$ is a non-trivial $2 \times 2$ matrix with distinct coefficients depending on hopping direction:
\begin{align}
	\mat{t}_{\v{m}\v{m'},\sigma,\sigma '} = \left\{ \begin{array}[c]{cl} -t_0\mat{I}_S - i t_{SO} \gmat{\sigma}_y & \text{if } \v{m} = \v{m} \pm a \v{e}_{x} \\
		-t_0\mat{I}_S + i t_{SO} \gmat{\sigma}_x & \text{if } \v{m} = \v{m} \pm a \v{e}_{y} \end{array} \right.
	\label{eqn:tsohopping}
\end{align}

picture or equation like Wimmer page 184 B.1
