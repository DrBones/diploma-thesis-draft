Although a discrete representation of the \hamil{} is found the matrix inversion is not trivial. The device is an open system and the incorporation of in- and outflow via the leads that stretch to infinity results in an infinite dimensional matrix.
Simple truncation of the matrices under consideration would effectively describe a closed system. Via the concept of the self-energy however the influence of the leads may be accurately "projected" onto the device.
The technique will be outlined following \textsc{Datta} for a system with only one contact as depicted in figure \ref{fig:selfenergy} but can be applied to an arbitray number of leads. \begin{figure}[h!]
\centering
\includegraphics[scale=0.5]{images/selfenergy}
\caption{Separation of conductor and lead}
\label{fig:selfenergy}
\end{figure}

\subsection{The Self Energy}
Consider the following partition of the infinite dimensional \gfnc{} matrix:
\begin{align}
  \begin{bmatrix}
  \mat{G}_{L} & \mat{G}_{L/C}\\
  \mat{G}_{C/L} & \mat{G}_{C}
  \end{bmatrix}
  =
  \begin{bmatrix}
  (E+i\eta)\mat{1}_{L} - \mat{H}_L  & -\v{\tau}^+ \\
	-\v{\tau} & E\mat{1} - \mat{H}_C
  \end{bmatrix}^{-1}
  \label{eqn:greendivided}
\end{align}
Here $G_L$ is the \gfnc{} of the isolated lead, $G_{L/C}$ and $G_{C/L}$ are \gfnc{} between lead and conductor, all of them are row and/or column infinite. $G_L$ is the finite \gfnc{} of the now isolated conductor. The coupling matrices $\tau$ are only nonzero on the interface of lead and conductor cf. figure \ref{fig:selfenergy}.
Basic agebra obtains\cite{Datta1997}:
\begin{align}
\mat{G}_C& = \left[E-\mat{H}_C - \v{\tau}^+ g_L^R \v{\tau}\right]^{-1} 
\label{eqn:finitegreensfunction}
\end{align}
with the so called retarded lead surface \gfnc{}
\begin{align}
g_L^R &= [(E+i\eta)\mat{1}_L - \mat{H}_L]^{-1}
\label{eqn:leadsurfacegfnc}
\end{align}
\todo[noline]{How can self energy be used for scattering, etc}
The following discussion hold for non-interacting systems i.e  when $\Sigma$ only holds contributions from the lead.

\subsection{Lead Surface \cgfnc s}
\todo[noline]{Finish, formulas are missing and explain what i used}
Even with this separation the direct evaluation of eqn. (\ref{eqn:leadsurfacegfnc}) is still unfeasible as it still includes $\mat{H}_L$ which is infinite. Fortunately there exist alternative ways of obtaining the leads \gfnc{}.
If translation invariance along the lead exists the so called \textsc{Ando} method \cite{PhysRevB.44.8017} computes the the surface \gfnc{} for a homogenuous lead consiting of repeating suppercells, i.e the smallest layer that exhibits periodicity.\par
For simple systems without fields it can even be calculated analytically \cite{Datta1997} or for leads including fields and further interactions semi analytically via the \emph{eigenstate decomposition} of their \emph{companion matrix} or \emph{transfer matrix} \cite{PhysRevB.55.5266} \cite{PhysRevB.66.205319}.
\begin{align}
  \mat{C} =
  \begin{bmatrix}
  \mat{0}  &\mat{I} \\
  -\mat{H}_{+}^{-1}\mat{H}_{-} &\mat{H}_{+}^{-1} (E\mat{1} - \mat{H}_{\perp})
  \end{bmatrix}
  \label{eqn:greendivided}
\end{align}
Here $\mat{H}_{\pm}$ signify matrices governing the interaction between slices with \hamil{} $\mat{H}_{\perp}$
For non-homogenous leads one can apply a continued fraction method \cite{Velev2004} that connects \gfnc{} from one layer to neighbouring layers and repeats that process until the interactions betweenlayers becomes negligible.\par
As the leads in this work need to include fields and spin-orbit effects analytical methods did not suffice. With the appropriate gauge one can ensure invariance along the lead and therefore the a modification of the eigenstate decomposition method is employed.\par
The lead surface \gnfc{} is expanded in eigenstates of an \emph{invariant subspace} which yields higher numerical stability as expansions in eigenstates of the full \hamil{} with the same result.
used eigenstate decomp and schur decomp (wimmer) as more stable if near singular, maybe picture of typical self energy matrix, maybe discussion what physical insights $\Gamma = \Sigma^R - \Sigma^A$ can give (lifetime, etc)

The concept of self energy may also be applied to include the effects of e.g. phonon scattering or electron-electron interactions in an approximate manner.

