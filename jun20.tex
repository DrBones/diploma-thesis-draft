% ***********************************************************
% ******************* PHYSICS HEADER ************************
% ***********************************************************
%\documentclass[11pt]{article} 
% \documentclass[12pt,twoside,a4paper]{article}
\documentclass[12pt,twoside,a4paper]{scrartcl}
\setkomafont{sectioning}{\normalcolor\bfseries}
%%%%%%%%%%%%%% PACKAGE INCLUDES %%%%%%%%%%%%%%%
\usepackage[utf8]{inputenc} %allows input of utf8 characters
\usepackage[american,ngerman]{babel}
\usepackage[T1]{fontenc} %more complete but lower quality than ComMod fontpacke
% \usepackage{ae,aecompl}
\usepackage[tbtags]{mathtools}
%\usepackage{amsmath} % AMS Math Package
\usepackage{amsthm} % Theorem Formatting
\usepackage{amssymb}	% Math symbols such as \mathbb
\usepackage{graphicx} % Allows for eps images
\usepackage{multirow}
\usepackage{multicol} % Allows for multiple columns
\usepackage{color} % well, supplies colors
\usepackage{todonotes}
\usepackage{booktabs}
\usepackage{setspace} %Zeilenabstand
%\onehalfspacing
% \usepackage{mathptmx}
% \usepackage{bbold}
% \usepackage{bold-extra} % Ugly bitmap fonts, but holds bold glyphs!
% \usepackage{kpfonts} % Pretty complete fontpackage but looks 'different'
\usepackage[
    pdftitle={Spin-dependend transport simulations in two-dimensionsal electron systems},    % title
    pdfauthor={Jonas Siegl},     % author
    pdfsubject={Quantum Transport},   % subject of the document
    pdfcreator={Jonas Siegl},   % creator of the document
    pdfproducer={Jonas Siegl}, % producer of the document
    pdfkeywords={Green's Function } {Numerical } {Transport}, % list of keywords
    % linktocpage=true,
    linkcolor=red,          % color of internal links
    citecolor=green,        % color of links to bibliography
    filecolor=magenta,      % color of file links
    urlcolor=cyan,           % color of external links
pagebackref=true]{hyperref}
\usepackage[all]{hypcap} % shifts the link point to the top of tables or figures so it is visible when clicked
%\usepackage{array}
%\usepackage[final]{svninfo}
\usepackage{tabulary}
\usepackage{tabularx}
%\usepackage{ifthen}
%\usepackage[dvips]{graphicx}
%\usepackage[small,bf]{caption}
\usepackage{subfig} 		%enabled the use of subfloats within a float
%\usepackage{textcomp}
%\usepackage{pstricks}
%\usepackage{courier}
%\usepackage{geometry}
%\usepackage{url}
\usepackage{tikz}
\tikzstyle{every picture}+=[remember picture]
\usetikzlibrary{decorations.pathreplacing}
%\usepackage{paralist}
%\usepackage{xspace}
%\usepackage[dvips,letterpaper,margin=0.75in,bottom=0.5in]{geometry}
 % Sets margins and page size
%\pagestyle{empty} % Removes page numbers
%\makeatletter % Need for anything that contains an @ command 
%\renewcommand{\maketitle} % Redefine maketitle to conserve space
%{ \begingroup \vskip 10pt \begin{center} \large {\bf \@title}
%	\vskip 10pt \large \@author \hskip 20pt \@date \end{center}
%  \vskip 10pt \endgroup \setcounter{footnote}{0} }
%\makeatother % End of region containing @ commands
\renewcommand{\labelenumi}{(\alph{enumi})} 		% Use letters for enumerate

% Shortcuts to Named Objects
\newcommand{\gfnc}{\textsc{Green}'s function} 		% call with {} like \gfnc{} to ensure following whitespace
\newcommand{\cgfnc}{\textsc{Green}'s Function} 		% call with {} like \gfnc{} to ensure following whitespace
\newcommand{\hamil}{\textsc{Hamilton}ian}
\newcommand{\rash}{\textsc{Rashba}}
\newcommand{\sdg}{\textsc{Schr\"odinger} equation}
\newcommand{\lanbform}{\textsc{Landauer-B\"uttiker} formalism}
\newcommand{\clanbform}{\textsc{Landauer-B\"uttiker} Formalism}

% Vectors and matrices
\let\vaccent=\v 					% rename builtin command \v{} to \vaccent{}
\renewcommand{\v}[1]{\ensuremath{\boldsymbol{#1}}} 	% for vectors
\newcommand{\mat}[1]{\ensuremath{\boldsymbol{#1}}} 	% for matrices
\newcommand{\uv}[1]{\ensuremath{\mathbf{\hat{#1}}}} 	% for unit vector
\newcommand{\abs}[1]{\left| #1 \right|} 		% for absolute value
\newcommand{\avg}[1]{\left< #1 \right>} 		% for average
\providecommand{\abs}[1]{\lvert#1\rvert}
\providecommand{\norm}[1]{\lVert#1\rVert}

% Derivatives: 
\let\underdot=\d 					% rename builtin command \d{} to \underdot{}
\renewcommand{\d}[2]{\frac{d #1}{d #2}} 		% for derivatives
\newcommand{\dd}[2]{\frac{d^2 #1}{d #2^2}} 		% for double derivatives
\newcommand{\pd}[2]{\frac{\partial #1}{\partial #2}} 	% for partial derivatives
\newcommand{\pdd}[2]{\frac{\partial^2 #1}{\partial #2^2}} 	% for double partial derivatives
\newcommand{\pdc}[3]{\left( \frac{\partial #1}{\partial #2} \right)_{#3}} % for thermodynamic partial derivatives

%Dirac style
\newcommand{\ket}[1]{\left| #1 \right>} 	% for Dirac kets
\newcommand{\bra}[1]{\left< #1 \right|} 	% for Dirac bras
\newcommand{\braket}[2]{\left< #1 \vphantom{#2} \right| \left. #2 \vphantom{#1} \right>} % for Dirac brackets
\newcommand{\matrixel}[3]{\left< #1 \vphantom{#2#3} \right| #2 \left| #3 \vphantom{#1#2} \right>} % for Dirac matrix elements

%Gradient and curls
\newcommand{\grad}[1]{\v{\nabla} #1} 		% for gradient
\let\divsymb=\div 				% rename builtin command \div to \divsymb
\renewcommand{\div}[1]{\v{\nabla} \cdot #1} 	% for divergence
\newcommand{\curl}[1]{\v{\nabla} \times #1}	% for curl

\let\baraccent=\= 				% rename builtin command \= to \baraccent
\renewcommand{\=}[1]{\stackrel{#1}{=}} 		% for putting numbers above =
\newtheorem{prop}{Proposition}
\newtheorem{thm}{Theorem}[section]
\newtheorem{lem}[thm]{Lemma}
\theoremstyle{definition}
\newtheorem{dfn}{Definition}
\theoremstyle{remark}
\newtheorem*{rmk}{Remark}

%%%%%%%%%%%%%% STYLE %%%%%%%%%%%%%%%
%\geometry{hmargin={1in,1in},vmargin={2.0in,1.5in}}
\usepackage{fancyhdr}
\pagestyle{fancy}
% \renewcommand{\chaptermark}[1]{\markboth{\MakeUppercase{\chaptername} \ \thechapter. \ \textbf{#1}}{}}
% \renewcommand{\sectionmark}[1]{\markright{\textbf{\thesection \ #1}}}
\fancyhead[RE,LO]{}
% \fancyhead[RO]{\rightmark}
% \fancyhead[LE]{\leftmark}
%%\fancyfoot{}
%%\fancyfoot[RO,LE]{\thepage}
\renewcommand{\footrulewidth}{0.4pt}
% \renewcommand{\headrulewidth}{0.4pt}

\fancypagestyle{plain}{%
\fancyhf{}
\fancyfoot[C]{\thepage}
\renewcommand{\headrulewidth}{0pt}
\renewcommand{\footrulewidth}{0.4pt}}


% configure subfig
\captionsetup[subfloat]{labelformat=simple,listofformat=subsimple}
\def\thesubfigure{(\alph{subfigure})} 
%
%% url
%\makeatletter
%\def\url@leostyle{%
%  \@ifundefined{selectfont}{\def\UrlFont{\sf}}{\def\UrlFont{\small\ttfamily}}}
%\makeatother
%\urlstyle{leo}
%
%
\numberwithin{equation}{section}
%
%%%%%%%%%%%%%%%% CODE LISTING SETTINGS %%%%%%%%%%%%%%%
%\usepackage{listings}
%
%\lstset{
%frame=tb,
%framesep=5pt,
%tabsize=2,
%basicstyle=\scriptsize\ttfamily,
%showstringspaces=false,
%stringstyle=\color{red},
%commentstyle=\color{gray},
%breaklines=true,
%numbers=left,
%numbersep=5pt
%%numberstyle=\ttfamily,
%%keywordstyle=\color{green},
%%identifierstyle=\color{blue},
%%xleftmargin=5pt,
%%xrightmargin=5pt,
%%aboveskip=\bigskipamount,
%%belowskip=\bigskipamount,
%%rulecolor=\color{Gray},
%%numberstyle=\color{blue},
%}

%\lstdefinelanguage{JavaScript} {
%morekeywords={
%break,const,continue,delete,do,while,export,for,in,function,
%if,else,import,in,instanceOf,label,let,new,return,switch,this,
%throw,try,catch,typeof,var,void,with,yield
%},
%sensitive=false,
%morecomment=[l]{//},
%morecomment=[s]{/*}{*/},
%morestring=[b]",
%morestring=[d]'
%}
%
%% MACROS
%\include{macros}

% ***********************************************************
% ********************** END HEADER *************************
% ***********************************************************

% INFO
\author{Jonas Siegl}

\begin{document}
% \selectlanguage{american}

\section{Boundary treatment}
This sectin will explain the theoretical background for incorporating the infinite leads in terms of self energy matrices, the content will closely resemble
explanations presented by Datta in ``From Atom to Transistor'' blah blah
\subsection{Closed System}
Beause system is open and the flux of electrons is the measured effect closed boundary conditions do not accurately model
the desired phenomena as they would impose a restriction  that would lead to discretized energies in that dimension.
Even the method of periodic boundary conditions does not lead to an improvemente in that regard. Although it does get rid of
boundaryeffects, the composed loop is still a closed system and leads therefore to inaccurate energy treatment.
\subsection{Open System}
In principle the modeling of open systems is of no conceptual difficulty. The Hamiltonian governing the device will simply be
extended over the Source and Drain Contacts and leads.
\begin{equation}
	G_T = \left[E\v{I}_T - \v{H}_T \right]^{-1}
	\label{Total Hamiltonian}
\end{equation}
The usefullness of this in actual numeric computations is limited as some kind of discretization must be applied rendering
the Hamiltonian-Matrix infinite because auf the infinite number of Nodes taking the infinte leads.
Split up the Green's function into Reservoir and device regions
\begin{equation}
  \begin{bmatrix}
	  E\mabol{1}_R - \mabol{H}_R +i\eta\mabol{1}_R 	& -\gv{\tau}^+ \\
	  -\gv{\tau} 				& E\mabol{1} - \mabol{H}
  \end{bmatrix}
  \begin{pmatrix*}[c]
	  \gv{\phi}_R + \gv{\chi}\\
    \gv{\psi}
  \end{pmatrix*}
  = 
  \begin{pmatrix*}[c]
    \v{S}_R \\
    \v{0}
  \end{pmatrix*}
  \label{green divided}
\end{equation}

In This expression the $\mabol{H}_R$ is still infinite.
Using the solution to the Schr\"odinger-equation of the reservoir:
\begin{equation}
	\left[E\mabol{1}_R - \mabol{H}_R +i\eta\mabol{1}_R\right]\left(\gv{\phi}_R\right) = \left(\v{S}_R\right)
	\label{schroedinger contact}
\end{equation}
to write the matrix equation:
\begin{align}
	\left[E\mabol{1}_R - \mabol{H}_R +i\eta\mabol{1}_R\right]\left(\gv{\chi}\right)-\left[\gv{\tau}^{+}\right]\left(\gv{\chi}\right) &= \left(\v{0}\right) \\
	\left[E\mabol{1} - \mabol{H} \right] -\left[\gv{\tau}\right]\left(\gv{\chi}\right) &= \left[\gv{\tau}\right]\left(\gv{\phi}_R\right)
	\label{separated green's function}
\end{align}

by simple matrix algebra one is able to express $\left(\chi\right)$ in terms of $\left(\psi\right)$:
\begin{equation}
	\left(\chi\right) = \left[E\mabol{1}_R - \mabol{H}_R +i\eta\mabol{1}_R\right]^{-1} \gv{\tau}^+ \left(\psi\right)
	\label{chi}
\end{equation}

let 
\begin{equation}
	\v{G}_R \equiv \left[E\mabol{1}_R - \mabol{H}_R +i\eta\mabol{1}_R\right]^{-1}
	\label{Green's function reservoir}
\end{equation}
and substitute into second equation to obtain:
\begin{equation}
	\left[ E\mabol{1} - \mabol{H} -\gv{\Sigma} \right] \left(\gv{\psi}\right) = \left(\v{S}\right)
	\label{}
\end{equation}
with
\begin{equation}
	\gv{\Sigma} \equiv \gv{\tau} \v{G}_R \gv{\tau}^{+} \quad \v{S} \equiv \gv{\tau} \gv{\psi}_R
	\label{Sigmadef}
\end{equation}

It seems as we had won nothing as the the Green's Function for the isolated contact $G_R$ is still infinite.
The problem at hand is simplified significantly though as i will elaborate.
\subsection*{Numerical reduction}
Although the matrix of the Green's function is still huge or even infinite, as is the coupling matrix $\gv{\tau}$, for practical purposes
it only connects to a subset of $r$ nodes along the surface elements of the contact adjacent to the device. So it is possible to truncate the coupling matrices
to this subset and define the surface Green's function $\v{g}_R$:
\begin{equation}
	\begin{matrix}
		\gv{\Sigma} &\equiv &\gv{\tau} &\v{g}_R &\gv{\tau}^+\\
		(d \times d) & & (d \times r) &(r \times r) &(r \times d)
	\end{matrix}
	\label{Sigmasubset}
\end{equation}

Now it is only neccessary to find and compute the surface Green's Function on a finite subset of elements in $\mathbb{R}^n$

\subsection*{Analytical reduction}
An alternative way to overcome the problem of infinite elements to compute is to push the area of evaluation into a ``nice'' area of the lead
and calculate the expression for Sigma analytically. This is presented for a semi infinte 2D lead in the Appendix. Here the result:
\begin{equation}
	\gv{\Sigma}(i,j) = -t \sum_{m \in \delta V} \chi_m(p_i) \chi_m(p_j) \exp(ik_m a)
	\label{analytic surface greenfunction}
\end{equation}

I cannot stress enough that the problem of an infinite grid has been reduced to the inversion of a finite matrix with the effects of the
infinite leads folded in \emph{exactly} with the use of the selfenergy matrix $\Sigma$. 
The term selfenergy\ldots explain\ldots.
Explain Selfenergy more general for scattering phonons, etc

\section{Recursive Green's Function Algorithm}
Inversion of matrix is for considered system ($e^4 \times  e^4$) too expensive, so an algorithm that only computes the needed main
diagonal is desireable. The presented algorithm computes the main diagonal blocks. This is still overhead but much faster than
direct inversion.


\section{Physical Observables}

Density matrix in general
\begin{equation}
	\rho = \int_{-\infty}^{+\infty}\frac{dE}{2 \pi} \left[G^n(E)\right]
	\label{densmatrixgeneral}
\end{equation}
In equilibrium:
\begin{equation}
	G^n(E) = [A(E)]f_0(E-\mu)
	\label{correlation function equi}
\end{equation}
Whereas in non-equilibrium:
\begin{equation}
	G^n(E) = [A_1(E)]f_0(E-\mu_1)+[A_2(E)]f_0(E-\mu_2)
	\label{correlation function non-equi}
\end{equation}
\end{document}
